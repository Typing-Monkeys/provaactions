\documentclass[a4paper,12 pt]{report}
\usepackage[T1]{fontenc}
\usepackage[utf8]{inputenc}
\usepackage{lmodern}
\usepackage{listings}
\usepackage{float}
\usepackage{subcaption}
\usepackage{wrapfig}

\usepackage{amsthm}

\usepackage{pgfplots}asdf
\setlength {\marginparwidth }{2cm}
\usepackage{todonotes}
\newcommand{\TODO}[2][]
{\todo[size=\scriptsize, color=red, #1]{#2}}

asdf
asdf
\pgfplotsset{compat=1.18}
% forza le footnote a stare il più in basso possibile
\usepackage[bottom]{footmisc}


%% STILE LISTINGS
%%aaa
\usepackage{xcolor}

\definecolor{codegreen}{rgb}{0,0.6,0}
\definecolor{codegray}{rgb}{0.5,0.5,0.5}
\definecolor{codepurple}{rgb}{0.58,0,0.82}
\definecolor{backcolour}{rgb}{0.95,0.95,0.92}

\lstdefinestyle{mystyle}{
    backgroundcolor=\color{backcolour},   
    commentstyle=\color{codegreen},
    keywordstyle=\color{magenta},
    numberstyle=\tiny\color{codegray},
    stringstyle=\color{codepurple},
    basicstyle=\ttfamily\footnotesize,
    breakatwhitespace=false,         
    breaklines=true,                 
    captionpos=b,                    
    keepspaces=true,                 
    numbers=left,                    
    numbersep=5pt,                  
    showspaces=false,                
    showstringspaces=false,
    showtabs=false,                  
    tabsize=2
}

\lstset{style=mystyle}

%% -----

% mostra le subsubsection nell'indice
\setcounter{tocdepth}{3}
\setcounter{secnumdepth}{3}

% Resetta la numerazione dei chapter quando
% una nuova part viene creata
\makeatletter
\@addtoreset{chapter}{part}
\makeatother

% Rimuove l'indentazione quando si crea un nuovo paragrafo
\setlength{\parindent}{0pt}

% footer
\pagestyle{fancyplain}
% rimuove la riga nell'header
\fancyhf{} % sets both header and footer to nothing
\renewcommand{\headrulewidth}{0pt}
\fancyfoot[L]{\href{https://github.com/Typing-Monkeys/AppuntiUniversita}{Typing Monkeys}}
\fancyfoot[C]{\emoji{gorilla}}
\fancyfoot[R]{\thepage}

% configurazione emoji
\usepackage{fontspec}
\usepackage{emoji}
\setemojifont{NotoColorEmoji.ttf}[Path=/usr/share/fonts/truetype/noto/]

\newtheorem{definition}{Definizione}
\newtheorem{lemma}{Lemma}
\newtheorem{theorem}{Teorema}
\newtheorem{corollary}{Corollario}

%% cambio nome al comando proof
\renewcommand*{\proofname}{Dimostrazione}

\begin{document}
\documentclass[a4paper,12 pt]{report}
\usepackage[T1]{fontenc}
\usepackage[utf8]{inputenc}
\usepackage{lmodern}
\usepackage{listings}

\usepackage{float}
\usepackage{subcaption}
\usepackage{wrapfig}
\usepackage{fancyhdr}
\usepackage{amsthm}

\usepackage{pgfplots}
\setlength {\marginparwidth }{2cm}
\usepackage{todonotes}
\newcommand{\TODO}[2][]
{\todo[size=\scriptsize, color=red, #1]{#2}}


asdf
\pgfplotsset{compat=1.18}
% forza le footnote a stare il più in basso possibile
\usepackage[bottom]{footmisc}


%% STILE LISTINGS
%%aaa
\usepackage{xcolor}

\definecolor{codegreen}{rgb}{0,0.6,0}
\definecolor{codegray}{rgb}{0.5,0.5,0.5}
\definecolor{codepurple}{rgb}{0.58,0,0.82}
\definecolor{backcolour}{rgb}{0.95,0.95,0.92}

\lstdefinestyle{mystyle}{
    backgroundcolor=\color{backcolour},   
    commentstyle=\color{codegreen},
    keywordstyle=\color{magenta},
    numberstyle=\tiny\color{codegray},
    stringstyle=\color{codepurple},
    basicstyle=\ttfamily\footnotesize,
    breakatwhitespace=false,         
    breaklines=true,                 
    captionpos=b,                    
    keepspaces=true,                 
    numbers=left,                    
    numbersep=5pt,                  
    showspaces=false,                
    showstringspaces=false,
    showtabs=false,                  
    tabsize=2
}

\lstset{style=mystyle}

%% -----

% mostra le subsubsection nell'indice
\setcounter{tocdepth}{3}
\setcounter{secnumdepth}{3}

% Resetta la numerazione dei chapter quando
% una nuova part viene creata
\makeatletter
\@addtoreset{chapter}{part}
\makeatother

% Rimuove l'indentazione quando si crea un nuovo paragrafo
\setlength{\parindent}{0pt}

% footer
\pagestyle{fancyplain}
% rimuove la riga nell'header
\fancyhf{} % sets both header and footer to nothing
\renewcommand{\headrulewidth}{0pt}
\fancyfoot[L]{\href{https://github.com/Typing-Monkeys/AppuntiUniversita}{Typing Monkeys}}
\fancyfoot[C]{\emoji{gorilla}}
\fancyfoot[R]{\thepage}

% configurazione emoji
\usepackage{fontspec}
\usepackage{emoji}
\setemojifont{NotoColorEmoji.ttf}[Path=/usr/share/fonts/truetype/noto/]

\newtheorem{definition}{Definizione}
\newtheorem{lemma}{Lemma}
\newtheorem{theorem}{Teorema}
\newtheorem{corollary}{Corollario}

%% cambio nome al comando proof
\renewcommand*{\proofname}{Dimostrazione}

\begin{document}
\documentclass[a4paper,12 pt]{report}
\usepackage[T1]{fontenc}
\usepackage[utf8]{inputenc}
\usepackage{lmodern}
\usepackage{listings}

\usepackage{float}
\usepackage{subcaption}
\usepackage{wrapfig}
\usepackage{fancyhdr}
\usepackage{amsthm}

\usepackage{pgfplots}
\setlength {\marginparwidth }{2cm}
\usepackage{todonotes}
\newcommand{\TODO}[2][]
{\todo[size=\scriptsize, color=red, #1]{#2}}


asdf
\pgfplotsset{compat=1.18}
% forza le footnote a stare il più in basso possibile
\usepackage[bottom]{footmisc}


%% STILE LISTINGS
%%aaa
\usepackage{xcolor}

\definecolor{codegreen}{rgb}{0,0.6,0}
\definecolor{codegray}{rgb}{0.5,0.5,0.5}
\definecolor{codepurple}{rgb}{0.58,0,0.82}
\definecolor{backcolour}{rgb}{0.95,0.95,0.92}

\lstdefinestyle{mystyle}{
    backgroundcolor=\color{backcolour},   
    commentstyle=\color{codegreen},
    keywordstyle=\color{magenta},
    numberstyle=\tiny\color{codegray},
    stringstyle=\color{codepurple},
    basicstyle=\ttfamily\footnotesize,
    breakatwhitespace=false,         
    breaklines=true,                 
    captionpos=b,                    
    keepspaces=true,                 
    numbers=left,                    
    numbersep=5pt,                  
    showspaces=false,                
    showstringspaces=false,
    showtabs=false,                  
    tabsize=2
}

\lstset{style=mystyle}

%% -----

% mostra le subsubsection nell'indice
\setcounter{tocdepth}{3}
\setcounter{secnumdepth}{3}

% Resetta la numerazione dei chapter quando
% una nuova part viene creata
\makeatletter
\@addtoreset{chapter}{part}
\makeatother

% Rimuove l'indentazione quando si crea un nuovo paragrafo
\setlength{\parindent}{0pt}

% footer
\pagestyle{fancyplain}
% rimuove la riga nell'header
\fancyhf{} % sets both header and footer to nothing
\renewcommand{\headrulewidth}{0pt}
\fancyfoot[L]{\href{https://github.com/Typing-Monkeys/AppuntiUniversita}{Typing Monkeys}}
\fancyfoot[C]{\emoji{gorilla}}
\fancyfoot[R]{\thepage}

% configurazione emoji
\usepackage{fontspec}
\usepackage{emoji}
\setemojifont{NotoColorEmoji.ttf}[Path=/usr/share/fonts/truetype/noto/]

\newtheorem{definition}{Definizione}
\newtheorem{lemma}{Lemma}
\newtheorem{theorem}{Teorema}
\newtheorem{corollary}{Corollario}

%% cambio nome al comando proof
\renewcommand*{\proofname}{Dimostrazione}

\begin{document}
\documentclass[a4paper,12 pt]{report}
\usepackage[T1]{fontenc}
\usepackage[utf8]{inputenc}
\usepackage{lmodern}
\usepackage{listings}

\usepackage{float}
\usepackage{subcaption}
\usepackage{wrapfig}
\usepackage{fancyhdr}
\usepackage{amsthm}

\usepackage{pgfplots}
\setlength {\marginparwidth }{2cm}
\usepackage{todonotes}
\newcommand{\TODO}[2][]
{\todo[size=\scriptsize, color=red, #1]{#2}}


asdf
\pgfplotsset{compat=1.18}
% forza le footnote a stare il più in basso possibile
\usepackage[bottom]{footmisc}


%% STILE LISTINGS
%%aaa
\usepackage{xcolor}

\definecolor{codegreen}{rgb}{0,0.6,0}
\definecolor{codegray}{rgb}{0.5,0.5,0.5}
\definecolor{codepurple}{rgb}{0.58,0,0.82}
\definecolor{backcolour}{rgb}{0.95,0.95,0.92}

\lstdefinestyle{mystyle}{
    backgroundcolor=\color{backcolour},   
    commentstyle=\color{codegreen},
    keywordstyle=\color{magenta},
    numberstyle=\tiny\color{codegray},
    stringstyle=\color{codepurple},
    basicstyle=\ttfamily\footnotesize,
    breakatwhitespace=false,         
    breaklines=true,                 
    captionpos=b,                    
    keepspaces=true,                 
    numbers=left,                    
    numbersep=5pt,                  
    showspaces=false,                
    showstringspaces=false,
    showtabs=false,                  
    tabsize=2
}

\lstset{style=mystyle}

%% -----

% mostra le subsubsection nell'indice
\setcounter{tocdepth}{3}
\setcounter{secnumdepth}{3}

% Resetta la numerazione dei chapter quando
% una nuova part viene creata
\makeatletter
\@addtoreset{chapter}{part}
\makeatother

% Rimuove l'indentazione quando si crea un nuovo paragrafo
\setlength{\parindent}{0pt}

% footer
\pagestyle{fancyplain}
% rimuove la riga nell'header
\fancyhf{} % sets both header and footer to nothing
\renewcommand{\headrulewidth}{0pt}
\fancyfoot[L]{\href{https://github.com/Typing-Monkeys/AppuntiUniversita}{Typing Monkeys}}
\fancyfoot[C]{\emoji{gorilla}}
\fancyfoot[R]{\thepage}

% configurazione emoji
\usepackage{fontspec}
\usepackage{emoji}
\setemojifont{NotoColorEmoji.ttf}[Path=/usr/share/fonts/truetype/noto/]

\newtheorem{definition}{Definizione}
\newtheorem{lemma}{Lemma}
\newtheorem{theorem}{Teorema}
\newtheorem{corollary}{Corollario}

%% cambio nome al comando proof
\renewcommand*{\proofname}{Dimostrazione}

\begin{document}
\include{frontmatter/main.tex}

\tableofcontents

\include{quote/main.tex}

%% Aggiungere i capitoli qui sotto
\include{capitoli/richiami/main.tex}
\include{capitoli/immagini/main.tex}

\end{document}


\tableofcontents

\documentclass[a4paper,12 pt]{report}
\usepackage[T1]{fontenc}
\usepackage[utf8]{inputenc}
\usepackage{lmodern}
\usepackage{listings}

\usepackage{float}
\usepackage{subcaption}
\usepackage{wrapfig}
\usepackage{fancyhdr}
\usepackage{amsthm}

\usepackage{pgfplots}
\setlength {\marginparwidth }{2cm}
\usepackage{todonotes}
\newcommand{\TODO}[2][]
{\todo[size=\scriptsize, color=red, #1]{#2}}


asdf
\pgfplotsset{compat=1.18}
% forza le footnote a stare il più in basso possibile
\usepackage[bottom]{footmisc}


%% STILE LISTINGS
%%aaa
\usepackage{xcolor}

\definecolor{codegreen}{rgb}{0,0.6,0}
\definecolor{codegray}{rgb}{0.5,0.5,0.5}
\definecolor{codepurple}{rgb}{0.58,0,0.82}
\definecolor{backcolour}{rgb}{0.95,0.95,0.92}

\lstdefinestyle{mystyle}{
    backgroundcolor=\color{backcolour},   
    commentstyle=\color{codegreen},
    keywordstyle=\color{magenta},
    numberstyle=\tiny\color{codegray},
    stringstyle=\color{codepurple},
    basicstyle=\ttfamily\footnotesize,
    breakatwhitespace=false,         
    breaklines=true,                 
    captionpos=b,                    
    keepspaces=true,                 
    numbers=left,                    
    numbersep=5pt,                  
    showspaces=false,                
    showstringspaces=false,
    showtabs=false,                  
    tabsize=2
}

\lstset{style=mystyle}

%% -----

% mostra le subsubsection nell'indice
\setcounter{tocdepth}{3}
\setcounter{secnumdepth}{3}

% Resetta la numerazione dei chapter quando
% una nuova part viene creata
\makeatletter
\@addtoreset{chapter}{part}
\makeatother

% Rimuove l'indentazione quando si crea un nuovo paragrafo
\setlength{\parindent}{0pt}

% footer
\pagestyle{fancyplain}
% rimuove la riga nell'header
\fancyhf{} % sets both header and footer to nothing
\renewcommand{\headrulewidth}{0pt}
\fancyfoot[L]{\href{https://github.com/Typing-Monkeys/AppuntiUniversita}{Typing Monkeys}}
\fancyfoot[C]{\emoji{gorilla}}
\fancyfoot[R]{\thepage}

% configurazione emoji
\usepackage{fontspec}
\usepackage{emoji}
\setemojifont{NotoColorEmoji.ttf}[Path=/usr/share/fonts/truetype/noto/]

\newtheorem{definition}{Definizione}
\newtheorem{lemma}{Lemma}
\newtheorem{theorem}{Teorema}
\newtheorem{corollary}{Corollario}

%% cambio nome al comando proof
\renewcommand*{\proofname}{Dimostrazione}

\begin{document}
\include{frontmatter/main.tex}

\tableofcontents

\include{quote/main.tex}

%% Aggiungere i capitoli qui sotto
\include{capitoli/richiami/main.tex}
\include{capitoli/immagini/main.tex}

\end{document}


%% Aggiungere i capitoli qui sotto
\documentclass[a4paper,12 pt]{report}
\usepackage[T1]{fontenc}
\usepackage[utf8]{inputenc}
\usepackage{lmodern}
\usepackage{listings}

\usepackage{float}
\usepackage{subcaption}
\usepackage{wrapfig}
\usepackage{fancyhdr}
\usepackage{amsthm}

\usepackage{pgfplots}
\setlength {\marginparwidth }{2cm}
\usepackage{todonotes}
\newcommand{\TODO}[2][]
{\todo[size=\scriptsize, color=red, #1]{#2}}


asdf
\pgfplotsset{compat=1.18}
% forza le footnote a stare il più in basso possibile
\usepackage[bottom]{footmisc}


%% STILE LISTINGS
%%aaa
\usepackage{xcolor}

\definecolor{codegreen}{rgb}{0,0.6,0}
\definecolor{codegray}{rgb}{0.5,0.5,0.5}
\definecolor{codepurple}{rgb}{0.58,0,0.82}
\definecolor{backcolour}{rgb}{0.95,0.95,0.92}

\lstdefinestyle{mystyle}{
    backgroundcolor=\color{backcolour},   
    commentstyle=\color{codegreen},
    keywordstyle=\color{magenta},
    numberstyle=\tiny\color{codegray},
    stringstyle=\color{codepurple},
    basicstyle=\ttfamily\footnotesize,
    breakatwhitespace=false,         
    breaklines=true,                 
    captionpos=b,                    
    keepspaces=true,                 
    numbers=left,                    
    numbersep=5pt,                  
    showspaces=false,                
    showstringspaces=false,
    showtabs=false,                  
    tabsize=2
}

\lstset{style=mystyle}

%% -----

% mostra le subsubsection nell'indice
\setcounter{tocdepth}{3}
\setcounter{secnumdepth}{3}

% Resetta la numerazione dei chapter quando
% una nuova part viene creata
\makeatletter
\@addtoreset{chapter}{part}
\makeatother

% Rimuove l'indentazione quando si crea un nuovo paragrafo
\setlength{\parindent}{0pt}

% footer
\pagestyle{fancyplain}
% rimuove la riga nell'header
\fancyhf{} % sets both header and footer to nothing
\renewcommand{\headrulewidth}{0pt}
\fancyfoot[L]{\href{https://github.com/Typing-Monkeys/AppuntiUniversita}{Typing Monkeys}}
\fancyfoot[C]{\emoji{gorilla}}
\fancyfoot[R]{\thepage}

% configurazione emoji
\usepackage{fontspec}
\usepackage{emoji}
\setemojifont{NotoColorEmoji.ttf}[Path=/usr/share/fonts/truetype/noto/]

\newtheorem{definition}{Definizione}
\newtheorem{lemma}{Lemma}
\newtheorem{theorem}{Teorema}
\newtheorem{corollary}{Corollario}

%% cambio nome al comando proof
\renewcommand*{\proofname}{Dimostrazione}

\begin{document}
\include{frontmatter/main.tex}

\tableofcontents

\include{quote/main.tex}

%% Aggiungere i capitoli qui sotto
\include{capitoli/richiami/main.tex}
\include{capitoli/immagini/main.tex}

\end{document}

\documentclass[a4paper,12 pt]{report}
\usepackage[T1]{fontenc}
\usepackage[utf8]{inputenc}
\usepackage{lmodern}
\usepackage{listings}

\usepackage{float}
\usepackage{subcaption}
\usepackage{wrapfig}
\usepackage{fancyhdr}
\usepackage{amsthm}

\usepackage{pgfplots}
\setlength {\marginparwidth }{2cm}
\usepackage{todonotes}
\newcommand{\TODO}[2][]
{\todo[size=\scriptsize, color=red, #1]{#2}}


asdf
\pgfplotsset{compat=1.18}
% forza le footnote a stare il più in basso possibile
\usepackage[bottom]{footmisc}


%% STILE LISTINGS
%%aaa
\usepackage{xcolor}

\definecolor{codegreen}{rgb}{0,0.6,0}
\definecolor{codegray}{rgb}{0.5,0.5,0.5}
\definecolor{codepurple}{rgb}{0.58,0,0.82}
\definecolor{backcolour}{rgb}{0.95,0.95,0.92}

\lstdefinestyle{mystyle}{
    backgroundcolor=\color{backcolour},   
    commentstyle=\color{codegreen},
    keywordstyle=\color{magenta},
    numberstyle=\tiny\color{codegray},
    stringstyle=\color{codepurple},
    basicstyle=\ttfamily\footnotesize,
    breakatwhitespace=false,         
    breaklines=true,                 
    captionpos=b,                    
    keepspaces=true,                 
    numbers=left,                    
    numbersep=5pt,                  
    showspaces=false,                
    showstringspaces=false,
    showtabs=false,                  
    tabsize=2
}

\lstset{style=mystyle}

%% -----

% mostra le subsubsection nell'indice
\setcounter{tocdepth}{3}
\setcounter{secnumdepth}{3}

% Resetta la numerazione dei chapter quando
% una nuova part viene creata
\makeatletter
\@addtoreset{chapter}{part}
\makeatother

% Rimuove l'indentazione quando si crea un nuovo paragrafo
\setlength{\parindent}{0pt}

% footer
\pagestyle{fancyplain}
% rimuove la riga nell'header
\fancyhf{} % sets both header and footer to nothing
\renewcommand{\headrulewidth}{0pt}
\fancyfoot[L]{\href{https://github.com/Typing-Monkeys/AppuntiUniversita}{Typing Monkeys}}
\fancyfoot[C]{\emoji{gorilla}}
\fancyfoot[R]{\thepage}

% configurazione emoji
\usepackage{fontspec}
\usepackage{emoji}
\setemojifont{NotoColorEmoji.ttf}[Path=/usr/share/fonts/truetype/noto/]

\newtheorem{definition}{Definizione}
\newtheorem{lemma}{Lemma}
\newtheorem{theorem}{Teorema}
\newtheorem{corollary}{Corollario}

%% cambio nome al comando proof
\renewcommand*{\proofname}{Dimostrazione}

\begin{document}
\include{frontmatter/main.tex}

\tableofcontents

\include{quote/main.tex}

%% Aggiungere i capitoli qui sotto
\include{capitoli/richiami/main.tex}
\include{capitoli/immagini/main.tex}

\end{document}


\end{document}


\tableofcontents

\documentclass[a4paper,12 pt]{report}
\usepackage[T1]{fontenc}
\usepackage[utf8]{inputenc}
\usepackage{lmodern}
\usepackage{listings}

\usepackage{float}
\usepackage{subcaption}
\usepackage{wrapfig}
\usepackage{fancyhdr}
\usepackage{amsthm}

\usepackage{pgfplots}
\setlength {\marginparwidth }{2cm}
\usepackage{todonotes}
\newcommand{\TODO}[2][]
{\todo[size=\scriptsize, color=red, #1]{#2}}


asdf
\pgfplotsset{compat=1.18}
% forza le footnote a stare il più in basso possibile
\usepackage[bottom]{footmisc}


%% STILE LISTINGS
%%aaa
\usepackage{xcolor}

\definecolor{codegreen}{rgb}{0,0.6,0}
\definecolor{codegray}{rgb}{0.5,0.5,0.5}
\definecolor{codepurple}{rgb}{0.58,0,0.82}
\definecolor{backcolour}{rgb}{0.95,0.95,0.92}

\lstdefinestyle{mystyle}{
    backgroundcolor=\color{backcolour},   
    commentstyle=\color{codegreen},
    keywordstyle=\color{magenta},
    numberstyle=\tiny\color{codegray},
    stringstyle=\color{codepurple},
    basicstyle=\ttfamily\footnotesize,
    breakatwhitespace=false,         
    breaklines=true,                 
    captionpos=b,                    
    keepspaces=true,                 
    numbers=left,                    
    numbersep=5pt,                  
    showspaces=false,                
    showstringspaces=false,
    showtabs=false,                  
    tabsize=2
}

\lstset{style=mystyle}

%% -----

% mostra le subsubsection nell'indice
\setcounter{tocdepth}{3}
\setcounter{secnumdepth}{3}

% Resetta la numerazione dei chapter quando
% una nuova part viene creata
\makeatletter
\@addtoreset{chapter}{part}
\makeatother

% Rimuove l'indentazione quando si crea un nuovo paragrafo
\setlength{\parindent}{0pt}

% footer
\pagestyle{fancyplain}
% rimuove la riga nell'header
\fancyhf{} % sets both header and footer to nothing
\renewcommand{\headrulewidth}{0pt}
\fancyfoot[L]{\href{https://github.com/Typing-Monkeys/AppuntiUniversita}{Typing Monkeys}}
\fancyfoot[C]{\emoji{gorilla}}
\fancyfoot[R]{\thepage}

% configurazione emoji
\usepackage{fontspec}
\usepackage{emoji}
\setemojifont{NotoColorEmoji.ttf}[Path=/usr/share/fonts/truetype/noto/]

\newtheorem{definition}{Definizione}
\newtheorem{lemma}{Lemma}
\newtheorem{theorem}{Teorema}
\newtheorem{corollary}{Corollario}

%% cambio nome al comando proof
\renewcommand*{\proofname}{Dimostrazione}

\begin{document}
\documentclass[a4paper,12 pt]{report}
\usepackage[T1]{fontenc}
\usepackage[utf8]{inputenc}
\usepackage{lmodern}
\usepackage{listings}

\usepackage{float}
\usepackage{subcaption}
\usepackage{wrapfig}
\usepackage{fancyhdr}
\usepackage{amsthm}

\usepackage{pgfplots}
\setlength {\marginparwidth }{2cm}
\usepackage{todonotes}
\newcommand{\TODO}[2][]
{\todo[size=\scriptsize, color=red, #1]{#2}}


asdf
\pgfplotsset{compat=1.18}
% forza le footnote a stare il più in basso possibile
\usepackage[bottom]{footmisc}


%% STILE LISTINGS
%%aaa
\usepackage{xcolor}

\definecolor{codegreen}{rgb}{0,0.6,0}
\definecolor{codegray}{rgb}{0.5,0.5,0.5}
\definecolor{codepurple}{rgb}{0.58,0,0.82}
\definecolor{backcolour}{rgb}{0.95,0.95,0.92}

\lstdefinestyle{mystyle}{
    backgroundcolor=\color{backcolour},   
    commentstyle=\color{codegreen},
    keywordstyle=\color{magenta},
    numberstyle=\tiny\color{codegray},
    stringstyle=\color{codepurple},
    basicstyle=\ttfamily\footnotesize,
    breakatwhitespace=false,         
    breaklines=true,                 
    captionpos=b,                    
    keepspaces=true,                 
    numbers=left,                    
    numbersep=5pt,                  
    showspaces=false,                
    showstringspaces=false,
    showtabs=false,                  
    tabsize=2
}

\lstset{style=mystyle}

%% -----

% mostra le subsubsection nell'indice
\setcounter{tocdepth}{3}
\setcounter{secnumdepth}{3}

% Resetta la numerazione dei chapter quando
% una nuova part viene creata
\makeatletter
\@addtoreset{chapter}{part}
\makeatother

% Rimuove l'indentazione quando si crea un nuovo paragrafo
\setlength{\parindent}{0pt}

% footer
\pagestyle{fancyplain}
% rimuove la riga nell'header
\fancyhf{} % sets both header and footer to nothing
\renewcommand{\headrulewidth}{0pt}
\fancyfoot[L]{\href{https://github.com/Typing-Monkeys/AppuntiUniversita}{Typing Monkeys}}
\fancyfoot[C]{\emoji{gorilla}}
\fancyfoot[R]{\thepage}

% configurazione emoji
\usepackage{fontspec}
\usepackage{emoji}
\setemojifont{NotoColorEmoji.ttf}[Path=/usr/share/fonts/truetype/noto/]

\newtheorem{definition}{Definizione}
\newtheorem{lemma}{Lemma}
\newtheorem{theorem}{Teorema}
\newtheorem{corollary}{Corollario}

%% cambio nome al comando proof
\renewcommand*{\proofname}{Dimostrazione}

\begin{document}
\include{frontmatter/main.tex}

\tableofcontents

\include{quote/main.tex}

%% Aggiungere i capitoli qui sotto
\include{capitoli/richiami/main.tex}
\include{capitoli/immagini/main.tex}

\end{document}


\tableofcontents

\documentclass[a4paper,12 pt]{report}
\usepackage[T1]{fontenc}
\usepackage[utf8]{inputenc}
\usepackage{lmodern}
\usepackage{listings}

\usepackage{float}
\usepackage{subcaption}
\usepackage{wrapfig}
\usepackage{fancyhdr}
\usepackage{amsthm}

\usepackage{pgfplots}
\setlength {\marginparwidth }{2cm}
\usepackage{todonotes}
\newcommand{\TODO}[2][]
{\todo[size=\scriptsize, color=red, #1]{#2}}


asdf
\pgfplotsset{compat=1.18}
% forza le footnote a stare il più in basso possibile
\usepackage[bottom]{footmisc}


%% STILE LISTINGS
%%aaa
\usepackage{xcolor}

\definecolor{codegreen}{rgb}{0,0.6,0}
\definecolor{codegray}{rgb}{0.5,0.5,0.5}
\definecolor{codepurple}{rgb}{0.58,0,0.82}
\definecolor{backcolour}{rgb}{0.95,0.95,0.92}

\lstdefinestyle{mystyle}{
    backgroundcolor=\color{backcolour},   
    commentstyle=\color{codegreen},
    keywordstyle=\color{magenta},
    numberstyle=\tiny\color{codegray},
    stringstyle=\color{codepurple},
    basicstyle=\ttfamily\footnotesize,
    breakatwhitespace=false,         
    breaklines=true,                 
    captionpos=b,                    
    keepspaces=true,                 
    numbers=left,                    
    numbersep=5pt,                  
    showspaces=false,                
    showstringspaces=false,
    showtabs=false,                  
    tabsize=2
}

\lstset{style=mystyle}

%% -----

% mostra le subsubsection nell'indice
\setcounter{tocdepth}{3}
\setcounter{secnumdepth}{3}

% Resetta la numerazione dei chapter quando
% una nuova part viene creata
\makeatletter
\@addtoreset{chapter}{part}
\makeatother

% Rimuove l'indentazione quando si crea un nuovo paragrafo
\setlength{\parindent}{0pt}

% footer
\pagestyle{fancyplain}
% rimuove la riga nell'header
\fancyhf{} % sets both header and footer to nothing
\renewcommand{\headrulewidth}{0pt}
\fancyfoot[L]{\href{https://github.com/Typing-Monkeys/AppuntiUniversita}{Typing Monkeys}}
\fancyfoot[C]{\emoji{gorilla}}
\fancyfoot[R]{\thepage}

% configurazione emoji
\usepackage{fontspec}
\usepackage{emoji}
\setemojifont{NotoColorEmoji.ttf}[Path=/usr/share/fonts/truetype/noto/]

\newtheorem{definition}{Definizione}
\newtheorem{lemma}{Lemma}
\newtheorem{theorem}{Teorema}
\newtheorem{corollary}{Corollario}

%% cambio nome al comando proof
\renewcommand*{\proofname}{Dimostrazione}

\begin{document}
\include{frontmatter/main.tex}

\tableofcontents

\include{quote/main.tex}

%% Aggiungere i capitoli qui sotto
\include{capitoli/richiami/main.tex}
\include{capitoli/immagini/main.tex}

\end{document}


%% Aggiungere i capitoli qui sotto
\documentclass[a4paper,12 pt]{report}
\usepackage[T1]{fontenc}
\usepackage[utf8]{inputenc}
\usepackage{lmodern}
\usepackage{listings}

\usepackage{float}
\usepackage{subcaption}
\usepackage{wrapfig}
\usepackage{fancyhdr}
\usepackage{amsthm}

\usepackage{pgfplots}
\setlength {\marginparwidth }{2cm}
\usepackage{todonotes}
\newcommand{\TODO}[2][]
{\todo[size=\scriptsize, color=red, #1]{#2}}


asdf
\pgfplotsset{compat=1.18}
% forza le footnote a stare il più in basso possibile
\usepackage[bottom]{footmisc}


%% STILE LISTINGS
%%aaa
\usepackage{xcolor}

\definecolor{codegreen}{rgb}{0,0.6,0}
\definecolor{codegray}{rgb}{0.5,0.5,0.5}
\definecolor{codepurple}{rgb}{0.58,0,0.82}
\definecolor{backcolour}{rgb}{0.95,0.95,0.92}

\lstdefinestyle{mystyle}{
    backgroundcolor=\color{backcolour},   
    commentstyle=\color{codegreen},
    keywordstyle=\color{magenta},
    numberstyle=\tiny\color{codegray},
    stringstyle=\color{codepurple},
    basicstyle=\ttfamily\footnotesize,
    breakatwhitespace=false,         
    breaklines=true,                 
    captionpos=b,                    
    keepspaces=true,                 
    numbers=left,                    
    numbersep=5pt,                  
    showspaces=false,                
    showstringspaces=false,
    showtabs=false,                  
    tabsize=2
}

\lstset{style=mystyle}

%% -----

% mostra le subsubsection nell'indice
\setcounter{tocdepth}{3}
\setcounter{secnumdepth}{3}

% Resetta la numerazione dei chapter quando
% una nuova part viene creata
\makeatletter
\@addtoreset{chapter}{part}
\makeatother

% Rimuove l'indentazione quando si crea un nuovo paragrafo
\setlength{\parindent}{0pt}

% footer
\pagestyle{fancyplain}
% rimuove la riga nell'header
\fancyhf{} % sets both header and footer to nothing
\renewcommand{\headrulewidth}{0pt}
\fancyfoot[L]{\href{https://github.com/Typing-Monkeys/AppuntiUniversita}{Typing Monkeys}}
\fancyfoot[C]{\emoji{gorilla}}
\fancyfoot[R]{\thepage}

% configurazione emoji
\usepackage{fontspec}
\usepackage{emoji}
\setemojifont{NotoColorEmoji.ttf}[Path=/usr/share/fonts/truetype/noto/]

\newtheorem{definition}{Definizione}
\newtheorem{lemma}{Lemma}
\newtheorem{theorem}{Teorema}
\newtheorem{corollary}{Corollario}

%% cambio nome al comando proof
\renewcommand*{\proofname}{Dimostrazione}

\begin{document}
\include{frontmatter/main.tex}

\tableofcontents

\include{quote/main.tex}

%% Aggiungere i capitoli qui sotto
\include{capitoli/richiami/main.tex}
\include{capitoli/immagini/main.tex}

\end{document}

\documentclass[a4paper,12 pt]{report}
\usepackage[T1]{fontenc}
\usepackage[utf8]{inputenc}
\usepackage{lmodern}
\usepackage{listings}

\usepackage{float}
\usepackage{subcaption}
\usepackage{wrapfig}
\usepackage{fancyhdr}
\usepackage{amsthm}

\usepackage{pgfplots}
\setlength {\marginparwidth }{2cm}
\usepackage{todonotes}
\newcommand{\TODO}[2][]
{\todo[size=\scriptsize, color=red, #1]{#2}}


asdf
\pgfplotsset{compat=1.18}
% forza le footnote a stare il più in basso possibile
\usepackage[bottom]{footmisc}


%% STILE LISTINGS
%%aaa
\usepackage{xcolor}

\definecolor{codegreen}{rgb}{0,0.6,0}
\definecolor{codegray}{rgb}{0.5,0.5,0.5}
\definecolor{codepurple}{rgb}{0.58,0,0.82}
\definecolor{backcolour}{rgb}{0.95,0.95,0.92}

\lstdefinestyle{mystyle}{
    backgroundcolor=\color{backcolour},   
    commentstyle=\color{codegreen},
    keywordstyle=\color{magenta},
    numberstyle=\tiny\color{codegray},
    stringstyle=\color{codepurple},
    basicstyle=\ttfamily\footnotesize,
    breakatwhitespace=false,         
    breaklines=true,                 
    captionpos=b,                    
    keepspaces=true,                 
    numbers=left,                    
    numbersep=5pt,                  
    showspaces=false,                
    showstringspaces=false,
    showtabs=false,                  
    tabsize=2
}

\lstset{style=mystyle}

%% -----

% mostra le subsubsection nell'indice
\setcounter{tocdepth}{3}
\setcounter{secnumdepth}{3}

% Resetta la numerazione dei chapter quando
% una nuova part viene creata
\makeatletter
\@addtoreset{chapter}{part}
\makeatother

% Rimuove l'indentazione quando si crea un nuovo paragrafo
\setlength{\parindent}{0pt}

% footer
\pagestyle{fancyplain}
% rimuove la riga nell'header
\fancyhf{} % sets both header and footer to nothing
\renewcommand{\headrulewidth}{0pt}
\fancyfoot[L]{\href{https://github.com/Typing-Monkeys/AppuntiUniversita}{Typing Monkeys}}
\fancyfoot[C]{\emoji{gorilla}}
\fancyfoot[R]{\thepage}

% configurazione emoji
\usepackage{fontspec}
\usepackage{emoji}
\setemojifont{NotoColorEmoji.ttf}[Path=/usr/share/fonts/truetype/noto/]

\newtheorem{definition}{Definizione}
\newtheorem{lemma}{Lemma}
\newtheorem{theorem}{Teorema}
\newtheorem{corollary}{Corollario}

%% cambio nome al comando proof
\renewcommand*{\proofname}{Dimostrazione}

\begin{document}
\include{frontmatter/main.tex}

\tableofcontents

\include{quote/main.tex}

%% Aggiungere i capitoli qui sotto
\include{capitoli/richiami/main.tex}
\include{capitoli/immagini/main.tex}

\end{document}


\end{document}


%% Aggiungere i capitoli qui sotto
\documentclass[a4paper,12 pt]{report}
\usepackage[T1]{fontenc}
\usepackage[utf8]{inputenc}
\usepackage{lmodern}
\usepackage{listings}

\usepackage{float}
\usepackage{subcaption}
\usepackage{wrapfig}
\usepackage{fancyhdr}
\usepackage{amsthm}

\usepackage{pgfplots}
\setlength {\marginparwidth }{2cm}
\usepackage{todonotes}
\newcommand{\TODO}[2][]
{\todo[size=\scriptsize, color=red, #1]{#2}}


asdf
\pgfplotsset{compat=1.18}
% forza le footnote a stare il più in basso possibile
\usepackage[bottom]{footmisc}


%% STILE LISTINGS
%%aaa
\usepackage{xcolor}

\definecolor{codegreen}{rgb}{0,0.6,0}
\definecolor{codegray}{rgb}{0.5,0.5,0.5}
\definecolor{codepurple}{rgb}{0.58,0,0.82}
\definecolor{backcolour}{rgb}{0.95,0.95,0.92}

\lstdefinestyle{mystyle}{
    backgroundcolor=\color{backcolour},   
    commentstyle=\color{codegreen},
    keywordstyle=\color{magenta},
    numberstyle=\tiny\color{codegray},
    stringstyle=\color{codepurple},
    basicstyle=\ttfamily\footnotesize,
    breakatwhitespace=false,         
    breaklines=true,                 
    captionpos=b,                    
    keepspaces=true,                 
    numbers=left,                    
    numbersep=5pt,                  
    showspaces=false,                
    showstringspaces=false,
    showtabs=false,                  
    tabsize=2
}

\lstset{style=mystyle}

%% -----

% mostra le subsubsection nell'indice
\setcounter{tocdepth}{3}
\setcounter{secnumdepth}{3}

% Resetta la numerazione dei chapter quando
% una nuova part viene creata
\makeatletter
\@addtoreset{chapter}{part}
\makeatother

% Rimuove l'indentazione quando si crea un nuovo paragrafo
\setlength{\parindent}{0pt}

% footer
\pagestyle{fancyplain}
% rimuove la riga nell'header
\fancyhf{} % sets both header and footer to nothing
\renewcommand{\headrulewidth}{0pt}
\fancyfoot[L]{\href{https://github.com/Typing-Monkeys/AppuntiUniversita}{Typing Monkeys}}
\fancyfoot[C]{\emoji{gorilla}}
\fancyfoot[R]{\thepage}

% configurazione emoji
\usepackage{fontspec}
\usepackage{emoji}
\setemojifont{NotoColorEmoji.ttf}[Path=/usr/share/fonts/truetype/noto/]

\newtheorem{definition}{Definizione}
\newtheorem{lemma}{Lemma}
\newtheorem{theorem}{Teorema}
\newtheorem{corollary}{Corollario}

%% cambio nome al comando proof
\renewcommand*{\proofname}{Dimostrazione}

\begin{document}
\documentclass[a4paper,12 pt]{report}
\usepackage[T1]{fontenc}
\usepackage[utf8]{inputenc}
\usepackage{lmodern}
\usepackage{listings}

\usepackage{float}
\usepackage{subcaption}
\usepackage{wrapfig}
\usepackage{fancyhdr}
\usepackage{amsthm}

\usepackage{pgfplots}
\setlength {\marginparwidth }{2cm}
\usepackage{todonotes}
\newcommand{\TODO}[2][]
{\todo[size=\scriptsize, color=red, #1]{#2}}


asdf
\pgfplotsset{compat=1.18}
% forza le footnote a stare il più in basso possibile
\usepackage[bottom]{footmisc}


%% STILE LISTINGS
%%aaa
\usepackage{xcolor}

\definecolor{codegreen}{rgb}{0,0.6,0}
\definecolor{codegray}{rgb}{0.5,0.5,0.5}
\definecolor{codepurple}{rgb}{0.58,0,0.82}
\definecolor{backcolour}{rgb}{0.95,0.95,0.92}

\lstdefinestyle{mystyle}{
    backgroundcolor=\color{backcolour},   
    commentstyle=\color{codegreen},
    keywordstyle=\color{magenta},
    numberstyle=\tiny\color{codegray},
    stringstyle=\color{codepurple},
    basicstyle=\ttfamily\footnotesize,
    breakatwhitespace=false,         
    breaklines=true,                 
    captionpos=b,                    
    keepspaces=true,                 
    numbers=left,                    
    numbersep=5pt,                  
    showspaces=false,                
    showstringspaces=false,
    showtabs=false,                  
    tabsize=2
}

\lstset{style=mystyle}

%% -----

% mostra le subsubsection nell'indice
\setcounter{tocdepth}{3}
\setcounter{secnumdepth}{3}

% Resetta la numerazione dei chapter quando
% una nuova part viene creata
\makeatletter
\@addtoreset{chapter}{part}
\makeatother

% Rimuove l'indentazione quando si crea un nuovo paragrafo
\setlength{\parindent}{0pt}

% footer
\pagestyle{fancyplain}
% rimuove la riga nell'header
\fancyhf{} % sets both header and footer to nothing
\renewcommand{\headrulewidth}{0pt}
\fancyfoot[L]{\href{https://github.com/Typing-Monkeys/AppuntiUniversita}{Typing Monkeys}}
\fancyfoot[C]{\emoji{gorilla}}
\fancyfoot[R]{\thepage}

% configurazione emoji
\usepackage{fontspec}
\usepackage{emoji}
\setemojifont{NotoColorEmoji.ttf}[Path=/usr/share/fonts/truetype/noto/]

\newtheorem{definition}{Definizione}
\newtheorem{lemma}{Lemma}
\newtheorem{theorem}{Teorema}
\newtheorem{corollary}{Corollario}

%% cambio nome al comando proof
\renewcommand*{\proofname}{Dimostrazione}

\begin{document}
\include{frontmatter/main.tex}

\tableofcontents

\include{quote/main.tex}

%% Aggiungere i capitoli qui sotto
\include{capitoli/richiami/main.tex}
\include{capitoli/immagini/main.tex}

\end{document}


\tableofcontents

\documentclass[a4paper,12 pt]{report}
\usepackage[T1]{fontenc}
\usepackage[utf8]{inputenc}
\usepackage{lmodern}
\usepackage{listings}

\usepackage{float}
\usepackage{subcaption}
\usepackage{wrapfig}
\usepackage{fancyhdr}
\usepackage{amsthm}

\usepackage{pgfplots}
\setlength {\marginparwidth }{2cm}
\usepackage{todonotes}
\newcommand{\TODO}[2][]
{\todo[size=\scriptsize, color=red, #1]{#2}}


asdf
\pgfplotsset{compat=1.18}
% forza le footnote a stare il più in basso possibile
\usepackage[bottom]{footmisc}


%% STILE LISTINGS
%%aaa
\usepackage{xcolor}

\definecolor{codegreen}{rgb}{0,0.6,0}
\definecolor{codegray}{rgb}{0.5,0.5,0.5}
\definecolor{codepurple}{rgb}{0.58,0,0.82}
\definecolor{backcolour}{rgb}{0.95,0.95,0.92}

\lstdefinestyle{mystyle}{
    backgroundcolor=\color{backcolour},   
    commentstyle=\color{codegreen},
    keywordstyle=\color{magenta},
    numberstyle=\tiny\color{codegray},
    stringstyle=\color{codepurple},
    basicstyle=\ttfamily\footnotesize,
    breakatwhitespace=false,         
    breaklines=true,                 
    captionpos=b,                    
    keepspaces=true,                 
    numbers=left,                    
    numbersep=5pt,                  
    showspaces=false,                
    showstringspaces=false,
    showtabs=false,                  
    tabsize=2
}

\lstset{style=mystyle}

%% -----

% mostra le subsubsection nell'indice
\setcounter{tocdepth}{3}
\setcounter{secnumdepth}{3}

% Resetta la numerazione dei chapter quando
% una nuova part viene creata
\makeatletter
\@addtoreset{chapter}{part}
\makeatother

% Rimuove l'indentazione quando si crea un nuovo paragrafo
\setlength{\parindent}{0pt}

% footer
\pagestyle{fancyplain}
% rimuove la riga nell'header
\fancyhf{} % sets both header and footer to nothing
\renewcommand{\headrulewidth}{0pt}
\fancyfoot[L]{\href{https://github.com/Typing-Monkeys/AppuntiUniversita}{Typing Monkeys}}
\fancyfoot[C]{\emoji{gorilla}}
\fancyfoot[R]{\thepage}

% configurazione emoji
\usepackage{fontspec}
\usepackage{emoji}
\setemojifont{NotoColorEmoji.ttf}[Path=/usr/share/fonts/truetype/noto/]

\newtheorem{definition}{Definizione}
\newtheorem{lemma}{Lemma}
\newtheorem{theorem}{Teorema}
\newtheorem{corollary}{Corollario}

%% cambio nome al comando proof
\renewcommand*{\proofname}{Dimostrazione}

\begin{document}
\include{frontmatter/main.tex}

\tableofcontents

\include{quote/main.tex}

%% Aggiungere i capitoli qui sotto
\include{capitoli/richiami/main.tex}
\include{capitoli/immagini/main.tex}

\end{document}


%% Aggiungere i capitoli qui sotto
\documentclass[a4paper,12 pt]{report}
\usepackage[T1]{fontenc}
\usepackage[utf8]{inputenc}
\usepackage{lmodern}
\usepackage{listings}

\usepackage{float}
\usepackage{subcaption}
\usepackage{wrapfig}
\usepackage{fancyhdr}
\usepackage{amsthm}

\usepackage{pgfplots}
\setlength {\marginparwidth }{2cm}
\usepackage{todonotes}
\newcommand{\TODO}[2][]
{\todo[size=\scriptsize, color=red, #1]{#2}}


asdf
\pgfplotsset{compat=1.18}
% forza le footnote a stare il più in basso possibile
\usepackage[bottom]{footmisc}


%% STILE LISTINGS
%%aaa
\usepackage{xcolor}

\definecolor{codegreen}{rgb}{0,0.6,0}
\definecolor{codegray}{rgb}{0.5,0.5,0.5}
\definecolor{codepurple}{rgb}{0.58,0,0.82}
\definecolor{backcolour}{rgb}{0.95,0.95,0.92}

\lstdefinestyle{mystyle}{
    backgroundcolor=\color{backcolour},   
    commentstyle=\color{codegreen},
    keywordstyle=\color{magenta},
    numberstyle=\tiny\color{codegray},
    stringstyle=\color{codepurple},
    basicstyle=\ttfamily\footnotesize,
    breakatwhitespace=false,         
    breaklines=true,                 
    captionpos=b,                    
    keepspaces=true,                 
    numbers=left,                    
    numbersep=5pt,                  
    showspaces=false,                
    showstringspaces=false,
    showtabs=false,                  
    tabsize=2
}

\lstset{style=mystyle}

%% -----

% mostra le subsubsection nell'indice
\setcounter{tocdepth}{3}
\setcounter{secnumdepth}{3}

% Resetta la numerazione dei chapter quando
% una nuova part viene creata
\makeatletter
\@addtoreset{chapter}{part}
\makeatother

% Rimuove l'indentazione quando si crea un nuovo paragrafo
\setlength{\parindent}{0pt}

% footer
\pagestyle{fancyplain}
% rimuove la riga nell'header
\fancyhf{} % sets both header and footer to nothing
\renewcommand{\headrulewidth}{0pt}
\fancyfoot[L]{\href{https://github.com/Typing-Monkeys/AppuntiUniversita}{Typing Monkeys}}
\fancyfoot[C]{\emoji{gorilla}}
\fancyfoot[R]{\thepage}

% configurazione emoji
\usepackage{fontspec}
\usepackage{emoji}
\setemojifont{NotoColorEmoji.ttf}[Path=/usr/share/fonts/truetype/noto/]

\newtheorem{definition}{Definizione}
\newtheorem{lemma}{Lemma}
\newtheorem{theorem}{Teorema}
\newtheorem{corollary}{Corollario}

%% cambio nome al comando proof
\renewcommand*{\proofname}{Dimostrazione}

\begin{document}
\include{frontmatter/main.tex}

\tableofcontents

\include{quote/main.tex}

%% Aggiungere i capitoli qui sotto
\include{capitoli/richiami/main.tex}
\include{capitoli/immagini/main.tex}

\end{document}

\documentclass[a4paper,12 pt]{report}
\usepackage[T1]{fontenc}
\usepackage[utf8]{inputenc}
\usepackage{lmodern}
\usepackage{listings}

\usepackage{float}
\usepackage{subcaption}
\usepackage{wrapfig}
\usepackage{fancyhdr}
\usepackage{amsthm}

\usepackage{pgfplots}
\setlength {\marginparwidth }{2cm}
\usepackage{todonotes}
\newcommand{\TODO}[2][]
{\todo[size=\scriptsize, color=red, #1]{#2}}


asdf
\pgfplotsset{compat=1.18}
% forza le footnote a stare il più in basso possibile
\usepackage[bottom]{footmisc}


%% STILE LISTINGS
%%aaa
\usepackage{xcolor}

\definecolor{codegreen}{rgb}{0,0.6,0}
\definecolor{codegray}{rgb}{0.5,0.5,0.5}
\definecolor{codepurple}{rgb}{0.58,0,0.82}
\definecolor{backcolour}{rgb}{0.95,0.95,0.92}

\lstdefinestyle{mystyle}{
    backgroundcolor=\color{backcolour},   
    commentstyle=\color{codegreen},
    keywordstyle=\color{magenta},
    numberstyle=\tiny\color{codegray},
    stringstyle=\color{codepurple},
    basicstyle=\ttfamily\footnotesize,
    breakatwhitespace=false,         
    breaklines=true,                 
    captionpos=b,                    
    keepspaces=true,                 
    numbers=left,                    
    numbersep=5pt,                  
    showspaces=false,                
    showstringspaces=false,
    showtabs=false,                  
    tabsize=2
}

\lstset{style=mystyle}

%% -----

% mostra le subsubsection nell'indice
\setcounter{tocdepth}{3}
\setcounter{secnumdepth}{3}

% Resetta la numerazione dei chapter quando
% una nuova part viene creata
\makeatletter
\@addtoreset{chapter}{part}
\makeatother

% Rimuove l'indentazione quando si crea un nuovo paragrafo
\setlength{\parindent}{0pt}

% footer
\pagestyle{fancyplain}
% rimuove la riga nell'header
\fancyhf{} % sets both header and footer to nothing
\renewcommand{\headrulewidth}{0pt}
\fancyfoot[L]{\href{https://github.com/Typing-Monkeys/AppuntiUniversita}{Typing Monkeys}}
\fancyfoot[C]{\emoji{gorilla}}
\fancyfoot[R]{\thepage}

% configurazione emoji
\usepackage{fontspec}
\usepackage{emoji}
\setemojifont{NotoColorEmoji.ttf}[Path=/usr/share/fonts/truetype/noto/]

\newtheorem{definition}{Definizione}
\newtheorem{lemma}{Lemma}
\newtheorem{theorem}{Teorema}
\newtheorem{corollary}{Corollario}

%% cambio nome al comando proof
\renewcommand*{\proofname}{Dimostrazione}

\begin{document}
\include{frontmatter/main.tex}

\tableofcontents

\include{quote/main.tex}

%% Aggiungere i capitoli qui sotto
\include{capitoli/richiami/main.tex}
\include{capitoli/immagini/main.tex}

\end{document}


\end{document}

\documentclass[a4paper,12 pt]{report}
\usepackage[T1]{fontenc}
\usepackage[utf8]{inputenc}
\usepackage{lmodern}
\usepackage{listings}

\usepackage{float}
\usepackage{subcaption}
\usepackage{wrapfig}
\usepackage{fancyhdr}
\usepackage{amsthm}

\usepackage{pgfplots}
\setlength {\marginparwidth }{2cm}
\usepackage{todonotes}
\newcommand{\TODO}[2][]
{\todo[size=\scriptsize, color=red, #1]{#2}}


asdf
\pgfplotsset{compat=1.18}
% forza le footnote a stare il più in basso possibile
\usepackage[bottom]{footmisc}


%% STILE LISTINGS
%%aaa
\usepackage{xcolor}

\definecolor{codegreen}{rgb}{0,0.6,0}
\definecolor{codegray}{rgb}{0.5,0.5,0.5}
\definecolor{codepurple}{rgb}{0.58,0,0.82}
\definecolor{backcolour}{rgb}{0.95,0.95,0.92}

\lstdefinestyle{mystyle}{
    backgroundcolor=\color{backcolour},   
    commentstyle=\color{codegreen},
    keywordstyle=\color{magenta},
    numberstyle=\tiny\color{codegray},
    stringstyle=\color{codepurple},
    basicstyle=\ttfamily\footnotesize,
    breakatwhitespace=false,         
    breaklines=true,                 
    captionpos=b,                    
    keepspaces=true,                 
    numbers=left,                    
    numbersep=5pt,                  
    showspaces=false,                
    showstringspaces=false,
    showtabs=false,                  
    tabsize=2
}

\lstset{style=mystyle}

%% -----

% mostra le subsubsection nell'indice
\setcounter{tocdepth}{3}
\setcounter{secnumdepth}{3}

% Resetta la numerazione dei chapter quando
% una nuova part viene creata
\makeatletter
\@addtoreset{chapter}{part}
\makeatother

% Rimuove l'indentazione quando si crea un nuovo paragrafo
\setlength{\parindent}{0pt}

% footer
\pagestyle{fancyplain}
% rimuove la riga nell'header
\fancyhf{} % sets both header and footer to nothing
\renewcommand{\headrulewidth}{0pt}
\fancyfoot[L]{\href{https://github.com/Typing-Monkeys/AppuntiUniversita}{Typing Monkeys}}
\fancyfoot[C]{\emoji{gorilla}}
\fancyfoot[R]{\thepage}

% configurazione emoji
\usepackage{fontspec}
\usepackage{emoji}
\setemojifont{NotoColorEmoji.ttf}[Path=/usr/share/fonts/truetype/noto/]

\newtheorem{definition}{Definizione}
\newtheorem{lemma}{Lemma}
\newtheorem{theorem}{Teorema}
\newtheorem{corollary}{Corollario}

%% cambio nome al comando proof
\renewcommand*{\proofname}{Dimostrazione}

\begin{document}
\documentclass[a4paper,12 pt]{report}
\usepackage[T1]{fontenc}
\usepackage[utf8]{inputenc}
\usepackage{lmodern}
\usepackage{listings}

\usepackage{float}
\usepackage{subcaption}
\usepackage{wrapfig}
\usepackage{fancyhdr}
\usepackage{amsthm}

\usepackage{pgfplots}
\setlength {\marginparwidth }{2cm}
\usepackage{todonotes}
\newcommand{\TODO}[2][]
{\todo[size=\scriptsize, color=red, #1]{#2}}


asdf
\pgfplotsset{compat=1.18}
% forza le footnote a stare il più in basso possibile
\usepackage[bottom]{footmisc}


%% STILE LISTINGS
%%aaa
\usepackage{xcolor}

\definecolor{codegreen}{rgb}{0,0.6,0}
\definecolor{codegray}{rgb}{0.5,0.5,0.5}
\definecolor{codepurple}{rgb}{0.58,0,0.82}
\definecolor{backcolour}{rgb}{0.95,0.95,0.92}

\lstdefinestyle{mystyle}{
    backgroundcolor=\color{backcolour},   
    commentstyle=\color{codegreen},
    keywordstyle=\color{magenta},
    numberstyle=\tiny\color{codegray},
    stringstyle=\color{codepurple},
    basicstyle=\ttfamily\footnotesize,
    breakatwhitespace=false,         
    breaklines=true,                 
    captionpos=b,                    
    keepspaces=true,                 
    numbers=left,                    
    numbersep=5pt,                  
    showspaces=false,                
    showstringspaces=false,
    showtabs=false,                  
    tabsize=2
}

\lstset{style=mystyle}

%% -----

% mostra le subsubsection nell'indice
\setcounter{tocdepth}{3}
\setcounter{secnumdepth}{3}

% Resetta la numerazione dei chapter quando
% una nuova part viene creata
\makeatletter
\@addtoreset{chapter}{part}
\makeatother

% Rimuove l'indentazione quando si crea un nuovo paragrafo
\setlength{\parindent}{0pt}

% footer
\pagestyle{fancyplain}
% rimuove la riga nell'header
\fancyhf{} % sets both header and footer to nothing
\renewcommand{\headrulewidth}{0pt}
\fancyfoot[L]{\href{https://github.com/Typing-Monkeys/AppuntiUniversita}{Typing Monkeys}}
\fancyfoot[C]{\emoji{gorilla}}
\fancyfoot[R]{\thepage}

% configurazione emoji
\usepackage{fontspec}
\usepackage{emoji}
\setemojifont{NotoColorEmoji.ttf}[Path=/usr/share/fonts/truetype/noto/]

\newtheorem{definition}{Definizione}
\newtheorem{lemma}{Lemma}
\newtheorem{theorem}{Teorema}
\newtheorem{corollary}{Corollario}

%% cambio nome al comando proof
\renewcommand*{\proofname}{Dimostrazione}

\begin{document}
\include{frontmatter/main.tex}

\tableofcontents

\include{quote/main.tex}

%% Aggiungere i capitoli qui sotto
\include{capitoli/richiami/main.tex}
\include{capitoli/immagini/main.tex}

\end{document}


\tableofcontents

\documentclass[a4paper,12 pt]{report}
\usepackage[T1]{fontenc}
\usepackage[utf8]{inputenc}
\usepackage{lmodern}
\usepackage{listings}

\usepackage{float}
\usepackage{subcaption}
\usepackage{wrapfig}
\usepackage{fancyhdr}
\usepackage{amsthm}

\usepackage{pgfplots}
\setlength {\marginparwidth }{2cm}
\usepackage{todonotes}
\newcommand{\TODO}[2][]
{\todo[size=\scriptsize, color=red, #1]{#2}}


asdf
\pgfplotsset{compat=1.18}
% forza le footnote a stare il più in basso possibile
\usepackage[bottom]{footmisc}


%% STILE LISTINGS
%%aaa
\usepackage{xcolor}

\definecolor{codegreen}{rgb}{0,0.6,0}
\definecolor{codegray}{rgb}{0.5,0.5,0.5}
\definecolor{codepurple}{rgb}{0.58,0,0.82}
\definecolor{backcolour}{rgb}{0.95,0.95,0.92}

\lstdefinestyle{mystyle}{
    backgroundcolor=\color{backcolour},   
    commentstyle=\color{codegreen},
    keywordstyle=\color{magenta},
    numberstyle=\tiny\color{codegray},
    stringstyle=\color{codepurple},
    basicstyle=\ttfamily\footnotesize,
    breakatwhitespace=false,         
    breaklines=true,                 
    captionpos=b,                    
    keepspaces=true,                 
    numbers=left,                    
    numbersep=5pt,                  
    showspaces=false,                
    showstringspaces=false,
    showtabs=false,                  
    tabsize=2
}

\lstset{style=mystyle}

%% -----

% mostra le subsubsection nell'indice
\setcounter{tocdepth}{3}
\setcounter{secnumdepth}{3}

% Resetta la numerazione dei chapter quando
% una nuova part viene creata
\makeatletter
\@addtoreset{chapter}{part}
\makeatother

% Rimuove l'indentazione quando si crea un nuovo paragrafo
\setlength{\parindent}{0pt}

% footer
\pagestyle{fancyplain}
% rimuove la riga nell'header
\fancyhf{} % sets both header and footer to nothing
\renewcommand{\headrulewidth}{0pt}
\fancyfoot[L]{\href{https://github.com/Typing-Monkeys/AppuntiUniversita}{Typing Monkeys}}
\fancyfoot[C]{\emoji{gorilla}}
\fancyfoot[R]{\thepage}

% configurazione emoji
\usepackage{fontspec}
\usepackage{emoji}
\setemojifont{NotoColorEmoji.ttf}[Path=/usr/share/fonts/truetype/noto/]

\newtheorem{definition}{Definizione}
\newtheorem{lemma}{Lemma}
\newtheorem{theorem}{Teorema}
\newtheorem{corollary}{Corollario}

%% cambio nome al comando proof
\renewcommand*{\proofname}{Dimostrazione}

\begin{document}
\include{frontmatter/main.tex}

\tableofcontents

\include{quote/main.tex}

%% Aggiungere i capitoli qui sotto
\include{capitoli/richiami/main.tex}
\include{capitoli/immagini/main.tex}

\end{document}


%% Aggiungere i capitoli qui sotto
\documentclass[a4paper,12 pt]{report}
\usepackage[T1]{fontenc}
\usepackage[utf8]{inputenc}
\usepackage{lmodern}
\usepackage{listings}

\usepackage{float}
\usepackage{subcaption}
\usepackage{wrapfig}
\usepackage{fancyhdr}
\usepackage{amsthm}

\usepackage{pgfplots}
\setlength {\marginparwidth }{2cm}
\usepackage{todonotes}
\newcommand{\TODO}[2][]
{\todo[size=\scriptsize, color=red, #1]{#2}}


asdf
\pgfplotsset{compat=1.18}
% forza le footnote a stare il più in basso possibile
\usepackage[bottom]{footmisc}


%% STILE LISTINGS
%%aaa
\usepackage{xcolor}

\definecolor{codegreen}{rgb}{0,0.6,0}
\definecolor{codegray}{rgb}{0.5,0.5,0.5}
\definecolor{codepurple}{rgb}{0.58,0,0.82}
\definecolor{backcolour}{rgb}{0.95,0.95,0.92}

\lstdefinestyle{mystyle}{
    backgroundcolor=\color{backcolour},   
    commentstyle=\color{codegreen},
    keywordstyle=\color{magenta},
    numberstyle=\tiny\color{codegray},
    stringstyle=\color{codepurple},
    basicstyle=\ttfamily\footnotesize,
    breakatwhitespace=false,         
    breaklines=true,                 
    captionpos=b,                    
    keepspaces=true,                 
    numbers=left,                    
    numbersep=5pt,                  
    showspaces=false,                
    showstringspaces=false,
    showtabs=false,                  
    tabsize=2
}

\lstset{style=mystyle}

%% -----

% mostra le subsubsection nell'indice
\setcounter{tocdepth}{3}
\setcounter{secnumdepth}{3}

% Resetta la numerazione dei chapter quando
% una nuova part viene creata
\makeatletter
\@addtoreset{chapter}{part}
\makeatother

% Rimuove l'indentazione quando si crea un nuovo paragrafo
\setlength{\parindent}{0pt}

% footer
\pagestyle{fancyplain}
% rimuove la riga nell'header
\fancyhf{} % sets both header and footer to nothing
\renewcommand{\headrulewidth}{0pt}
\fancyfoot[L]{\href{https://github.com/Typing-Monkeys/AppuntiUniversita}{Typing Monkeys}}
\fancyfoot[C]{\emoji{gorilla}}
\fancyfoot[R]{\thepage}

% configurazione emoji
\usepackage{fontspec}
\usepackage{emoji}
\setemojifont{NotoColorEmoji.ttf}[Path=/usr/share/fonts/truetype/noto/]

\newtheorem{definition}{Definizione}
\newtheorem{lemma}{Lemma}
\newtheorem{theorem}{Teorema}
\newtheorem{corollary}{Corollario}

%% cambio nome al comando proof
\renewcommand*{\proofname}{Dimostrazione}

\begin{document}
\include{frontmatter/main.tex}

\tableofcontents

\include{quote/main.tex}

%% Aggiungere i capitoli qui sotto
\include{capitoli/richiami/main.tex}
\include{capitoli/immagini/main.tex}

\end{document}

\documentclass[a4paper,12 pt]{report}
\usepackage[T1]{fontenc}
\usepackage[utf8]{inputenc}
\usepackage{lmodern}
\usepackage{listings}

\usepackage{float}
\usepackage{subcaption}
\usepackage{wrapfig}
\usepackage{fancyhdr}
\usepackage{amsthm}

\usepackage{pgfplots}
\setlength {\marginparwidth }{2cm}
\usepackage{todonotes}
\newcommand{\TODO}[2][]
{\todo[size=\scriptsize, color=red, #1]{#2}}


asdf
\pgfplotsset{compat=1.18}
% forza le footnote a stare il più in basso possibile
\usepackage[bottom]{footmisc}


%% STILE LISTINGS
%%aaa
\usepackage{xcolor}

\definecolor{codegreen}{rgb}{0,0.6,0}
\definecolor{codegray}{rgb}{0.5,0.5,0.5}
\definecolor{codepurple}{rgb}{0.58,0,0.82}
\definecolor{backcolour}{rgb}{0.95,0.95,0.92}

\lstdefinestyle{mystyle}{
    backgroundcolor=\color{backcolour},   
    commentstyle=\color{codegreen},
    keywordstyle=\color{magenta},
    numberstyle=\tiny\color{codegray},
    stringstyle=\color{codepurple},
    basicstyle=\ttfamily\footnotesize,
    breakatwhitespace=false,         
    breaklines=true,                 
    captionpos=b,                    
    keepspaces=true,                 
    numbers=left,                    
    numbersep=5pt,                  
    showspaces=false,                
    showstringspaces=false,
    showtabs=false,                  
    tabsize=2
}

\lstset{style=mystyle}

%% -----

% mostra le subsubsection nell'indice
\setcounter{tocdepth}{3}
\setcounter{secnumdepth}{3}

% Resetta la numerazione dei chapter quando
% una nuova part viene creata
\makeatletter
\@addtoreset{chapter}{part}
\makeatother

% Rimuove l'indentazione quando si crea un nuovo paragrafo
\setlength{\parindent}{0pt}

% footer
\pagestyle{fancyplain}
% rimuove la riga nell'header
\fancyhf{} % sets both header and footer to nothing
\renewcommand{\headrulewidth}{0pt}
\fancyfoot[L]{\href{https://github.com/Typing-Monkeys/AppuntiUniversita}{Typing Monkeys}}
\fancyfoot[C]{\emoji{gorilla}}
\fancyfoot[R]{\thepage}

% configurazione emoji
\usepackage{fontspec}
\usepackage{emoji}
\setemojifont{NotoColorEmoji.ttf}[Path=/usr/share/fonts/truetype/noto/]

\newtheorem{definition}{Definizione}
\newtheorem{lemma}{Lemma}
\newtheorem{theorem}{Teorema}
\newtheorem{corollary}{Corollario}

%% cambio nome al comando proof
\renewcommand*{\proofname}{Dimostrazione}

\begin{document}
\include{frontmatter/main.tex}

\tableofcontents

\include{quote/main.tex}

%% Aggiungere i capitoli qui sotto
\include{capitoli/richiami/main.tex}
\include{capitoli/immagini/main.tex}

\end{document}


\end{document}


\end{document}


\tableofcontents

\documentclass[a4paper,12 pt]{report}
\usepackage[T1]{fontenc}
\usepackage[utf8]{inputenc}
\usepackage{lmodern}
\usepackage{listings}

\usepackage{float}
\usepackage{subcaption}
\usepackage{wrapfig}
\usepackage{fancyhdr}
\usepackage{amsthm}

\usepackage{pgfplots}
\setlength {\marginparwidth }{2cm}
\usepackage{todonotes}
\newcommand{\TODO}[2][]
{\todo[size=\scriptsize, color=red, #1]{#2}}


asdf
\pgfplotsset{compat=1.18}
% forza le footnote a stare il più in basso possibile
\usepackage[bottom]{footmisc}


%% STILE LISTINGS
%%aaa
\usepackage{xcolor}

\definecolor{codegreen}{rgb}{0,0.6,0}
\definecolor{codegray}{rgb}{0.5,0.5,0.5}
\definecolor{codepurple}{rgb}{0.58,0,0.82}
\definecolor{backcolour}{rgb}{0.95,0.95,0.92}

\lstdefinestyle{mystyle}{
    backgroundcolor=\color{backcolour},   
    commentstyle=\color{codegreen},
    keywordstyle=\color{magenta},
    numberstyle=\tiny\color{codegray},
    stringstyle=\color{codepurple},
    basicstyle=\ttfamily\footnotesize,
    breakatwhitespace=false,         
    breaklines=true,                 
    captionpos=b,                    
    keepspaces=true,                 
    numbers=left,                    
    numbersep=5pt,                  
    showspaces=false,                
    showstringspaces=false,
    showtabs=false,                  
    tabsize=2
}

\lstset{style=mystyle}

%% -----

% mostra le subsubsection nell'indice
\setcounter{tocdepth}{3}
\setcounter{secnumdepth}{3}

% Resetta la numerazione dei chapter quando
% una nuova part viene creata
\makeatletter
\@addtoreset{chapter}{part}
\makeatother

% Rimuove l'indentazione quando si crea un nuovo paragrafo
\setlength{\parindent}{0pt}

% footer
\pagestyle{fancyplain}
% rimuove la riga nell'header
\fancyhf{} % sets both header and footer to nothing
\renewcommand{\headrulewidth}{0pt}
\fancyfoot[L]{\href{https://github.com/Typing-Monkeys/AppuntiUniversita}{Typing Monkeys}}
\fancyfoot[C]{\emoji{gorilla}}
\fancyfoot[R]{\thepage}

% configurazione emoji
\usepackage{fontspec}
\usepackage{emoji}
\setemojifont{NotoColorEmoji.ttf}[Path=/usr/share/fonts/truetype/noto/]

\newtheorem{definition}{Definizione}
\newtheorem{lemma}{Lemma}
\newtheorem{theorem}{Teorema}
\newtheorem{corollary}{Corollario}

%% cambio nome al comando proof
\renewcommand*{\proofname}{Dimostrazione}

\begin{document}
\documentclass[a4paper,12 pt]{report}
\usepackage[T1]{fontenc}
\usepackage[utf8]{inputenc}
\usepackage{lmodern}
\usepackage{listings}

\usepackage{float}
\usepackage{subcaption}
\usepackage{wrapfig}
\usepackage{fancyhdr}
\usepackage{amsthm}

\usepackage{pgfplots}
\setlength {\marginparwidth }{2cm}
\usepackage{todonotes}
\newcommand{\TODO}[2][]
{\todo[size=\scriptsize, color=red, #1]{#2}}


asdf
\pgfplotsset{compat=1.18}
% forza le footnote a stare il più in basso possibile
\usepackage[bottom]{footmisc}


%% STILE LISTINGS
%%aaa
\usepackage{xcolor}

\definecolor{codegreen}{rgb}{0,0.6,0}
\definecolor{codegray}{rgb}{0.5,0.5,0.5}
\definecolor{codepurple}{rgb}{0.58,0,0.82}
\definecolor{backcolour}{rgb}{0.95,0.95,0.92}

\lstdefinestyle{mystyle}{
    backgroundcolor=\color{backcolour},   
    commentstyle=\color{codegreen},
    keywordstyle=\color{magenta},
    numberstyle=\tiny\color{codegray},
    stringstyle=\color{codepurple},
    basicstyle=\ttfamily\footnotesize,
    breakatwhitespace=false,         
    breaklines=true,                 
    captionpos=b,                    
    keepspaces=true,                 
    numbers=left,                    
    numbersep=5pt,                  
    showspaces=false,                
    showstringspaces=false,
    showtabs=false,                  
    tabsize=2
}

\lstset{style=mystyle}

%% -----

% mostra le subsubsection nell'indice
\setcounter{tocdepth}{3}
\setcounter{secnumdepth}{3}

% Resetta la numerazione dei chapter quando
% una nuova part viene creata
\makeatletter
\@addtoreset{chapter}{part}
\makeatother

% Rimuove l'indentazione quando si crea un nuovo paragrafo
\setlength{\parindent}{0pt}

% footer
\pagestyle{fancyplain}
% rimuove la riga nell'header
\fancyhf{} % sets both header and footer to nothing
\renewcommand{\headrulewidth}{0pt}
\fancyfoot[L]{\href{https://github.com/Typing-Monkeys/AppuntiUniversita}{Typing Monkeys}}
\fancyfoot[C]{\emoji{gorilla}}
\fancyfoot[R]{\thepage}

% configurazione emoji
\usepackage{fontspec}
\usepackage{emoji}
\setemojifont{NotoColorEmoji.ttf}[Path=/usr/share/fonts/truetype/noto/]

\newtheorem{definition}{Definizione}
\newtheorem{lemma}{Lemma}
\newtheorem{theorem}{Teorema}
\newtheorem{corollary}{Corollario}

%% cambio nome al comando proof
\renewcommand*{\proofname}{Dimostrazione}

\begin{document}
\documentclass[a4paper,12 pt]{report}
\usepackage[T1]{fontenc}
\usepackage[utf8]{inputenc}
\usepackage{lmodern}
\usepackage{listings}

\usepackage{float}
\usepackage{subcaption}
\usepackage{wrapfig}
\usepackage{fancyhdr}
\usepackage{amsthm}

\usepackage{pgfplots}
\setlength {\marginparwidth }{2cm}
\usepackage{todonotes}
\newcommand{\TODO}[2][]
{\todo[size=\scriptsize, color=red, #1]{#2}}


asdf
\pgfplotsset{compat=1.18}
% forza le footnote a stare il più in basso possibile
\usepackage[bottom]{footmisc}


%% STILE LISTINGS
%%aaa
\usepackage{xcolor}

\definecolor{codegreen}{rgb}{0,0.6,0}
\definecolor{codegray}{rgb}{0.5,0.5,0.5}
\definecolor{codepurple}{rgb}{0.58,0,0.82}
\definecolor{backcolour}{rgb}{0.95,0.95,0.92}

\lstdefinestyle{mystyle}{
    backgroundcolor=\color{backcolour},   
    commentstyle=\color{codegreen},
    keywordstyle=\color{magenta},
    numberstyle=\tiny\color{codegray},
    stringstyle=\color{codepurple},
    basicstyle=\ttfamily\footnotesize,
    breakatwhitespace=false,         
    breaklines=true,                 
    captionpos=b,                    
    keepspaces=true,                 
    numbers=left,                    
    numbersep=5pt,                  
    showspaces=false,                
    showstringspaces=false,
    showtabs=false,                  
    tabsize=2
}

\lstset{style=mystyle}

%% -----

% mostra le subsubsection nell'indice
\setcounter{tocdepth}{3}
\setcounter{secnumdepth}{3}

% Resetta la numerazione dei chapter quando
% una nuova part viene creata
\makeatletter
\@addtoreset{chapter}{part}
\makeatother

% Rimuove l'indentazione quando si crea un nuovo paragrafo
\setlength{\parindent}{0pt}

% footer
\pagestyle{fancyplain}
% rimuove la riga nell'header
\fancyhf{} % sets both header and footer to nothing
\renewcommand{\headrulewidth}{0pt}
\fancyfoot[L]{\href{https://github.com/Typing-Monkeys/AppuntiUniversita}{Typing Monkeys}}
\fancyfoot[C]{\emoji{gorilla}}
\fancyfoot[R]{\thepage}

% configurazione emoji
\usepackage{fontspec}
\usepackage{emoji}
\setemojifont{NotoColorEmoji.ttf}[Path=/usr/share/fonts/truetype/noto/]

\newtheorem{definition}{Definizione}
\newtheorem{lemma}{Lemma}
\newtheorem{theorem}{Teorema}
\newtheorem{corollary}{Corollario}

%% cambio nome al comando proof
\renewcommand*{\proofname}{Dimostrazione}

\begin{document}
\include{frontmatter/main.tex}

\tableofcontents

\include{quote/main.tex}

%% Aggiungere i capitoli qui sotto
\include{capitoli/richiami/main.tex}
\include{capitoli/immagini/main.tex}

\end{document}


\tableofcontents

\documentclass[a4paper,12 pt]{report}
\usepackage[T1]{fontenc}
\usepackage[utf8]{inputenc}
\usepackage{lmodern}
\usepackage{listings}

\usepackage{float}
\usepackage{subcaption}
\usepackage{wrapfig}
\usepackage{fancyhdr}
\usepackage{amsthm}

\usepackage{pgfplots}
\setlength {\marginparwidth }{2cm}
\usepackage{todonotes}
\newcommand{\TODO}[2][]
{\todo[size=\scriptsize, color=red, #1]{#2}}


asdf
\pgfplotsset{compat=1.18}
% forza le footnote a stare il più in basso possibile
\usepackage[bottom]{footmisc}


%% STILE LISTINGS
%%aaa
\usepackage{xcolor}

\definecolor{codegreen}{rgb}{0,0.6,0}
\definecolor{codegray}{rgb}{0.5,0.5,0.5}
\definecolor{codepurple}{rgb}{0.58,0,0.82}
\definecolor{backcolour}{rgb}{0.95,0.95,0.92}

\lstdefinestyle{mystyle}{
    backgroundcolor=\color{backcolour},   
    commentstyle=\color{codegreen},
    keywordstyle=\color{magenta},
    numberstyle=\tiny\color{codegray},
    stringstyle=\color{codepurple},
    basicstyle=\ttfamily\footnotesize,
    breakatwhitespace=false,         
    breaklines=true,                 
    captionpos=b,                    
    keepspaces=true,                 
    numbers=left,                    
    numbersep=5pt,                  
    showspaces=false,                
    showstringspaces=false,
    showtabs=false,                  
    tabsize=2
}

\lstset{style=mystyle}

%% -----

% mostra le subsubsection nell'indice
\setcounter{tocdepth}{3}
\setcounter{secnumdepth}{3}

% Resetta la numerazione dei chapter quando
% una nuova part viene creata
\makeatletter
\@addtoreset{chapter}{part}
\makeatother

% Rimuove l'indentazione quando si crea un nuovo paragrafo
\setlength{\parindent}{0pt}

% footer
\pagestyle{fancyplain}
% rimuove la riga nell'header
\fancyhf{} % sets both header and footer to nothing
\renewcommand{\headrulewidth}{0pt}
\fancyfoot[L]{\href{https://github.com/Typing-Monkeys/AppuntiUniversita}{Typing Monkeys}}
\fancyfoot[C]{\emoji{gorilla}}
\fancyfoot[R]{\thepage}

% configurazione emoji
\usepackage{fontspec}
\usepackage{emoji}
\setemojifont{NotoColorEmoji.ttf}[Path=/usr/share/fonts/truetype/noto/]

\newtheorem{definition}{Definizione}
\newtheorem{lemma}{Lemma}
\newtheorem{theorem}{Teorema}
\newtheorem{corollary}{Corollario}

%% cambio nome al comando proof
\renewcommand*{\proofname}{Dimostrazione}

\begin{document}
\include{frontmatter/main.tex}

\tableofcontents

\include{quote/main.tex}

%% Aggiungere i capitoli qui sotto
\include{capitoli/richiami/main.tex}
\include{capitoli/immagini/main.tex}

\end{document}


%% Aggiungere i capitoli qui sotto
\documentclass[a4paper,12 pt]{report}
\usepackage[T1]{fontenc}
\usepackage[utf8]{inputenc}
\usepackage{lmodern}
\usepackage{listings}

\usepackage{float}
\usepackage{subcaption}
\usepackage{wrapfig}
\usepackage{fancyhdr}
\usepackage{amsthm}

\usepackage{pgfplots}
\setlength {\marginparwidth }{2cm}
\usepackage{todonotes}
\newcommand{\TODO}[2][]
{\todo[size=\scriptsize, color=red, #1]{#2}}


asdf
\pgfplotsset{compat=1.18}
% forza le footnote a stare il più in basso possibile
\usepackage[bottom]{footmisc}


%% STILE LISTINGS
%%aaa
\usepackage{xcolor}

\definecolor{codegreen}{rgb}{0,0.6,0}
\definecolor{codegray}{rgb}{0.5,0.5,0.5}
\definecolor{codepurple}{rgb}{0.58,0,0.82}
\definecolor{backcolour}{rgb}{0.95,0.95,0.92}

\lstdefinestyle{mystyle}{
    backgroundcolor=\color{backcolour},   
    commentstyle=\color{codegreen},
    keywordstyle=\color{magenta},
    numberstyle=\tiny\color{codegray},
    stringstyle=\color{codepurple},
    basicstyle=\ttfamily\footnotesize,
    breakatwhitespace=false,         
    breaklines=true,                 
    captionpos=b,                    
    keepspaces=true,                 
    numbers=left,                    
    numbersep=5pt,                  
    showspaces=false,                
    showstringspaces=false,
    showtabs=false,                  
    tabsize=2
}

\lstset{style=mystyle}

%% -----

% mostra le subsubsection nell'indice
\setcounter{tocdepth}{3}
\setcounter{secnumdepth}{3}

% Resetta la numerazione dei chapter quando
% una nuova part viene creata
\makeatletter
\@addtoreset{chapter}{part}
\makeatother

% Rimuove l'indentazione quando si crea un nuovo paragrafo
\setlength{\parindent}{0pt}

% footer
\pagestyle{fancyplain}
% rimuove la riga nell'header
\fancyhf{} % sets both header and footer to nothing
\renewcommand{\headrulewidth}{0pt}
\fancyfoot[L]{\href{https://github.com/Typing-Monkeys/AppuntiUniversita}{Typing Monkeys}}
\fancyfoot[C]{\emoji{gorilla}}
\fancyfoot[R]{\thepage}

% configurazione emoji
\usepackage{fontspec}
\usepackage{emoji}
\setemojifont{NotoColorEmoji.ttf}[Path=/usr/share/fonts/truetype/noto/]

\newtheorem{definition}{Definizione}
\newtheorem{lemma}{Lemma}
\newtheorem{theorem}{Teorema}
\newtheorem{corollary}{Corollario}

%% cambio nome al comando proof
\renewcommand*{\proofname}{Dimostrazione}

\begin{document}
\include{frontmatter/main.tex}

\tableofcontents

\include{quote/main.tex}

%% Aggiungere i capitoli qui sotto
\include{capitoli/richiami/main.tex}
\include{capitoli/immagini/main.tex}

\end{document}

\documentclass[a4paper,12 pt]{report}
\usepackage[T1]{fontenc}
\usepackage[utf8]{inputenc}
\usepackage{lmodern}
\usepackage{listings}

\usepackage{float}
\usepackage{subcaption}
\usepackage{wrapfig}
\usepackage{fancyhdr}
\usepackage{amsthm}

\usepackage{pgfplots}
\setlength {\marginparwidth }{2cm}
\usepackage{todonotes}
\newcommand{\TODO}[2][]
{\todo[size=\scriptsize, color=red, #1]{#2}}


asdf
\pgfplotsset{compat=1.18}
% forza le footnote a stare il più in basso possibile
\usepackage[bottom]{footmisc}


%% STILE LISTINGS
%%aaa
\usepackage{xcolor}

\definecolor{codegreen}{rgb}{0,0.6,0}
\definecolor{codegray}{rgb}{0.5,0.5,0.5}
\definecolor{codepurple}{rgb}{0.58,0,0.82}
\definecolor{backcolour}{rgb}{0.95,0.95,0.92}

\lstdefinestyle{mystyle}{
    backgroundcolor=\color{backcolour},   
    commentstyle=\color{codegreen},
    keywordstyle=\color{magenta},
    numberstyle=\tiny\color{codegray},
    stringstyle=\color{codepurple},
    basicstyle=\ttfamily\footnotesize,
    breakatwhitespace=false,         
    breaklines=true,                 
    captionpos=b,                    
    keepspaces=true,                 
    numbers=left,                    
    numbersep=5pt,                  
    showspaces=false,                
    showstringspaces=false,
    showtabs=false,                  
    tabsize=2
}

\lstset{style=mystyle}

%% -----

% mostra le subsubsection nell'indice
\setcounter{tocdepth}{3}
\setcounter{secnumdepth}{3}

% Resetta la numerazione dei chapter quando
% una nuova part viene creata
\makeatletter
\@addtoreset{chapter}{part}
\makeatother

% Rimuove l'indentazione quando si crea un nuovo paragrafo
\setlength{\parindent}{0pt}

% footer
\pagestyle{fancyplain}
% rimuove la riga nell'header
\fancyhf{} % sets both header and footer to nothing
\renewcommand{\headrulewidth}{0pt}
\fancyfoot[L]{\href{https://github.com/Typing-Monkeys/AppuntiUniversita}{Typing Monkeys}}
\fancyfoot[C]{\emoji{gorilla}}
\fancyfoot[R]{\thepage}

% configurazione emoji
\usepackage{fontspec}
\usepackage{emoji}
\setemojifont{NotoColorEmoji.ttf}[Path=/usr/share/fonts/truetype/noto/]

\newtheorem{definition}{Definizione}
\newtheorem{lemma}{Lemma}
\newtheorem{theorem}{Teorema}
\newtheorem{corollary}{Corollario}

%% cambio nome al comando proof
\renewcommand*{\proofname}{Dimostrazione}

\begin{document}
\include{frontmatter/main.tex}

\tableofcontents

\include{quote/main.tex}

%% Aggiungere i capitoli qui sotto
\include{capitoli/richiami/main.tex}
\include{capitoli/immagini/main.tex}

\end{document}


\end{document}


\tableofcontents

\documentclass[a4paper,12 pt]{report}
\usepackage[T1]{fontenc}
\usepackage[utf8]{inputenc}
\usepackage{lmodern}
\usepackage{listings}

\usepackage{float}
\usepackage{subcaption}
\usepackage{wrapfig}
\usepackage{fancyhdr}
\usepackage{amsthm}

\usepackage{pgfplots}
\setlength {\marginparwidth }{2cm}
\usepackage{todonotes}
\newcommand{\TODO}[2][]
{\todo[size=\scriptsize, color=red, #1]{#2}}


asdf
\pgfplotsset{compat=1.18}
% forza le footnote a stare il più in basso possibile
\usepackage[bottom]{footmisc}


%% STILE LISTINGS
%%aaa
\usepackage{xcolor}

\definecolor{codegreen}{rgb}{0,0.6,0}
\definecolor{codegray}{rgb}{0.5,0.5,0.5}
\definecolor{codepurple}{rgb}{0.58,0,0.82}
\definecolor{backcolour}{rgb}{0.95,0.95,0.92}

\lstdefinestyle{mystyle}{
    backgroundcolor=\color{backcolour},   
    commentstyle=\color{codegreen},
    keywordstyle=\color{magenta},
    numberstyle=\tiny\color{codegray},
    stringstyle=\color{codepurple},
    basicstyle=\ttfamily\footnotesize,
    breakatwhitespace=false,         
    breaklines=true,                 
    captionpos=b,                    
    keepspaces=true,                 
    numbers=left,                    
    numbersep=5pt,                  
    showspaces=false,                
    showstringspaces=false,
    showtabs=false,                  
    tabsize=2
}

\lstset{style=mystyle}

%% -----

% mostra le subsubsection nell'indice
\setcounter{tocdepth}{3}
\setcounter{secnumdepth}{3}

% Resetta la numerazione dei chapter quando
% una nuova part viene creata
\makeatletter
\@addtoreset{chapter}{part}
\makeatother

% Rimuove l'indentazione quando si crea un nuovo paragrafo
\setlength{\parindent}{0pt}

% footer
\pagestyle{fancyplain}
% rimuove la riga nell'header
\fancyhf{} % sets both header and footer to nothing
\renewcommand{\headrulewidth}{0pt}
\fancyfoot[L]{\href{https://github.com/Typing-Monkeys/AppuntiUniversita}{Typing Monkeys}}
\fancyfoot[C]{\emoji{gorilla}}
\fancyfoot[R]{\thepage}

% configurazione emoji
\usepackage{fontspec}
\usepackage{emoji}
\setemojifont{NotoColorEmoji.ttf}[Path=/usr/share/fonts/truetype/noto/]

\newtheorem{definition}{Definizione}
\newtheorem{lemma}{Lemma}
\newtheorem{theorem}{Teorema}
\newtheorem{corollary}{Corollario}

%% cambio nome al comando proof
\renewcommand*{\proofname}{Dimostrazione}

\begin{document}
\documentclass[a4paper,12 pt]{report}
\usepackage[T1]{fontenc}
\usepackage[utf8]{inputenc}
\usepackage{lmodern}
\usepackage{listings}

\usepackage{float}
\usepackage{subcaption}
\usepackage{wrapfig}
\usepackage{fancyhdr}
\usepackage{amsthm}

\usepackage{pgfplots}
\setlength {\marginparwidth }{2cm}
\usepackage{todonotes}
\newcommand{\TODO}[2][]
{\todo[size=\scriptsize, color=red, #1]{#2}}


asdf
\pgfplotsset{compat=1.18}
% forza le footnote a stare il più in basso possibile
\usepackage[bottom]{footmisc}


%% STILE LISTINGS
%%aaa
\usepackage{xcolor}

\definecolor{codegreen}{rgb}{0,0.6,0}
\definecolor{codegray}{rgb}{0.5,0.5,0.5}
\definecolor{codepurple}{rgb}{0.58,0,0.82}
\definecolor{backcolour}{rgb}{0.95,0.95,0.92}

\lstdefinestyle{mystyle}{
    backgroundcolor=\color{backcolour},   
    commentstyle=\color{codegreen},
    keywordstyle=\color{magenta},
    numberstyle=\tiny\color{codegray},
    stringstyle=\color{codepurple},
    basicstyle=\ttfamily\footnotesize,
    breakatwhitespace=false,         
    breaklines=true,                 
    captionpos=b,                    
    keepspaces=true,                 
    numbers=left,                    
    numbersep=5pt,                  
    showspaces=false,                
    showstringspaces=false,
    showtabs=false,                  
    tabsize=2
}

\lstset{style=mystyle}

%% -----

% mostra le subsubsection nell'indice
\setcounter{tocdepth}{3}
\setcounter{secnumdepth}{3}

% Resetta la numerazione dei chapter quando
% una nuova part viene creata
\makeatletter
\@addtoreset{chapter}{part}
\makeatother

% Rimuove l'indentazione quando si crea un nuovo paragrafo
\setlength{\parindent}{0pt}

% footer
\pagestyle{fancyplain}
% rimuove la riga nell'header
\fancyhf{} % sets both header and footer to nothing
\renewcommand{\headrulewidth}{0pt}
\fancyfoot[L]{\href{https://github.com/Typing-Monkeys/AppuntiUniversita}{Typing Monkeys}}
\fancyfoot[C]{\emoji{gorilla}}
\fancyfoot[R]{\thepage}

% configurazione emoji
\usepackage{fontspec}
\usepackage{emoji}
\setemojifont{NotoColorEmoji.ttf}[Path=/usr/share/fonts/truetype/noto/]

\newtheorem{definition}{Definizione}
\newtheorem{lemma}{Lemma}
\newtheorem{theorem}{Teorema}
\newtheorem{corollary}{Corollario}

%% cambio nome al comando proof
\renewcommand*{\proofname}{Dimostrazione}

\begin{document}
\include{frontmatter/main.tex}

\tableofcontents

\include{quote/main.tex}

%% Aggiungere i capitoli qui sotto
\include{capitoli/richiami/main.tex}
\include{capitoli/immagini/main.tex}

\end{document}


\tableofcontents

\documentclass[a4paper,12 pt]{report}
\usepackage[T1]{fontenc}
\usepackage[utf8]{inputenc}
\usepackage{lmodern}
\usepackage{listings}

\usepackage{float}
\usepackage{subcaption}
\usepackage{wrapfig}
\usepackage{fancyhdr}
\usepackage{amsthm}

\usepackage{pgfplots}
\setlength {\marginparwidth }{2cm}
\usepackage{todonotes}
\newcommand{\TODO}[2][]
{\todo[size=\scriptsize, color=red, #1]{#2}}


asdf
\pgfplotsset{compat=1.18}
% forza le footnote a stare il più in basso possibile
\usepackage[bottom]{footmisc}


%% STILE LISTINGS
%%aaa
\usepackage{xcolor}

\definecolor{codegreen}{rgb}{0,0.6,0}
\definecolor{codegray}{rgb}{0.5,0.5,0.5}
\definecolor{codepurple}{rgb}{0.58,0,0.82}
\definecolor{backcolour}{rgb}{0.95,0.95,0.92}

\lstdefinestyle{mystyle}{
    backgroundcolor=\color{backcolour},   
    commentstyle=\color{codegreen},
    keywordstyle=\color{magenta},
    numberstyle=\tiny\color{codegray},
    stringstyle=\color{codepurple},
    basicstyle=\ttfamily\footnotesize,
    breakatwhitespace=false,         
    breaklines=true,                 
    captionpos=b,                    
    keepspaces=true,                 
    numbers=left,                    
    numbersep=5pt,                  
    showspaces=false,                
    showstringspaces=false,
    showtabs=false,                  
    tabsize=2
}

\lstset{style=mystyle}

%% -----

% mostra le subsubsection nell'indice
\setcounter{tocdepth}{3}
\setcounter{secnumdepth}{3}

% Resetta la numerazione dei chapter quando
% una nuova part viene creata
\makeatletter
\@addtoreset{chapter}{part}
\makeatother

% Rimuove l'indentazione quando si crea un nuovo paragrafo
\setlength{\parindent}{0pt}

% footer
\pagestyle{fancyplain}
% rimuove la riga nell'header
\fancyhf{} % sets both header and footer to nothing
\renewcommand{\headrulewidth}{0pt}
\fancyfoot[L]{\href{https://github.com/Typing-Monkeys/AppuntiUniversita}{Typing Monkeys}}
\fancyfoot[C]{\emoji{gorilla}}
\fancyfoot[R]{\thepage}

% configurazione emoji
\usepackage{fontspec}
\usepackage{emoji}
\setemojifont{NotoColorEmoji.ttf}[Path=/usr/share/fonts/truetype/noto/]

\newtheorem{definition}{Definizione}
\newtheorem{lemma}{Lemma}
\newtheorem{theorem}{Teorema}
\newtheorem{corollary}{Corollario}

%% cambio nome al comando proof
\renewcommand*{\proofname}{Dimostrazione}

\begin{document}
\include{frontmatter/main.tex}

\tableofcontents

\include{quote/main.tex}

%% Aggiungere i capitoli qui sotto
\include{capitoli/richiami/main.tex}
\include{capitoli/immagini/main.tex}

\end{document}


%% Aggiungere i capitoli qui sotto
\documentclass[a4paper,12 pt]{report}
\usepackage[T1]{fontenc}
\usepackage[utf8]{inputenc}
\usepackage{lmodern}
\usepackage{listings}

\usepackage{float}
\usepackage{subcaption}
\usepackage{wrapfig}
\usepackage{fancyhdr}
\usepackage{amsthm}

\usepackage{pgfplots}
\setlength {\marginparwidth }{2cm}
\usepackage{todonotes}
\newcommand{\TODO}[2][]
{\todo[size=\scriptsize, color=red, #1]{#2}}


asdf
\pgfplotsset{compat=1.18}
% forza le footnote a stare il più in basso possibile
\usepackage[bottom]{footmisc}


%% STILE LISTINGS
%%aaa
\usepackage{xcolor}

\definecolor{codegreen}{rgb}{0,0.6,0}
\definecolor{codegray}{rgb}{0.5,0.5,0.5}
\definecolor{codepurple}{rgb}{0.58,0,0.82}
\definecolor{backcolour}{rgb}{0.95,0.95,0.92}

\lstdefinestyle{mystyle}{
    backgroundcolor=\color{backcolour},   
    commentstyle=\color{codegreen},
    keywordstyle=\color{magenta},
    numberstyle=\tiny\color{codegray},
    stringstyle=\color{codepurple},
    basicstyle=\ttfamily\footnotesize,
    breakatwhitespace=false,         
    breaklines=true,                 
    captionpos=b,                    
    keepspaces=true,                 
    numbers=left,                    
    numbersep=5pt,                  
    showspaces=false,                
    showstringspaces=false,
    showtabs=false,                  
    tabsize=2
}

\lstset{style=mystyle}

%% -----

% mostra le subsubsection nell'indice
\setcounter{tocdepth}{3}
\setcounter{secnumdepth}{3}

% Resetta la numerazione dei chapter quando
% una nuova part viene creata
\makeatletter
\@addtoreset{chapter}{part}
\makeatother

% Rimuove l'indentazione quando si crea un nuovo paragrafo
\setlength{\parindent}{0pt}

% footer
\pagestyle{fancyplain}
% rimuove la riga nell'header
\fancyhf{} % sets both header and footer to nothing
\renewcommand{\headrulewidth}{0pt}
\fancyfoot[L]{\href{https://github.com/Typing-Monkeys/AppuntiUniversita}{Typing Monkeys}}
\fancyfoot[C]{\emoji{gorilla}}
\fancyfoot[R]{\thepage}

% configurazione emoji
\usepackage{fontspec}
\usepackage{emoji}
\setemojifont{NotoColorEmoji.ttf}[Path=/usr/share/fonts/truetype/noto/]

\newtheorem{definition}{Definizione}
\newtheorem{lemma}{Lemma}
\newtheorem{theorem}{Teorema}
\newtheorem{corollary}{Corollario}

%% cambio nome al comando proof
\renewcommand*{\proofname}{Dimostrazione}

\begin{document}
\include{frontmatter/main.tex}

\tableofcontents

\include{quote/main.tex}

%% Aggiungere i capitoli qui sotto
\include{capitoli/richiami/main.tex}
\include{capitoli/immagini/main.tex}

\end{document}

\documentclass[a4paper,12 pt]{report}
\usepackage[T1]{fontenc}
\usepackage[utf8]{inputenc}
\usepackage{lmodern}
\usepackage{listings}

\usepackage{float}
\usepackage{subcaption}
\usepackage{wrapfig}
\usepackage{fancyhdr}
\usepackage{amsthm}

\usepackage{pgfplots}
\setlength {\marginparwidth }{2cm}
\usepackage{todonotes}
\newcommand{\TODO}[2][]
{\todo[size=\scriptsize, color=red, #1]{#2}}


asdf
\pgfplotsset{compat=1.18}
% forza le footnote a stare il più in basso possibile
\usepackage[bottom]{footmisc}


%% STILE LISTINGS
%%aaa
\usepackage{xcolor}

\definecolor{codegreen}{rgb}{0,0.6,0}
\definecolor{codegray}{rgb}{0.5,0.5,0.5}
\definecolor{codepurple}{rgb}{0.58,0,0.82}
\definecolor{backcolour}{rgb}{0.95,0.95,0.92}

\lstdefinestyle{mystyle}{
    backgroundcolor=\color{backcolour},   
    commentstyle=\color{codegreen},
    keywordstyle=\color{magenta},
    numberstyle=\tiny\color{codegray},
    stringstyle=\color{codepurple},
    basicstyle=\ttfamily\footnotesize,
    breakatwhitespace=false,         
    breaklines=true,                 
    captionpos=b,                    
    keepspaces=true,                 
    numbers=left,                    
    numbersep=5pt,                  
    showspaces=false,                
    showstringspaces=false,
    showtabs=false,                  
    tabsize=2
}

\lstset{style=mystyle}

%% -----

% mostra le subsubsection nell'indice
\setcounter{tocdepth}{3}
\setcounter{secnumdepth}{3}

% Resetta la numerazione dei chapter quando
% una nuova part viene creata
\makeatletter
\@addtoreset{chapter}{part}
\makeatother

% Rimuove l'indentazione quando si crea un nuovo paragrafo
\setlength{\parindent}{0pt}

% footer
\pagestyle{fancyplain}
% rimuove la riga nell'header
\fancyhf{} % sets both header and footer to nothing
\renewcommand{\headrulewidth}{0pt}
\fancyfoot[L]{\href{https://github.com/Typing-Monkeys/AppuntiUniversita}{Typing Monkeys}}
\fancyfoot[C]{\emoji{gorilla}}
\fancyfoot[R]{\thepage}

% configurazione emoji
\usepackage{fontspec}
\usepackage{emoji}
\setemojifont{NotoColorEmoji.ttf}[Path=/usr/share/fonts/truetype/noto/]

\newtheorem{definition}{Definizione}
\newtheorem{lemma}{Lemma}
\newtheorem{theorem}{Teorema}
\newtheorem{corollary}{Corollario}

%% cambio nome al comando proof
\renewcommand*{\proofname}{Dimostrazione}

\begin{document}
\include{frontmatter/main.tex}

\tableofcontents

\include{quote/main.tex}

%% Aggiungere i capitoli qui sotto
\include{capitoli/richiami/main.tex}
\include{capitoli/immagini/main.tex}

\end{document}


\end{document}


%% Aggiungere i capitoli qui sotto
\documentclass[a4paper,12 pt]{report}
\usepackage[T1]{fontenc}
\usepackage[utf8]{inputenc}
\usepackage{lmodern}
\usepackage{listings}

\usepackage{float}
\usepackage{subcaption}
\usepackage{wrapfig}
\usepackage{fancyhdr}
\usepackage{amsthm}

\usepackage{pgfplots}
\setlength {\marginparwidth }{2cm}
\usepackage{todonotes}
\newcommand{\TODO}[2][]
{\todo[size=\scriptsize, color=red, #1]{#2}}


asdf
\pgfplotsset{compat=1.18}
% forza le footnote a stare il più in basso possibile
\usepackage[bottom]{footmisc}


%% STILE LISTINGS
%%aaa
\usepackage{xcolor}

\definecolor{codegreen}{rgb}{0,0.6,0}
\definecolor{codegray}{rgb}{0.5,0.5,0.5}
\definecolor{codepurple}{rgb}{0.58,0,0.82}
\definecolor{backcolour}{rgb}{0.95,0.95,0.92}

\lstdefinestyle{mystyle}{
    backgroundcolor=\color{backcolour},   
    commentstyle=\color{codegreen},
    keywordstyle=\color{magenta},
    numberstyle=\tiny\color{codegray},
    stringstyle=\color{codepurple},
    basicstyle=\ttfamily\footnotesize,
    breakatwhitespace=false,         
    breaklines=true,                 
    captionpos=b,                    
    keepspaces=true,                 
    numbers=left,                    
    numbersep=5pt,                  
    showspaces=false,                
    showstringspaces=false,
    showtabs=false,                  
    tabsize=2
}

\lstset{style=mystyle}

%% -----

% mostra le subsubsection nell'indice
\setcounter{tocdepth}{3}
\setcounter{secnumdepth}{3}

% Resetta la numerazione dei chapter quando
% una nuova part viene creata
\makeatletter
\@addtoreset{chapter}{part}
\makeatother

% Rimuove l'indentazione quando si crea un nuovo paragrafo
\setlength{\parindent}{0pt}

% footer
\pagestyle{fancyplain}
% rimuove la riga nell'header
\fancyhf{} % sets both header and footer to nothing
\renewcommand{\headrulewidth}{0pt}
\fancyfoot[L]{\href{https://github.com/Typing-Monkeys/AppuntiUniversita}{Typing Monkeys}}
\fancyfoot[C]{\emoji{gorilla}}
\fancyfoot[R]{\thepage}

% configurazione emoji
\usepackage{fontspec}
\usepackage{emoji}
\setemojifont{NotoColorEmoji.ttf}[Path=/usr/share/fonts/truetype/noto/]

\newtheorem{definition}{Definizione}
\newtheorem{lemma}{Lemma}
\newtheorem{theorem}{Teorema}
\newtheorem{corollary}{Corollario}

%% cambio nome al comando proof
\renewcommand*{\proofname}{Dimostrazione}

\begin{document}
\documentclass[a4paper,12 pt]{report}
\usepackage[T1]{fontenc}
\usepackage[utf8]{inputenc}
\usepackage{lmodern}
\usepackage{listings}

\usepackage{float}
\usepackage{subcaption}
\usepackage{wrapfig}
\usepackage{fancyhdr}
\usepackage{amsthm}

\usepackage{pgfplots}
\setlength {\marginparwidth }{2cm}
\usepackage{todonotes}
\newcommand{\TODO}[2][]
{\todo[size=\scriptsize, color=red, #1]{#2}}


asdf
\pgfplotsset{compat=1.18}
% forza le footnote a stare il più in basso possibile
\usepackage[bottom]{footmisc}


%% STILE LISTINGS
%%aaa
\usepackage{xcolor}

\definecolor{codegreen}{rgb}{0,0.6,0}
\definecolor{codegray}{rgb}{0.5,0.5,0.5}
\definecolor{codepurple}{rgb}{0.58,0,0.82}
\definecolor{backcolour}{rgb}{0.95,0.95,0.92}

\lstdefinestyle{mystyle}{
    backgroundcolor=\color{backcolour},   
    commentstyle=\color{codegreen},
    keywordstyle=\color{magenta},
    numberstyle=\tiny\color{codegray},
    stringstyle=\color{codepurple},
    basicstyle=\ttfamily\footnotesize,
    breakatwhitespace=false,         
    breaklines=true,                 
    captionpos=b,                    
    keepspaces=true,                 
    numbers=left,                    
    numbersep=5pt,                  
    showspaces=false,                
    showstringspaces=false,
    showtabs=false,                  
    tabsize=2
}

\lstset{style=mystyle}

%% -----

% mostra le subsubsection nell'indice
\setcounter{tocdepth}{3}
\setcounter{secnumdepth}{3}

% Resetta la numerazione dei chapter quando
% una nuova part viene creata
\makeatletter
\@addtoreset{chapter}{part}
\makeatother

% Rimuove l'indentazione quando si crea un nuovo paragrafo
\setlength{\parindent}{0pt}

% footer
\pagestyle{fancyplain}
% rimuove la riga nell'header
\fancyhf{} % sets both header and footer to nothing
\renewcommand{\headrulewidth}{0pt}
\fancyfoot[L]{\href{https://github.com/Typing-Monkeys/AppuntiUniversita}{Typing Monkeys}}
\fancyfoot[C]{\emoji{gorilla}}
\fancyfoot[R]{\thepage}

% configurazione emoji
\usepackage{fontspec}
\usepackage{emoji}
\setemojifont{NotoColorEmoji.ttf}[Path=/usr/share/fonts/truetype/noto/]

\newtheorem{definition}{Definizione}
\newtheorem{lemma}{Lemma}
\newtheorem{theorem}{Teorema}
\newtheorem{corollary}{Corollario}

%% cambio nome al comando proof
\renewcommand*{\proofname}{Dimostrazione}

\begin{document}
\include{frontmatter/main.tex}

\tableofcontents

\include{quote/main.tex}

%% Aggiungere i capitoli qui sotto
\include{capitoli/richiami/main.tex}
\include{capitoli/immagini/main.tex}

\end{document}


\tableofcontents

\documentclass[a4paper,12 pt]{report}
\usepackage[T1]{fontenc}
\usepackage[utf8]{inputenc}
\usepackage{lmodern}
\usepackage{listings}

\usepackage{float}
\usepackage{subcaption}
\usepackage{wrapfig}
\usepackage{fancyhdr}
\usepackage{amsthm}

\usepackage{pgfplots}
\setlength {\marginparwidth }{2cm}
\usepackage{todonotes}
\newcommand{\TODO}[2][]
{\todo[size=\scriptsize, color=red, #1]{#2}}


asdf
\pgfplotsset{compat=1.18}
% forza le footnote a stare il più in basso possibile
\usepackage[bottom]{footmisc}


%% STILE LISTINGS
%%aaa
\usepackage{xcolor}

\definecolor{codegreen}{rgb}{0,0.6,0}
\definecolor{codegray}{rgb}{0.5,0.5,0.5}
\definecolor{codepurple}{rgb}{0.58,0,0.82}
\definecolor{backcolour}{rgb}{0.95,0.95,0.92}

\lstdefinestyle{mystyle}{
    backgroundcolor=\color{backcolour},   
    commentstyle=\color{codegreen},
    keywordstyle=\color{magenta},
    numberstyle=\tiny\color{codegray},
    stringstyle=\color{codepurple},
    basicstyle=\ttfamily\footnotesize,
    breakatwhitespace=false,         
    breaklines=true,                 
    captionpos=b,                    
    keepspaces=true,                 
    numbers=left,                    
    numbersep=5pt,                  
    showspaces=false,                
    showstringspaces=false,
    showtabs=false,                  
    tabsize=2
}

\lstset{style=mystyle}

%% -----

% mostra le subsubsection nell'indice
\setcounter{tocdepth}{3}
\setcounter{secnumdepth}{3}

% Resetta la numerazione dei chapter quando
% una nuova part viene creata
\makeatletter
\@addtoreset{chapter}{part}
\makeatother

% Rimuove l'indentazione quando si crea un nuovo paragrafo
\setlength{\parindent}{0pt}

% footer
\pagestyle{fancyplain}
% rimuove la riga nell'header
\fancyhf{} % sets both header and footer to nothing
\renewcommand{\headrulewidth}{0pt}
\fancyfoot[L]{\href{https://github.com/Typing-Monkeys/AppuntiUniversita}{Typing Monkeys}}
\fancyfoot[C]{\emoji{gorilla}}
\fancyfoot[R]{\thepage}

% configurazione emoji
\usepackage{fontspec}
\usepackage{emoji}
\setemojifont{NotoColorEmoji.ttf}[Path=/usr/share/fonts/truetype/noto/]

\newtheorem{definition}{Definizione}
\newtheorem{lemma}{Lemma}
\newtheorem{theorem}{Teorema}
\newtheorem{corollary}{Corollario}

%% cambio nome al comando proof
\renewcommand*{\proofname}{Dimostrazione}

\begin{document}
\include{frontmatter/main.tex}

\tableofcontents

\include{quote/main.tex}

%% Aggiungere i capitoli qui sotto
\include{capitoli/richiami/main.tex}
\include{capitoli/immagini/main.tex}

\end{document}


%% Aggiungere i capitoli qui sotto
\documentclass[a4paper,12 pt]{report}
\usepackage[T1]{fontenc}
\usepackage[utf8]{inputenc}
\usepackage{lmodern}
\usepackage{listings}

\usepackage{float}
\usepackage{subcaption}
\usepackage{wrapfig}
\usepackage{fancyhdr}
\usepackage{amsthm}

\usepackage{pgfplots}
\setlength {\marginparwidth }{2cm}
\usepackage{todonotes}
\newcommand{\TODO}[2][]
{\todo[size=\scriptsize, color=red, #1]{#2}}


asdf
\pgfplotsset{compat=1.18}
% forza le footnote a stare il più in basso possibile
\usepackage[bottom]{footmisc}


%% STILE LISTINGS
%%aaa
\usepackage{xcolor}

\definecolor{codegreen}{rgb}{0,0.6,0}
\definecolor{codegray}{rgb}{0.5,0.5,0.5}
\definecolor{codepurple}{rgb}{0.58,0,0.82}
\definecolor{backcolour}{rgb}{0.95,0.95,0.92}

\lstdefinestyle{mystyle}{
    backgroundcolor=\color{backcolour},   
    commentstyle=\color{codegreen},
    keywordstyle=\color{magenta},
    numberstyle=\tiny\color{codegray},
    stringstyle=\color{codepurple},
    basicstyle=\ttfamily\footnotesize,
    breakatwhitespace=false,         
    breaklines=true,                 
    captionpos=b,                    
    keepspaces=true,                 
    numbers=left,                    
    numbersep=5pt,                  
    showspaces=false,                
    showstringspaces=false,
    showtabs=false,                  
    tabsize=2
}

\lstset{style=mystyle}

%% -----

% mostra le subsubsection nell'indice
\setcounter{tocdepth}{3}
\setcounter{secnumdepth}{3}

% Resetta la numerazione dei chapter quando
% una nuova part viene creata
\makeatletter
\@addtoreset{chapter}{part}
\makeatother

% Rimuove l'indentazione quando si crea un nuovo paragrafo
\setlength{\parindent}{0pt}

% footer
\pagestyle{fancyplain}
% rimuove la riga nell'header
\fancyhf{} % sets both header and footer to nothing
\renewcommand{\headrulewidth}{0pt}
\fancyfoot[L]{\href{https://github.com/Typing-Monkeys/AppuntiUniversita}{Typing Monkeys}}
\fancyfoot[C]{\emoji{gorilla}}
\fancyfoot[R]{\thepage}

% configurazione emoji
\usepackage{fontspec}
\usepackage{emoji}
\setemojifont{NotoColorEmoji.ttf}[Path=/usr/share/fonts/truetype/noto/]

\newtheorem{definition}{Definizione}
\newtheorem{lemma}{Lemma}
\newtheorem{theorem}{Teorema}
\newtheorem{corollary}{Corollario}

%% cambio nome al comando proof
\renewcommand*{\proofname}{Dimostrazione}

\begin{document}
\include{frontmatter/main.tex}

\tableofcontents

\include{quote/main.tex}

%% Aggiungere i capitoli qui sotto
\include{capitoli/richiami/main.tex}
\include{capitoli/immagini/main.tex}

\end{document}

\documentclass[a4paper,12 pt]{report}
\usepackage[T1]{fontenc}
\usepackage[utf8]{inputenc}
\usepackage{lmodern}
\usepackage{listings}

\usepackage{float}
\usepackage{subcaption}
\usepackage{wrapfig}
\usepackage{fancyhdr}
\usepackage{amsthm}

\usepackage{pgfplots}
\setlength {\marginparwidth }{2cm}
\usepackage{todonotes}
\newcommand{\TODO}[2][]
{\todo[size=\scriptsize, color=red, #1]{#2}}


asdf
\pgfplotsset{compat=1.18}
% forza le footnote a stare il più in basso possibile
\usepackage[bottom]{footmisc}


%% STILE LISTINGS
%%aaa
\usepackage{xcolor}

\definecolor{codegreen}{rgb}{0,0.6,0}
\definecolor{codegray}{rgb}{0.5,0.5,0.5}
\definecolor{codepurple}{rgb}{0.58,0,0.82}
\definecolor{backcolour}{rgb}{0.95,0.95,0.92}

\lstdefinestyle{mystyle}{
    backgroundcolor=\color{backcolour},   
    commentstyle=\color{codegreen},
    keywordstyle=\color{magenta},
    numberstyle=\tiny\color{codegray},
    stringstyle=\color{codepurple},
    basicstyle=\ttfamily\footnotesize,
    breakatwhitespace=false,         
    breaklines=true,                 
    captionpos=b,                    
    keepspaces=true,                 
    numbers=left,                    
    numbersep=5pt,                  
    showspaces=false,                
    showstringspaces=false,
    showtabs=false,                  
    tabsize=2
}

\lstset{style=mystyle}

%% -----

% mostra le subsubsection nell'indice
\setcounter{tocdepth}{3}
\setcounter{secnumdepth}{3}

% Resetta la numerazione dei chapter quando
% una nuova part viene creata
\makeatletter
\@addtoreset{chapter}{part}
\makeatother

% Rimuove l'indentazione quando si crea un nuovo paragrafo
\setlength{\parindent}{0pt}

% footer
\pagestyle{fancyplain}
% rimuove la riga nell'header
\fancyhf{} % sets both header and footer to nothing
\renewcommand{\headrulewidth}{0pt}
\fancyfoot[L]{\href{https://github.com/Typing-Monkeys/AppuntiUniversita}{Typing Monkeys}}
\fancyfoot[C]{\emoji{gorilla}}
\fancyfoot[R]{\thepage}

% configurazione emoji
\usepackage{fontspec}
\usepackage{emoji}
\setemojifont{NotoColorEmoji.ttf}[Path=/usr/share/fonts/truetype/noto/]

\newtheorem{definition}{Definizione}
\newtheorem{lemma}{Lemma}
\newtheorem{theorem}{Teorema}
\newtheorem{corollary}{Corollario}

%% cambio nome al comando proof
\renewcommand*{\proofname}{Dimostrazione}

\begin{document}
\include{frontmatter/main.tex}

\tableofcontents

\include{quote/main.tex}

%% Aggiungere i capitoli qui sotto
\include{capitoli/richiami/main.tex}
\include{capitoli/immagini/main.tex}

\end{document}


\end{document}

\documentclass[a4paper,12 pt]{report}
\usepackage[T1]{fontenc}
\usepackage[utf8]{inputenc}
\usepackage{lmodern}
\usepackage{listings}

\usepackage{float}
\usepackage{subcaption}
\usepackage{wrapfig}
\usepackage{fancyhdr}
\usepackage{amsthm}

\usepackage{pgfplots}
\setlength {\marginparwidth }{2cm}
\usepackage{todonotes}
\newcommand{\TODO}[2][]
{\todo[size=\scriptsize, color=red, #1]{#2}}


asdf
\pgfplotsset{compat=1.18}
% forza le footnote a stare il più in basso possibile
\usepackage[bottom]{footmisc}


%% STILE LISTINGS
%%aaa
\usepackage{xcolor}

\definecolor{codegreen}{rgb}{0,0.6,0}
\definecolor{codegray}{rgb}{0.5,0.5,0.5}
\definecolor{codepurple}{rgb}{0.58,0,0.82}
\definecolor{backcolour}{rgb}{0.95,0.95,0.92}

\lstdefinestyle{mystyle}{
    backgroundcolor=\color{backcolour},   
    commentstyle=\color{codegreen},
    keywordstyle=\color{magenta},
    numberstyle=\tiny\color{codegray},
    stringstyle=\color{codepurple},
    basicstyle=\ttfamily\footnotesize,
    breakatwhitespace=false,         
    breaklines=true,                 
    captionpos=b,                    
    keepspaces=true,                 
    numbers=left,                    
    numbersep=5pt,                  
    showspaces=false,                
    showstringspaces=false,
    showtabs=false,                  
    tabsize=2
}

\lstset{style=mystyle}

%% -----

% mostra le subsubsection nell'indice
\setcounter{tocdepth}{3}
\setcounter{secnumdepth}{3}

% Resetta la numerazione dei chapter quando
% una nuova part viene creata
\makeatletter
\@addtoreset{chapter}{part}
\makeatother

% Rimuove l'indentazione quando si crea un nuovo paragrafo
\setlength{\parindent}{0pt}

% footer
\pagestyle{fancyplain}
% rimuove la riga nell'header
\fancyhf{} % sets both header and footer to nothing
\renewcommand{\headrulewidth}{0pt}
\fancyfoot[L]{\href{https://github.com/Typing-Monkeys/AppuntiUniversita}{Typing Monkeys}}
\fancyfoot[C]{\emoji{gorilla}}
\fancyfoot[R]{\thepage}

% configurazione emoji
\usepackage{fontspec}
\usepackage{emoji}
\setemojifont{NotoColorEmoji.ttf}[Path=/usr/share/fonts/truetype/noto/]

\newtheorem{definition}{Definizione}
\newtheorem{lemma}{Lemma}
\newtheorem{theorem}{Teorema}
\newtheorem{corollary}{Corollario}

%% cambio nome al comando proof
\renewcommand*{\proofname}{Dimostrazione}

\begin{document}
\documentclass[a4paper,12 pt]{report}
\usepackage[T1]{fontenc}
\usepackage[utf8]{inputenc}
\usepackage{lmodern}
\usepackage{listings}

\usepackage{float}
\usepackage{subcaption}
\usepackage{wrapfig}
\usepackage{fancyhdr}
\usepackage{amsthm}

\usepackage{pgfplots}
\setlength {\marginparwidth }{2cm}
\usepackage{todonotes}
\newcommand{\TODO}[2][]
{\todo[size=\scriptsize, color=red, #1]{#2}}


asdf
\pgfplotsset{compat=1.18}
% forza le footnote a stare il più in basso possibile
\usepackage[bottom]{footmisc}


%% STILE LISTINGS
%%aaa
\usepackage{xcolor}

\definecolor{codegreen}{rgb}{0,0.6,0}
\definecolor{codegray}{rgb}{0.5,0.5,0.5}
\definecolor{codepurple}{rgb}{0.58,0,0.82}
\definecolor{backcolour}{rgb}{0.95,0.95,0.92}

\lstdefinestyle{mystyle}{
    backgroundcolor=\color{backcolour},   
    commentstyle=\color{codegreen},
    keywordstyle=\color{magenta},
    numberstyle=\tiny\color{codegray},
    stringstyle=\color{codepurple},
    basicstyle=\ttfamily\footnotesize,
    breakatwhitespace=false,         
    breaklines=true,                 
    captionpos=b,                    
    keepspaces=true,                 
    numbers=left,                    
    numbersep=5pt,                  
    showspaces=false,                
    showstringspaces=false,
    showtabs=false,                  
    tabsize=2
}

\lstset{style=mystyle}

%% -----

% mostra le subsubsection nell'indice
\setcounter{tocdepth}{3}
\setcounter{secnumdepth}{3}

% Resetta la numerazione dei chapter quando
% una nuova part viene creata
\makeatletter
\@addtoreset{chapter}{part}
\makeatother

% Rimuove l'indentazione quando si crea un nuovo paragrafo
\setlength{\parindent}{0pt}

% footer
\pagestyle{fancyplain}
% rimuove la riga nell'header
\fancyhf{} % sets both header and footer to nothing
\renewcommand{\headrulewidth}{0pt}
\fancyfoot[L]{\href{https://github.com/Typing-Monkeys/AppuntiUniversita}{Typing Monkeys}}
\fancyfoot[C]{\emoji{gorilla}}
\fancyfoot[R]{\thepage}

% configurazione emoji
\usepackage{fontspec}
\usepackage{emoji}
\setemojifont{NotoColorEmoji.ttf}[Path=/usr/share/fonts/truetype/noto/]

\newtheorem{definition}{Definizione}
\newtheorem{lemma}{Lemma}
\newtheorem{theorem}{Teorema}
\newtheorem{corollary}{Corollario}

%% cambio nome al comando proof
\renewcommand*{\proofname}{Dimostrazione}

\begin{document}
\include{frontmatter/main.tex}

\tableofcontents

\include{quote/main.tex}

%% Aggiungere i capitoli qui sotto
\include{capitoli/richiami/main.tex}
\include{capitoli/immagini/main.tex}

\end{document}


\tableofcontents

\documentclass[a4paper,12 pt]{report}
\usepackage[T1]{fontenc}
\usepackage[utf8]{inputenc}
\usepackage{lmodern}
\usepackage{listings}

\usepackage{float}
\usepackage{subcaption}
\usepackage{wrapfig}
\usepackage{fancyhdr}
\usepackage{amsthm}

\usepackage{pgfplots}
\setlength {\marginparwidth }{2cm}
\usepackage{todonotes}
\newcommand{\TODO}[2][]
{\todo[size=\scriptsize, color=red, #1]{#2}}


asdf
\pgfplotsset{compat=1.18}
% forza le footnote a stare il più in basso possibile
\usepackage[bottom]{footmisc}


%% STILE LISTINGS
%%aaa
\usepackage{xcolor}

\definecolor{codegreen}{rgb}{0,0.6,0}
\definecolor{codegray}{rgb}{0.5,0.5,0.5}
\definecolor{codepurple}{rgb}{0.58,0,0.82}
\definecolor{backcolour}{rgb}{0.95,0.95,0.92}

\lstdefinestyle{mystyle}{
    backgroundcolor=\color{backcolour},   
    commentstyle=\color{codegreen},
    keywordstyle=\color{magenta},
    numberstyle=\tiny\color{codegray},
    stringstyle=\color{codepurple},
    basicstyle=\ttfamily\footnotesize,
    breakatwhitespace=false,         
    breaklines=true,                 
    captionpos=b,                    
    keepspaces=true,                 
    numbers=left,                    
    numbersep=5pt,                  
    showspaces=false,                
    showstringspaces=false,
    showtabs=false,                  
    tabsize=2
}

\lstset{style=mystyle}

%% -----

% mostra le subsubsection nell'indice
\setcounter{tocdepth}{3}
\setcounter{secnumdepth}{3}

% Resetta la numerazione dei chapter quando
% una nuova part viene creata
\makeatletter
\@addtoreset{chapter}{part}
\makeatother

% Rimuove l'indentazione quando si crea un nuovo paragrafo
\setlength{\parindent}{0pt}

% footer
\pagestyle{fancyplain}
% rimuove la riga nell'header
\fancyhf{} % sets both header and footer to nothing
\renewcommand{\headrulewidth}{0pt}
\fancyfoot[L]{\href{https://github.com/Typing-Monkeys/AppuntiUniversita}{Typing Monkeys}}
\fancyfoot[C]{\emoji{gorilla}}
\fancyfoot[R]{\thepage}

% configurazione emoji
\usepackage{fontspec}
\usepackage{emoji}
\setemojifont{NotoColorEmoji.ttf}[Path=/usr/share/fonts/truetype/noto/]

\newtheorem{definition}{Definizione}
\newtheorem{lemma}{Lemma}
\newtheorem{theorem}{Teorema}
\newtheorem{corollary}{Corollario}

%% cambio nome al comando proof
\renewcommand*{\proofname}{Dimostrazione}

\begin{document}
\include{frontmatter/main.tex}

\tableofcontents

\include{quote/main.tex}

%% Aggiungere i capitoli qui sotto
\include{capitoli/richiami/main.tex}
\include{capitoli/immagini/main.tex}

\end{document}


%% Aggiungere i capitoli qui sotto
\documentclass[a4paper,12 pt]{report}
\usepackage[T1]{fontenc}
\usepackage[utf8]{inputenc}
\usepackage{lmodern}
\usepackage{listings}

\usepackage{float}
\usepackage{subcaption}
\usepackage{wrapfig}
\usepackage{fancyhdr}
\usepackage{amsthm}

\usepackage{pgfplots}
\setlength {\marginparwidth }{2cm}
\usepackage{todonotes}
\newcommand{\TODO}[2][]
{\todo[size=\scriptsize, color=red, #1]{#2}}


asdf
\pgfplotsset{compat=1.18}
% forza le footnote a stare il più in basso possibile
\usepackage[bottom]{footmisc}


%% STILE LISTINGS
%%aaa
\usepackage{xcolor}

\definecolor{codegreen}{rgb}{0,0.6,0}
\definecolor{codegray}{rgb}{0.5,0.5,0.5}
\definecolor{codepurple}{rgb}{0.58,0,0.82}
\definecolor{backcolour}{rgb}{0.95,0.95,0.92}

\lstdefinestyle{mystyle}{
    backgroundcolor=\color{backcolour},   
    commentstyle=\color{codegreen},
    keywordstyle=\color{magenta},
    numberstyle=\tiny\color{codegray},
    stringstyle=\color{codepurple},
    basicstyle=\ttfamily\footnotesize,
    breakatwhitespace=false,         
    breaklines=true,                 
    captionpos=b,                    
    keepspaces=true,                 
    numbers=left,                    
    numbersep=5pt,                  
    showspaces=false,                
    showstringspaces=false,
    showtabs=false,                  
    tabsize=2
}

\lstset{style=mystyle}

%% -----

% mostra le subsubsection nell'indice
\setcounter{tocdepth}{3}
\setcounter{secnumdepth}{3}

% Resetta la numerazione dei chapter quando
% una nuova part viene creata
\makeatletter
\@addtoreset{chapter}{part}
\makeatother

% Rimuove l'indentazione quando si crea un nuovo paragrafo
\setlength{\parindent}{0pt}

% footer
\pagestyle{fancyplain}
% rimuove la riga nell'header
\fancyhf{} % sets both header and footer to nothing
\renewcommand{\headrulewidth}{0pt}
\fancyfoot[L]{\href{https://github.com/Typing-Monkeys/AppuntiUniversita}{Typing Monkeys}}
\fancyfoot[C]{\emoji{gorilla}}
\fancyfoot[R]{\thepage}

% configurazione emoji
\usepackage{fontspec}
\usepackage{emoji}
\setemojifont{NotoColorEmoji.ttf}[Path=/usr/share/fonts/truetype/noto/]

\newtheorem{definition}{Definizione}
\newtheorem{lemma}{Lemma}
\newtheorem{theorem}{Teorema}
\newtheorem{corollary}{Corollario}

%% cambio nome al comando proof
\renewcommand*{\proofname}{Dimostrazione}

\begin{document}
\include{frontmatter/main.tex}

\tableofcontents

\include{quote/main.tex}

%% Aggiungere i capitoli qui sotto
\include{capitoli/richiami/main.tex}
\include{capitoli/immagini/main.tex}

\end{document}

\documentclass[a4paper,12 pt]{report}
\usepackage[T1]{fontenc}
\usepackage[utf8]{inputenc}
\usepackage{lmodern}
\usepackage{listings}

\usepackage{float}
\usepackage{subcaption}
\usepackage{wrapfig}
\usepackage{fancyhdr}
\usepackage{amsthm}

\usepackage{pgfplots}
\setlength {\marginparwidth }{2cm}
\usepackage{todonotes}
\newcommand{\TODO}[2][]
{\todo[size=\scriptsize, color=red, #1]{#2}}


asdf
\pgfplotsset{compat=1.18}
% forza le footnote a stare il più in basso possibile
\usepackage[bottom]{footmisc}


%% STILE LISTINGS
%%aaa
\usepackage{xcolor}

\definecolor{codegreen}{rgb}{0,0.6,0}
\definecolor{codegray}{rgb}{0.5,0.5,0.5}
\definecolor{codepurple}{rgb}{0.58,0,0.82}
\definecolor{backcolour}{rgb}{0.95,0.95,0.92}

\lstdefinestyle{mystyle}{
    backgroundcolor=\color{backcolour},   
    commentstyle=\color{codegreen},
    keywordstyle=\color{magenta},
    numberstyle=\tiny\color{codegray},
    stringstyle=\color{codepurple},
    basicstyle=\ttfamily\footnotesize,
    breakatwhitespace=false,         
    breaklines=true,                 
    captionpos=b,                    
    keepspaces=true,                 
    numbers=left,                    
    numbersep=5pt,                  
    showspaces=false,                
    showstringspaces=false,
    showtabs=false,                  
    tabsize=2
}

\lstset{style=mystyle}

%% -----

% mostra le subsubsection nell'indice
\setcounter{tocdepth}{3}
\setcounter{secnumdepth}{3}

% Resetta la numerazione dei chapter quando
% una nuova part viene creata
\makeatletter
\@addtoreset{chapter}{part}
\makeatother

% Rimuove l'indentazione quando si crea un nuovo paragrafo
\setlength{\parindent}{0pt}

% footer
\pagestyle{fancyplain}
% rimuove la riga nell'header
\fancyhf{} % sets both header and footer to nothing
\renewcommand{\headrulewidth}{0pt}
\fancyfoot[L]{\href{https://github.com/Typing-Monkeys/AppuntiUniversita}{Typing Monkeys}}
\fancyfoot[C]{\emoji{gorilla}}
\fancyfoot[R]{\thepage}

% configurazione emoji
\usepackage{fontspec}
\usepackage{emoji}
\setemojifont{NotoColorEmoji.ttf}[Path=/usr/share/fonts/truetype/noto/]

\newtheorem{definition}{Definizione}
\newtheorem{lemma}{Lemma}
\newtheorem{theorem}{Teorema}
\newtheorem{corollary}{Corollario}

%% cambio nome al comando proof
\renewcommand*{\proofname}{Dimostrazione}

\begin{document}
\include{frontmatter/main.tex}

\tableofcontents

\include{quote/main.tex}

%% Aggiungere i capitoli qui sotto
\include{capitoli/richiami/main.tex}
\include{capitoli/immagini/main.tex}

\end{document}


\end{document}


\end{document}


%% Aggiungere i capitoli qui sotto
\documentclass[a4paper,12 pt]{report}
\usepackage[T1]{fontenc}
\usepackage[utf8]{inputenc}
\usepackage{lmodern}
\usepackage{listings}

\usepackage{float}
\usepackage{subcaption}
\usepackage{wrapfig}
\usepackage{fancyhdr}
\usepackage{amsthm}

\usepackage{pgfplots}
\setlength {\marginparwidth }{2cm}
\usepackage{todonotes}
\newcommand{\TODO}[2][]
{\todo[size=\scriptsize, color=red, #1]{#2}}


asdf
\pgfplotsset{compat=1.18}
% forza le footnote a stare il più in basso possibile
\usepackage[bottom]{footmisc}


%% STILE LISTINGS
%%aaa
\usepackage{xcolor}

\definecolor{codegreen}{rgb}{0,0.6,0}
\definecolor{codegray}{rgb}{0.5,0.5,0.5}
\definecolor{codepurple}{rgb}{0.58,0,0.82}
\definecolor{backcolour}{rgb}{0.95,0.95,0.92}

\lstdefinestyle{mystyle}{
    backgroundcolor=\color{backcolour},   
    commentstyle=\color{codegreen},
    keywordstyle=\color{magenta},
    numberstyle=\tiny\color{codegray},
    stringstyle=\color{codepurple},
    basicstyle=\ttfamily\footnotesize,
    breakatwhitespace=false,         
    breaklines=true,                 
    captionpos=b,                    
    keepspaces=true,                 
    numbers=left,                    
    numbersep=5pt,                  
    showspaces=false,                
    showstringspaces=false,
    showtabs=false,                  
    tabsize=2
}

\lstset{style=mystyle}

%% -----

% mostra le subsubsection nell'indice
\setcounter{tocdepth}{3}
\setcounter{secnumdepth}{3}

% Resetta la numerazione dei chapter quando
% una nuova part viene creata
\makeatletter
\@addtoreset{chapter}{part}
\makeatother

% Rimuove l'indentazione quando si crea un nuovo paragrafo
\setlength{\parindent}{0pt}

% footer
\pagestyle{fancyplain}
% rimuove la riga nell'header
\fancyhf{} % sets both header and footer to nothing
\renewcommand{\headrulewidth}{0pt}
\fancyfoot[L]{\href{https://github.com/Typing-Monkeys/AppuntiUniversita}{Typing Monkeys}}
\fancyfoot[C]{\emoji{gorilla}}
\fancyfoot[R]{\thepage}

% configurazione emoji
\usepackage{fontspec}
\usepackage{emoji}
\setemojifont{NotoColorEmoji.ttf}[Path=/usr/share/fonts/truetype/noto/]

\newtheorem{definition}{Definizione}
\newtheorem{lemma}{Lemma}
\newtheorem{theorem}{Teorema}
\newtheorem{corollary}{Corollario}

%% cambio nome al comando proof
\renewcommand*{\proofname}{Dimostrazione}

\begin{document}
\documentclass[a4paper,12 pt]{report}
\usepackage[T1]{fontenc}
\usepackage[utf8]{inputenc}
\usepackage{lmodern}
\usepackage{listings}

\usepackage{float}
\usepackage{subcaption}
\usepackage{wrapfig}
\usepackage{fancyhdr}
\usepackage{amsthm}

\usepackage{pgfplots}
\setlength {\marginparwidth }{2cm}
\usepackage{todonotes}
\newcommand{\TODO}[2][]
{\todo[size=\scriptsize, color=red, #1]{#2}}


asdf
\pgfplotsset{compat=1.18}
% forza le footnote a stare il più in basso possibile
\usepackage[bottom]{footmisc}


%% STILE LISTINGS
%%aaa
\usepackage{xcolor}

\definecolor{codegreen}{rgb}{0,0.6,0}
\definecolor{codegray}{rgb}{0.5,0.5,0.5}
\definecolor{codepurple}{rgb}{0.58,0,0.82}
\definecolor{backcolour}{rgb}{0.95,0.95,0.92}

\lstdefinestyle{mystyle}{
    backgroundcolor=\color{backcolour},   
    commentstyle=\color{codegreen},
    keywordstyle=\color{magenta},
    numberstyle=\tiny\color{codegray},
    stringstyle=\color{codepurple},
    basicstyle=\ttfamily\footnotesize,
    breakatwhitespace=false,         
    breaklines=true,                 
    captionpos=b,                    
    keepspaces=true,                 
    numbers=left,                    
    numbersep=5pt,                  
    showspaces=false,                
    showstringspaces=false,
    showtabs=false,                  
    tabsize=2
}

\lstset{style=mystyle}

%% -----

% mostra le subsubsection nell'indice
\setcounter{tocdepth}{3}
\setcounter{secnumdepth}{3}

% Resetta la numerazione dei chapter quando
% una nuova part viene creata
\makeatletter
\@addtoreset{chapter}{part}
\makeatother

% Rimuove l'indentazione quando si crea un nuovo paragrafo
\setlength{\parindent}{0pt}

% footer
\pagestyle{fancyplain}
% rimuove la riga nell'header
\fancyhf{} % sets both header and footer to nothing
\renewcommand{\headrulewidth}{0pt}
\fancyfoot[L]{\href{https://github.com/Typing-Monkeys/AppuntiUniversita}{Typing Monkeys}}
\fancyfoot[C]{\emoji{gorilla}}
\fancyfoot[R]{\thepage}

% configurazione emoji
\usepackage{fontspec}
\usepackage{emoji}
\setemojifont{NotoColorEmoji.ttf}[Path=/usr/share/fonts/truetype/noto/]

\newtheorem{definition}{Definizione}
\newtheorem{lemma}{Lemma}
\newtheorem{theorem}{Teorema}
\newtheorem{corollary}{Corollario}

%% cambio nome al comando proof
\renewcommand*{\proofname}{Dimostrazione}

\begin{document}
\documentclass[a4paper,12 pt]{report}
\usepackage[T1]{fontenc}
\usepackage[utf8]{inputenc}
\usepackage{lmodern}
\usepackage{listings}

\usepackage{float}
\usepackage{subcaption}
\usepackage{wrapfig}
\usepackage{fancyhdr}
\usepackage{amsthm}

\usepackage{pgfplots}
\setlength {\marginparwidth }{2cm}
\usepackage{todonotes}
\newcommand{\TODO}[2][]
{\todo[size=\scriptsize, color=red, #1]{#2}}


asdf
\pgfplotsset{compat=1.18}
% forza le footnote a stare il più in basso possibile
\usepackage[bottom]{footmisc}


%% STILE LISTINGS
%%aaa
\usepackage{xcolor}

\definecolor{codegreen}{rgb}{0,0.6,0}
\definecolor{codegray}{rgb}{0.5,0.5,0.5}
\definecolor{codepurple}{rgb}{0.58,0,0.82}
\definecolor{backcolour}{rgb}{0.95,0.95,0.92}

\lstdefinestyle{mystyle}{
    backgroundcolor=\color{backcolour},   
    commentstyle=\color{codegreen},
    keywordstyle=\color{magenta},
    numberstyle=\tiny\color{codegray},
    stringstyle=\color{codepurple},
    basicstyle=\ttfamily\footnotesize,
    breakatwhitespace=false,         
    breaklines=true,                 
    captionpos=b,                    
    keepspaces=true,                 
    numbers=left,                    
    numbersep=5pt,                  
    showspaces=false,                
    showstringspaces=false,
    showtabs=false,                  
    tabsize=2
}

\lstset{style=mystyle}

%% -----

% mostra le subsubsection nell'indice
\setcounter{tocdepth}{3}
\setcounter{secnumdepth}{3}

% Resetta la numerazione dei chapter quando
% una nuova part viene creata
\makeatletter
\@addtoreset{chapter}{part}
\makeatother

% Rimuove l'indentazione quando si crea un nuovo paragrafo
\setlength{\parindent}{0pt}

% footer
\pagestyle{fancyplain}
% rimuove la riga nell'header
\fancyhf{} % sets both header and footer to nothing
\renewcommand{\headrulewidth}{0pt}
\fancyfoot[L]{\href{https://github.com/Typing-Monkeys/AppuntiUniversita}{Typing Monkeys}}
\fancyfoot[C]{\emoji{gorilla}}
\fancyfoot[R]{\thepage}

% configurazione emoji
\usepackage{fontspec}
\usepackage{emoji}
\setemojifont{NotoColorEmoji.ttf}[Path=/usr/share/fonts/truetype/noto/]

\newtheorem{definition}{Definizione}
\newtheorem{lemma}{Lemma}
\newtheorem{theorem}{Teorema}
\newtheorem{corollary}{Corollario}

%% cambio nome al comando proof
\renewcommand*{\proofname}{Dimostrazione}

\begin{document}
\include{frontmatter/main.tex}

\tableofcontents

\include{quote/main.tex}

%% Aggiungere i capitoli qui sotto
\include{capitoli/richiami/main.tex}
\include{capitoli/immagini/main.tex}

\end{document}


\tableofcontents

\documentclass[a4paper,12 pt]{report}
\usepackage[T1]{fontenc}
\usepackage[utf8]{inputenc}
\usepackage{lmodern}
\usepackage{listings}

\usepackage{float}
\usepackage{subcaption}
\usepackage{wrapfig}
\usepackage{fancyhdr}
\usepackage{amsthm}

\usepackage{pgfplots}
\setlength {\marginparwidth }{2cm}
\usepackage{todonotes}
\newcommand{\TODO}[2][]
{\todo[size=\scriptsize, color=red, #1]{#2}}


asdf
\pgfplotsset{compat=1.18}
% forza le footnote a stare il più in basso possibile
\usepackage[bottom]{footmisc}


%% STILE LISTINGS
%%aaa
\usepackage{xcolor}

\definecolor{codegreen}{rgb}{0,0.6,0}
\definecolor{codegray}{rgb}{0.5,0.5,0.5}
\definecolor{codepurple}{rgb}{0.58,0,0.82}
\definecolor{backcolour}{rgb}{0.95,0.95,0.92}

\lstdefinestyle{mystyle}{
    backgroundcolor=\color{backcolour},   
    commentstyle=\color{codegreen},
    keywordstyle=\color{magenta},
    numberstyle=\tiny\color{codegray},
    stringstyle=\color{codepurple},
    basicstyle=\ttfamily\footnotesize,
    breakatwhitespace=false,         
    breaklines=true,                 
    captionpos=b,                    
    keepspaces=true,                 
    numbers=left,                    
    numbersep=5pt,                  
    showspaces=false,                
    showstringspaces=false,
    showtabs=false,                  
    tabsize=2
}

\lstset{style=mystyle}

%% -----

% mostra le subsubsection nell'indice
\setcounter{tocdepth}{3}
\setcounter{secnumdepth}{3}

% Resetta la numerazione dei chapter quando
% una nuova part viene creata
\makeatletter
\@addtoreset{chapter}{part}
\makeatother

% Rimuove l'indentazione quando si crea un nuovo paragrafo
\setlength{\parindent}{0pt}

% footer
\pagestyle{fancyplain}
% rimuove la riga nell'header
\fancyhf{} % sets both header and footer to nothing
\renewcommand{\headrulewidth}{0pt}
\fancyfoot[L]{\href{https://github.com/Typing-Monkeys/AppuntiUniversita}{Typing Monkeys}}
\fancyfoot[C]{\emoji{gorilla}}
\fancyfoot[R]{\thepage}

% configurazione emoji
\usepackage{fontspec}
\usepackage{emoji}
\setemojifont{NotoColorEmoji.ttf}[Path=/usr/share/fonts/truetype/noto/]

\newtheorem{definition}{Definizione}
\newtheorem{lemma}{Lemma}
\newtheorem{theorem}{Teorema}
\newtheorem{corollary}{Corollario}

%% cambio nome al comando proof
\renewcommand*{\proofname}{Dimostrazione}

\begin{document}
\include{frontmatter/main.tex}

\tableofcontents

\include{quote/main.tex}

%% Aggiungere i capitoli qui sotto
\include{capitoli/richiami/main.tex}
\include{capitoli/immagini/main.tex}

\end{document}


%% Aggiungere i capitoli qui sotto
\documentclass[a4paper,12 pt]{report}
\usepackage[T1]{fontenc}
\usepackage[utf8]{inputenc}
\usepackage{lmodern}
\usepackage{listings}

\usepackage{float}
\usepackage{subcaption}
\usepackage{wrapfig}
\usepackage{fancyhdr}
\usepackage{amsthm}

\usepackage{pgfplots}
\setlength {\marginparwidth }{2cm}
\usepackage{todonotes}
\newcommand{\TODO}[2][]
{\todo[size=\scriptsize, color=red, #1]{#2}}


asdf
\pgfplotsset{compat=1.18}
% forza le footnote a stare il più in basso possibile
\usepackage[bottom]{footmisc}


%% STILE LISTINGS
%%aaa
\usepackage{xcolor}

\definecolor{codegreen}{rgb}{0,0.6,0}
\definecolor{codegray}{rgb}{0.5,0.5,0.5}
\definecolor{codepurple}{rgb}{0.58,0,0.82}
\definecolor{backcolour}{rgb}{0.95,0.95,0.92}

\lstdefinestyle{mystyle}{
    backgroundcolor=\color{backcolour},   
    commentstyle=\color{codegreen},
    keywordstyle=\color{magenta},
    numberstyle=\tiny\color{codegray},
    stringstyle=\color{codepurple},
    basicstyle=\ttfamily\footnotesize,
    breakatwhitespace=false,         
    breaklines=true,                 
    captionpos=b,                    
    keepspaces=true,                 
    numbers=left,                    
    numbersep=5pt,                  
    showspaces=false,                
    showstringspaces=false,
    showtabs=false,                  
    tabsize=2
}

\lstset{style=mystyle}

%% -----

% mostra le subsubsection nell'indice
\setcounter{tocdepth}{3}
\setcounter{secnumdepth}{3}

% Resetta la numerazione dei chapter quando
% una nuova part viene creata
\makeatletter
\@addtoreset{chapter}{part}
\makeatother

% Rimuove l'indentazione quando si crea un nuovo paragrafo
\setlength{\parindent}{0pt}

% footer
\pagestyle{fancyplain}
% rimuove la riga nell'header
\fancyhf{} % sets both header and footer to nothing
\renewcommand{\headrulewidth}{0pt}
\fancyfoot[L]{\href{https://github.com/Typing-Monkeys/AppuntiUniversita}{Typing Monkeys}}
\fancyfoot[C]{\emoji{gorilla}}
\fancyfoot[R]{\thepage}

% configurazione emoji
\usepackage{fontspec}
\usepackage{emoji}
\setemojifont{NotoColorEmoji.ttf}[Path=/usr/share/fonts/truetype/noto/]

\newtheorem{definition}{Definizione}
\newtheorem{lemma}{Lemma}
\newtheorem{theorem}{Teorema}
\newtheorem{corollary}{Corollario}

%% cambio nome al comando proof
\renewcommand*{\proofname}{Dimostrazione}

\begin{document}
\include{frontmatter/main.tex}

\tableofcontents

\include{quote/main.tex}

%% Aggiungere i capitoli qui sotto
\include{capitoli/richiami/main.tex}
\include{capitoli/immagini/main.tex}

\end{document}

\documentclass[a4paper,12 pt]{report}
\usepackage[T1]{fontenc}
\usepackage[utf8]{inputenc}
\usepackage{lmodern}
\usepackage{listings}

\usepackage{float}
\usepackage{subcaption}
\usepackage{wrapfig}
\usepackage{fancyhdr}
\usepackage{amsthm}

\usepackage{pgfplots}
\setlength {\marginparwidth }{2cm}
\usepackage{todonotes}
\newcommand{\TODO}[2][]
{\todo[size=\scriptsize, color=red, #1]{#2}}


asdf
\pgfplotsset{compat=1.18}
% forza le footnote a stare il più in basso possibile
\usepackage[bottom]{footmisc}


%% STILE LISTINGS
%%aaa
\usepackage{xcolor}

\definecolor{codegreen}{rgb}{0,0.6,0}
\definecolor{codegray}{rgb}{0.5,0.5,0.5}
\definecolor{codepurple}{rgb}{0.58,0,0.82}
\definecolor{backcolour}{rgb}{0.95,0.95,0.92}

\lstdefinestyle{mystyle}{
    backgroundcolor=\color{backcolour},   
    commentstyle=\color{codegreen},
    keywordstyle=\color{magenta},
    numberstyle=\tiny\color{codegray},
    stringstyle=\color{codepurple},
    basicstyle=\ttfamily\footnotesize,
    breakatwhitespace=false,         
    breaklines=true,                 
    captionpos=b,                    
    keepspaces=true,                 
    numbers=left,                    
    numbersep=5pt,                  
    showspaces=false,                
    showstringspaces=false,
    showtabs=false,                  
    tabsize=2
}

\lstset{style=mystyle}

%% -----

% mostra le subsubsection nell'indice
\setcounter{tocdepth}{3}
\setcounter{secnumdepth}{3}

% Resetta la numerazione dei chapter quando
% una nuova part viene creata
\makeatletter
\@addtoreset{chapter}{part}
\makeatother

% Rimuove l'indentazione quando si crea un nuovo paragrafo
\setlength{\parindent}{0pt}

% footer
\pagestyle{fancyplain}
% rimuove la riga nell'header
\fancyhf{} % sets both header and footer to nothing
\renewcommand{\headrulewidth}{0pt}
\fancyfoot[L]{\href{https://github.com/Typing-Monkeys/AppuntiUniversita}{Typing Monkeys}}
\fancyfoot[C]{\emoji{gorilla}}
\fancyfoot[R]{\thepage}

% configurazione emoji
\usepackage{fontspec}
\usepackage{emoji}
\setemojifont{NotoColorEmoji.ttf}[Path=/usr/share/fonts/truetype/noto/]

\newtheorem{definition}{Definizione}
\newtheorem{lemma}{Lemma}
\newtheorem{theorem}{Teorema}
\newtheorem{corollary}{Corollario}

%% cambio nome al comando proof
\renewcommand*{\proofname}{Dimostrazione}

\begin{document}
\include{frontmatter/main.tex}

\tableofcontents

\include{quote/main.tex}

%% Aggiungere i capitoli qui sotto
\include{capitoli/richiami/main.tex}
\include{capitoli/immagini/main.tex}

\end{document}


\end{document}


\tableofcontents

\documentclass[a4paper,12 pt]{report}
\usepackage[T1]{fontenc}
\usepackage[utf8]{inputenc}
\usepackage{lmodern}
\usepackage{listings}

\usepackage{float}
\usepackage{subcaption}
\usepackage{wrapfig}
\usepackage{fancyhdr}
\usepackage{amsthm}

\usepackage{pgfplots}
\setlength {\marginparwidth }{2cm}
\usepackage{todonotes}
\newcommand{\TODO}[2][]
{\todo[size=\scriptsize, color=red, #1]{#2}}


asdf
\pgfplotsset{compat=1.18}
% forza le footnote a stare il più in basso possibile
\usepackage[bottom]{footmisc}


%% STILE LISTINGS
%%aaa
\usepackage{xcolor}

\definecolor{codegreen}{rgb}{0,0.6,0}
\definecolor{codegray}{rgb}{0.5,0.5,0.5}
\definecolor{codepurple}{rgb}{0.58,0,0.82}
\definecolor{backcolour}{rgb}{0.95,0.95,0.92}

\lstdefinestyle{mystyle}{
    backgroundcolor=\color{backcolour},   
    commentstyle=\color{codegreen},
    keywordstyle=\color{magenta},
    numberstyle=\tiny\color{codegray},
    stringstyle=\color{codepurple},
    basicstyle=\ttfamily\footnotesize,
    breakatwhitespace=false,         
    breaklines=true,                 
    captionpos=b,                    
    keepspaces=true,                 
    numbers=left,                    
    numbersep=5pt,                  
    showspaces=false,                
    showstringspaces=false,
    showtabs=false,                  
    tabsize=2
}

\lstset{style=mystyle}

%% -----

% mostra le subsubsection nell'indice
\setcounter{tocdepth}{3}
\setcounter{secnumdepth}{3}

% Resetta la numerazione dei chapter quando
% una nuova part viene creata
\makeatletter
\@addtoreset{chapter}{part}
\makeatother

% Rimuove l'indentazione quando si crea un nuovo paragrafo
\setlength{\parindent}{0pt}

% footer
\pagestyle{fancyplain}
% rimuove la riga nell'header
\fancyhf{} % sets both header and footer to nothing
\renewcommand{\headrulewidth}{0pt}
\fancyfoot[L]{\href{https://github.com/Typing-Monkeys/AppuntiUniversita}{Typing Monkeys}}
\fancyfoot[C]{\emoji{gorilla}}
\fancyfoot[R]{\thepage}

% configurazione emoji
\usepackage{fontspec}
\usepackage{emoji}
\setemojifont{NotoColorEmoji.ttf}[Path=/usr/share/fonts/truetype/noto/]

\newtheorem{definition}{Definizione}
\newtheorem{lemma}{Lemma}
\newtheorem{theorem}{Teorema}
\newtheorem{corollary}{Corollario}

%% cambio nome al comando proof
\renewcommand*{\proofname}{Dimostrazione}

\begin{document}
\documentclass[a4paper,12 pt]{report}
\usepackage[T1]{fontenc}
\usepackage[utf8]{inputenc}
\usepackage{lmodern}
\usepackage{listings}

\usepackage{float}
\usepackage{subcaption}
\usepackage{wrapfig}
\usepackage{fancyhdr}
\usepackage{amsthm}

\usepackage{pgfplots}
\setlength {\marginparwidth }{2cm}
\usepackage{todonotes}
\newcommand{\TODO}[2][]
{\todo[size=\scriptsize, color=red, #1]{#2}}


asdf
\pgfplotsset{compat=1.18}
% forza le footnote a stare il più in basso possibile
\usepackage[bottom]{footmisc}


%% STILE LISTINGS
%%aaa
\usepackage{xcolor}

\definecolor{codegreen}{rgb}{0,0.6,0}
\definecolor{codegray}{rgb}{0.5,0.5,0.5}
\definecolor{codepurple}{rgb}{0.58,0,0.82}
\definecolor{backcolour}{rgb}{0.95,0.95,0.92}

\lstdefinestyle{mystyle}{
    backgroundcolor=\color{backcolour},   
    commentstyle=\color{codegreen},
    keywordstyle=\color{magenta},
    numberstyle=\tiny\color{codegray},
    stringstyle=\color{codepurple},
    basicstyle=\ttfamily\footnotesize,
    breakatwhitespace=false,         
    breaklines=true,                 
    captionpos=b,                    
    keepspaces=true,                 
    numbers=left,                    
    numbersep=5pt,                  
    showspaces=false,                
    showstringspaces=false,
    showtabs=false,                  
    tabsize=2
}

\lstset{style=mystyle}

%% -----

% mostra le subsubsection nell'indice
\setcounter{tocdepth}{3}
\setcounter{secnumdepth}{3}

% Resetta la numerazione dei chapter quando
% una nuova part viene creata
\makeatletter
\@addtoreset{chapter}{part}
\makeatother

% Rimuove l'indentazione quando si crea un nuovo paragrafo
\setlength{\parindent}{0pt}

% footer
\pagestyle{fancyplain}
% rimuove la riga nell'header
\fancyhf{} % sets both header and footer to nothing
\renewcommand{\headrulewidth}{0pt}
\fancyfoot[L]{\href{https://github.com/Typing-Monkeys/AppuntiUniversita}{Typing Monkeys}}
\fancyfoot[C]{\emoji{gorilla}}
\fancyfoot[R]{\thepage}

% configurazione emoji
\usepackage{fontspec}
\usepackage{emoji}
\setemojifont{NotoColorEmoji.ttf}[Path=/usr/share/fonts/truetype/noto/]

\newtheorem{definition}{Definizione}
\newtheorem{lemma}{Lemma}
\newtheorem{theorem}{Teorema}
\newtheorem{corollary}{Corollario}

%% cambio nome al comando proof
\renewcommand*{\proofname}{Dimostrazione}

\begin{document}
\include{frontmatter/main.tex}

\tableofcontents

\include{quote/main.tex}

%% Aggiungere i capitoli qui sotto
\include{capitoli/richiami/main.tex}
\include{capitoli/immagini/main.tex}

\end{document}


\tableofcontents

\documentclass[a4paper,12 pt]{report}
\usepackage[T1]{fontenc}
\usepackage[utf8]{inputenc}
\usepackage{lmodern}
\usepackage{listings}

\usepackage{float}
\usepackage{subcaption}
\usepackage{wrapfig}
\usepackage{fancyhdr}
\usepackage{amsthm}

\usepackage{pgfplots}
\setlength {\marginparwidth }{2cm}
\usepackage{todonotes}
\newcommand{\TODO}[2][]
{\todo[size=\scriptsize, color=red, #1]{#2}}


asdf
\pgfplotsset{compat=1.18}
% forza le footnote a stare il più in basso possibile
\usepackage[bottom]{footmisc}


%% STILE LISTINGS
%%aaa
\usepackage{xcolor}

\definecolor{codegreen}{rgb}{0,0.6,0}
\definecolor{codegray}{rgb}{0.5,0.5,0.5}
\definecolor{codepurple}{rgb}{0.58,0,0.82}
\definecolor{backcolour}{rgb}{0.95,0.95,0.92}

\lstdefinestyle{mystyle}{
    backgroundcolor=\color{backcolour},   
    commentstyle=\color{codegreen},
    keywordstyle=\color{magenta},
    numberstyle=\tiny\color{codegray},
    stringstyle=\color{codepurple},
    basicstyle=\ttfamily\footnotesize,
    breakatwhitespace=false,         
    breaklines=true,                 
    captionpos=b,                    
    keepspaces=true,                 
    numbers=left,                    
    numbersep=5pt,                  
    showspaces=false,                
    showstringspaces=false,
    showtabs=false,                  
    tabsize=2
}

\lstset{style=mystyle}

%% -----

% mostra le subsubsection nell'indice
\setcounter{tocdepth}{3}
\setcounter{secnumdepth}{3}

% Resetta la numerazione dei chapter quando
% una nuova part viene creata
\makeatletter
\@addtoreset{chapter}{part}
\makeatother

% Rimuove l'indentazione quando si crea un nuovo paragrafo
\setlength{\parindent}{0pt}

% footer
\pagestyle{fancyplain}
% rimuove la riga nell'header
\fancyhf{} % sets both header and footer to nothing
\renewcommand{\headrulewidth}{0pt}
\fancyfoot[L]{\href{https://github.com/Typing-Monkeys/AppuntiUniversita}{Typing Monkeys}}
\fancyfoot[C]{\emoji{gorilla}}
\fancyfoot[R]{\thepage}

% configurazione emoji
\usepackage{fontspec}
\usepackage{emoji}
\setemojifont{NotoColorEmoji.ttf}[Path=/usr/share/fonts/truetype/noto/]

\newtheorem{definition}{Definizione}
\newtheorem{lemma}{Lemma}
\newtheorem{theorem}{Teorema}
\newtheorem{corollary}{Corollario}

%% cambio nome al comando proof
\renewcommand*{\proofname}{Dimostrazione}

\begin{document}
\include{frontmatter/main.tex}

\tableofcontents

\include{quote/main.tex}

%% Aggiungere i capitoli qui sotto
\include{capitoli/richiami/main.tex}
\include{capitoli/immagini/main.tex}

\end{document}


%% Aggiungere i capitoli qui sotto
\documentclass[a4paper,12 pt]{report}
\usepackage[T1]{fontenc}
\usepackage[utf8]{inputenc}
\usepackage{lmodern}
\usepackage{listings}

\usepackage{float}
\usepackage{subcaption}
\usepackage{wrapfig}
\usepackage{fancyhdr}
\usepackage{amsthm}

\usepackage{pgfplots}
\setlength {\marginparwidth }{2cm}
\usepackage{todonotes}
\newcommand{\TODO}[2][]
{\todo[size=\scriptsize, color=red, #1]{#2}}


asdf
\pgfplotsset{compat=1.18}
% forza le footnote a stare il più in basso possibile
\usepackage[bottom]{footmisc}


%% STILE LISTINGS
%%aaa
\usepackage{xcolor}

\definecolor{codegreen}{rgb}{0,0.6,0}
\definecolor{codegray}{rgb}{0.5,0.5,0.5}
\definecolor{codepurple}{rgb}{0.58,0,0.82}
\definecolor{backcolour}{rgb}{0.95,0.95,0.92}

\lstdefinestyle{mystyle}{
    backgroundcolor=\color{backcolour},   
    commentstyle=\color{codegreen},
    keywordstyle=\color{magenta},
    numberstyle=\tiny\color{codegray},
    stringstyle=\color{codepurple},
    basicstyle=\ttfamily\footnotesize,
    breakatwhitespace=false,         
    breaklines=true,                 
    captionpos=b,                    
    keepspaces=true,                 
    numbers=left,                    
    numbersep=5pt,                  
    showspaces=false,                
    showstringspaces=false,
    showtabs=false,                  
    tabsize=2
}

\lstset{style=mystyle}

%% -----

% mostra le subsubsection nell'indice
\setcounter{tocdepth}{3}
\setcounter{secnumdepth}{3}

% Resetta la numerazione dei chapter quando
% una nuova part viene creata
\makeatletter
\@addtoreset{chapter}{part}
\makeatother

% Rimuove l'indentazione quando si crea un nuovo paragrafo
\setlength{\parindent}{0pt}

% footer
\pagestyle{fancyplain}
% rimuove la riga nell'header
\fancyhf{} % sets both header and footer to nothing
\renewcommand{\headrulewidth}{0pt}
\fancyfoot[L]{\href{https://github.com/Typing-Monkeys/AppuntiUniversita}{Typing Monkeys}}
\fancyfoot[C]{\emoji{gorilla}}
\fancyfoot[R]{\thepage}

% configurazione emoji
\usepackage{fontspec}
\usepackage{emoji}
\setemojifont{NotoColorEmoji.ttf}[Path=/usr/share/fonts/truetype/noto/]

\newtheorem{definition}{Definizione}
\newtheorem{lemma}{Lemma}
\newtheorem{theorem}{Teorema}
\newtheorem{corollary}{Corollario}

%% cambio nome al comando proof
\renewcommand*{\proofname}{Dimostrazione}

\begin{document}
\include{frontmatter/main.tex}

\tableofcontents

\include{quote/main.tex}

%% Aggiungere i capitoli qui sotto
\include{capitoli/richiami/main.tex}
\include{capitoli/immagini/main.tex}

\end{document}

\documentclass[a4paper,12 pt]{report}
\usepackage[T1]{fontenc}
\usepackage[utf8]{inputenc}
\usepackage{lmodern}
\usepackage{listings}

\usepackage{float}
\usepackage{subcaption}
\usepackage{wrapfig}
\usepackage{fancyhdr}
\usepackage{amsthm}

\usepackage{pgfplots}
\setlength {\marginparwidth }{2cm}
\usepackage{todonotes}
\newcommand{\TODO}[2][]
{\todo[size=\scriptsize, color=red, #1]{#2}}


asdf
\pgfplotsset{compat=1.18}
% forza le footnote a stare il più in basso possibile
\usepackage[bottom]{footmisc}


%% STILE LISTINGS
%%aaa
\usepackage{xcolor}

\definecolor{codegreen}{rgb}{0,0.6,0}
\definecolor{codegray}{rgb}{0.5,0.5,0.5}
\definecolor{codepurple}{rgb}{0.58,0,0.82}
\definecolor{backcolour}{rgb}{0.95,0.95,0.92}

\lstdefinestyle{mystyle}{
    backgroundcolor=\color{backcolour},   
    commentstyle=\color{codegreen},
    keywordstyle=\color{magenta},
    numberstyle=\tiny\color{codegray},
    stringstyle=\color{codepurple},
    basicstyle=\ttfamily\footnotesize,
    breakatwhitespace=false,         
    breaklines=true,                 
    captionpos=b,                    
    keepspaces=true,                 
    numbers=left,                    
    numbersep=5pt,                  
    showspaces=false,                
    showstringspaces=false,
    showtabs=false,                  
    tabsize=2
}

\lstset{style=mystyle}

%% -----

% mostra le subsubsection nell'indice
\setcounter{tocdepth}{3}
\setcounter{secnumdepth}{3}

% Resetta la numerazione dei chapter quando
% una nuova part viene creata
\makeatletter
\@addtoreset{chapter}{part}
\makeatother

% Rimuove l'indentazione quando si crea un nuovo paragrafo
\setlength{\parindent}{0pt}

% footer
\pagestyle{fancyplain}
% rimuove la riga nell'header
\fancyhf{} % sets both header and footer to nothing
\renewcommand{\headrulewidth}{0pt}
\fancyfoot[L]{\href{https://github.com/Typing-Monkeys/AppuntiUniversita}{Typing Monkeys}}
\fancyfoot[C]{\emoji{gorilla}}
\fancyfoot[R]{\thepage}

% configurazione emoji
\usepackage{fontspec}
\usepackage{emoji}
\setemojifont{NotoColorEmoji.ttf}[Path=/usr/share/fonts/truetype/noto/]

\newtheorem{definition}{Definizione}
\newtheorem{lemma}{Lemma}
\newtheorem{theorem}{Teorema}
\newtheorem{corollary}{Corollario}

%% cambio nome al comando proof
\renewcommand*{\proofname}{Dimostrazione}

\begin{document}
\include{frontmatter/main.tex}

\tableofcontents

\include{quote/main.tex}

%% Aggiungere i capitoli qui sotto
\include{capitoli/richiami/main.tex}
\include{capitoli/immagini/main.tex}

\end{document}


\end{document}


%% Aggiungere i capitoli qui sotto
\documentclass[a4paper,12 pt]{report}
\usepackage[T1]{fontenc}
\usepackage[utf8]{inputenc}
\usepackage{lmodern}
\usepackage{listings}

\usepackage{float}
\usepackage{subcaption}
\usepackage{wrapfig}
\usepackage{fancyhdr}
\usepackage{amsthm}

\usepackage{pgfplots}
\setlength {\marginparwidth }{2cm}
\usepackage{todonotes}
\newcommand{\TODO}[2][]
{\todo[size=\scriptsize, color=red, #1]{#2}}


asdf
\pgfplotsset{compat=1.18}
% forza le footnote a stare il più in basso possibile
\usepackage[bottom]{footmisc}


%% STILE LISTINGS
%%aaa
\usepackage{xcolor}

\definecolor{codegreen}{rgb}{0,0.6,0}
\definecolor{codegray}{rgb}{0.5,0.5,0.5}
\definecolor{codepurple}{rgb}{0.58,0,0.82}
\definecolor{backcolour}{rgb}{0.95,0.95,0.92}

\lstdefinestyle{mystyle}{
    backgroundcolor=\color{backcolour},   
    commentstyle=\color{codegreen},
    keywordstyle=\color{magenta},
    numberstyle=\tiny\color{codegray},
    stringstyle=\color{codepurple},
    basicstyle=\ttfamily\footnotesize,
    breakatwhitespace=false,         
    breaklines=true,                 
    captionpos=b,                    
    keepspaces=true,                 
    numbers=left,                    
    numbersep=5pt,                  
    showspaces=false,                
    showstringspaces=false,
    showtabs=false,                  
    tabsize=2
}

\lstset{style=mystyle}

%% -----

% mostra le subsubsection nell'indice
\setcounter{tocdepth}{3}
\setcounter{secnumdepth}{3}

% Resetta la numerazione dei chapter quando
% una nuova part viene creata
\makeatletter
\@addtoreset{chapter}{part}
\makeatother

% Rimuove l'indentazione quando si crea un nuovo paragrafo
\setlength{\parindent}{0pt}

% footer
\pagestyle{fancyplain}
% rimuove la riga nell'header
\fancyhf{} % sets both header and footer to nothing
\renewcommand{\headrulewidth}{0pt}
\fancyfoot[L]{\href{https://github.com/Typing-Monkeys/AppuntiUniversita}{Typing Monkeys}}
\fancyfoot[C]{\emoji{gorilla}}
\fancyfoot[R]{\thepage}

% configurazione emoji
\usepackage{fontspec}
\usepackage{emoji}
\setemojifont{NotoColorEmoji.ttf}[Path=/usr/share/fonts/truetype/noto/]

\newtheorem{definition}{Definizione}
\newtheorem{lemma}{Lemma}
\newtheorem{theorem}{Teorema}
\newtheorem{corollary}{Corollario}

%% cambio nome al comando proof
\renewcommand*{\proofname}{Dimostrazione}

\begin{document}
\documentclass[a4paper,12 pt]{report}
\usepackage[T1]{fontenc}
\usepackage[utf8]{inputenc}
\usepackage{lmodern}
\usepackage{listings}

\usepackage{float}
\usepackage{subcaption}
\usepackage{wrapfig}
\usepackage{fancyhdr}
\usepackage{amsthm}

\usepackage{pgfplots}
\setlength {\marginparwidth }{2cm}
\usepackage{todonotes}
\newcommand{\TODO}[2][]
{\todo[size=\scriptsize, color=red, #1]{#2}}


asdf
\pgfplotsset{compat=1.18}
% forza le footnote a stare il più in basso possibile
\usepackage[bottom]{footmisc}


%% STILE LISTINGS
%%aaa
\usepackage{xcolor}

\definecolor{codegreen}{rgb}{0,0.6,0}
\definecolor{codegray}{rgb}{0.5,0.5,0.5}
\definecolor{codepurple}{rgb}{0.58,0,0.82}
\definecolor{backcolour}{rgb}{0.95,0.95,0.92}

\lstdefinestyle{mystyle}{
    backgroundcolor=\color{backcolour},   
    commentstyle=\color{codegreen},
    keywordstyle=\color{magenta},
    numberstyle=\tiny\color{codegray},
    stringstyle=\color{codepurple},
    basicstyle=\ttfamily\footnotesize,
    breakatwhitespace=false,         
    breaklines=true,                 
    captionpos=b,                    
    keepspaces=true,                 
    numbers=left,                    
    numbersep=5pt,                  
    showspaces=false,                
    showstringspaces=false,
    showtabs=false,                  
    tabsize=2
}

\lstset{style=mystyle}

%% -----

% mostra le subsubsection nell'indice
\setcounter{tocdepth}{3}
\setcounter{secnumdepth}{3}

% Resetta la numerazione dei chapter quando
% una nuova part viene creata
\makeatletter
\@addtoreset{chapter}{part}
\makeatother

% Rimuove l'indentazione quando si crea un nuovo paragrafo
\setlength{\parindent}{0pt}

% footer
\pagestyle{fancyplain}
% rimuove la riga nell'header
\fancyhf{} % sets both header and footer to nothing
\renewcommand{\headrulewidth}{0pt}
\fancyfoot[L]{\href{https://github.com/Typing-Monkeys/AppuntiUniversita}{Typing Monkeys}}
\fancyfoot[C]{\emoji{gorilla}}
\fancyfoot[R]{\thepage}

% configurazione emoji
\usepackage{fontspec}
\usepackage{emoji}
\setemojifont{NotoColorEmoji.ttf}[Path=/usr/share/fonts/truetype/noto/]

\newtheorem{definition}{Definizione}
\newtheorem{lemma}{Lemma}
\newtheorem{theorem}{Teorema}
\newtheorem{corollary}{Corollario}

%% cambio nome al comando proof
\renewcommand*{\proofname}{Dimostrazione}

\begin{document}
\include{frontmatter/main.tex}

\tableofcontents

\include{quote/main.tex}

%% Aggiungere i capitoli qui sotto
\include{capitoli/richiami/main.tex}
\include{capitoli/immagini/main.tex}

\end{document}


\tableofcontents

\documentclass[a4paper,12 pt]{report}
\usepackage[T1]{fontenc}
\usepackage[utf8]{inputenc}
\usepackage{lmodern}
\usepackage{listings}

\usepackage{float}
\usepackage{subcaption}
\usepackage{wrapfig}
\usepackage{fancyhdr}
\usepackage{amsthm}

\usepackage{pgfplots}
\setlength {\marginparwidth }{2cm}
\usepackage{todonotes}
\newcommand{\TODO}[2][]
{\todo[size=\scriptsize, color=red, #1]{#2}}


asdf
\pgfplotsset{compat=1.18}
% forza le footnote a stare il più in basso possibile
\usepackage[bottom]{footmisc}


%% STILE LISTINGS
%%aaa
\usepackage{xcolor}

\definecolor{codegreen}{rgb}{0,0.6,0}
\definecolor{codegray}{rgb}{0.5,0.5,0.5}
\definecolor{codepurple}{rgb}{0.58,0,0.82}
\definecolor{backcolour}{rgb}{0.95,0.95,0.92}

\lstdefinestyle{mystyle}{
    backgroundcolor=\color{backcolour},   
    commentstyle=\color{codegreen},
    keywordstyle=\color{magenta},
    numberstyle=\tiny\color{codegray},
    stringstyle=\color{codepurple},
    basicstyle=\ttfamily\footnotesize,
    breakatwhitespace=false,         
    breaklines=true,                 
    captionpos=b,                    
    keepspaces=true,                 
    numbers=left,                    
    numbersep=5pt,                  
    showspaces=false,                
    showstringspaces=false,
    showtabs=false,                  
    tabsize=2
}

\lstset{style=mystyle}

%% -----

% mostra le subsubsection nell'indice
\setcounter{tocdepth}{3}
\setcounter{secnumdepth}{3}

% Resetta la numerazione dei chapter quando
% una nuova part viene creata
\makeatletter
\@addtoreset{chapter}{part}
\makeatother

% Rimuove l'indentazione quando si crea un nuovo paragrafo
\setlength{\parindent}{0pt}

% footer
\pagestyle{fancyplain}
% rimuove la riga nell'header
\fancyhf{} % sets both header and footer to nothing
\renewcommand{\headrulewidth}{0pt}
\fancyfoot[L]{\href{https://github.com/Typing-Monkeys/AppuntiUniversita}{Typing Monkeys}}
\fancyfoot[C]{\emoji{gorilla}}
\fancyfoot[R]{\thepage}

% configurazione emoji
\usepackage{fontspec}
\usepackage{emoji}
\setemojifont{NotoColorEmoji.ttf}[Path=/usr/share/fonts/truetype/noto/]

\newtheorem{definition}{Definizione}
\newtheorem{lemma}{Lemma}
\newtheorem{theorem}{Teorema}
\newtheorem{corollary}{Corollario}

%% cambio nome al comando proof
\renewcommand*{\proofname}{Dimostrazione}

\begin{document}
\include{frontmatter/main.tex}

\tableofcontents

\include{quote/main.tex}

%% Aggiungere i capitoli qui sotto
\include{capitoli/richiami/main.tex}
\include{capitoli/immagini/main.tex}

\end{document}


%% Aggiungere i capitoli qui sotto
\documentclass[a4paper,12 pt]{report}
\usepackage[T1]{fontenc}
\usepackage[utf8]{inputenc}
\usepackage{lmodern}
\usepackage{listings}

\usepackage{float}
\usepackage{subcaption}
\usepackage{wrapfig}
\usepackage{fancyhdr}
\usepackage{amsthm}

\usepackage{pgfplots}
\setlength {\marginparwidth }{2cm}
\usepackage{todonotes}
\newcommand{\TODO}[2][]
{\todo[size=\scriptsize, color=red, #1]{#2}}


asdf
\pgfplotsset{compat=1.18}
% forza le footnote a stare il più in basso possibile
\usepackage[bottom]{footmisc}


%% STILE LISTINGS
%%aaa
\usepackage{xcolor}

\definecolor{codegreen}{rgb}{0,0.6,0}
\definecolor{codegray}{rgb}{0.5,0.5,0.5}
\definecolor{codepurple}{rgb}{0.58,0,0.82}
\definecolor{backcolour}{rgb}{0.95,0.95,0.92}

\lstdefinestyle{mystyle}{
    backgroundcolor=\color{backcolour},   
    commentstyle=\color{codegreen},
    keywordstyle=\color{magenta},
    numberstyle=\tiny\color{codegray},
    stringstyle=\color{codepurple},
    basicstyle=\ttfamily\footnotesize,
    breakatwhitespace=false,         
    breaklines=true,                 
    captionpos=b,                    
    keepspaces=true,                 
    numbers=left,                    
    numbersep=5pt,                  
    showspaces=false,                
    showstringspaces=false,
    showtabs=false,                  
    tabsize=2
}

\lstset{style=mystyle}

%% -----

% mostra le subsubsection nell'indice
\setcounter{tocdepth}{3}
\setcounter{secnumdepth}{3}

% Resetta la numerazione dei chapter quando
% una nuova part viene creata
\makeatletter
\@addtoreset{chapter}{part}
\makeatother

% Rimuove l'indentazione quando si crea un nuovo paragrafo
\setlength{\parindent}{0pt}

% footer
\pagestyle{fancyplain}
% rimuove la riga nell'header
\fancyhf{} % sets both header and footer to nothing
\renewcommand{\headrulewidth}{0pt}
\fancyfoot[L]{\href{https://github.com/Typing-Monkeys/AppuntiUniversita}{Typing Monkeys}}
\fancyfoot[C]{\emoji{gorilla}}
\fancyfoot[R]{\thepage}

% configurazione emoji
\usepackage{fontspec}
\usepackage{emoji}
\setemojifont{NotoColorEmoji.ttf}[Path=/usr/share/fonts/truetype/noto/]

\newtheorem{definition}{Definizione}
\newtheorem{lemma}{Lemma}
\newtheorem{theorem}{Teorema}
\newtheorem{corollary}{Corollario}

%% cambio nome al comando proof
\renewcommand*{\proofname}{Dimostrazione}

\begin{document}
\include{frontmatter/main.tex}

\tableofcontents

\include{quote/main.tex}

%% Aggiungere i capitoli qui sotto
\include{capitoli/richiami/main.tex}
\include{capitoli/immagini/main.tex}

\end{document}

\documentclass[a4paper,12 pt]{report}
\usepackage[T1]{fontenc}
\usepackage[utf8]{inputenc}
\usepackage{lmodern}
\usepackage{listings}

\usepackage{float}
\usepackage{subcaption}
\usepackage{wrapfig}
\usepackage{fancyhdr}
\usepackage{amsthm}

\usepackage{pgfplots}
\setlength {\marginparwidth }{2cm}
\usepackage{todonotes}
\newcommand{\TODO}[2][]
{\todo[size=\scriptsize, color=red, #1]{#2}}


asdf
\pgfplotsset{compat=1.18}
% forza le footnote a stare il più in basso possibile
\usepackage[bottom]{footmisc}


%% STILE LISTINGS
%%aaa
\usepackage{xcolor}

\definecolor{codegreen}{rgb}{0,0.6,0}
\definecolor{codegray}{rgb}{0.5,0.5,0.5}
\definecolor{codepurple}{rgb}{0.58,0,0.82}
\definecolor{backcolour}{rgb}{0.95,0.95,0.92}

\lstdefinestyle{mystyle}{
    backgroundcolor=\color{backcolour},   
    commentstyle=\color{codegreen},
    keywordstyle=\color{magenta},
    numberstyle=\tiny\color{codegray},
    stringstyle=\color{codepurple},
    basicstyle=\ttfamily\footnotesize,
    breakatwhitespace=false,         
    breaklines=true,                 
    captionpos=b,                    
    keepspaces=true,                 
    numbers=left,                    
    numbersep=5pt,                  
    showspaces=false,                
    showstringspaces=false,
    showtabs=false,                  
    tabsize=2
}

\lstset{style=mystyle}

%% -----

% mostra le subsubsection nell'indice
\setcounter{tocdepth}{3}
\setcounter{secnumdepth}{3}

% Resetta la numerazione dei chapter quando
% una nuova part viene creata
\makeatletter
\@addtoreset{chapter}{part}
\makeatother

% Rimuove l'indentazione quando si crea un nuovo paragrafo
\setlength{\parindent}{0pt}

% footer
\pagestyle{fancyplain}
% rimuove la riga nell'header
\fancyhf{} % sets both header and footer to nothing
\renewcommand{\headrulewidth}{0pt}
\fancyfoot[L]{\href{https://github.com/Typing-Monkeys/AppuntiUniversita}{Typing Monkeys}}
\fancyfoot[C]{\emoji{gorilla}}
\fancyfoot[R]{\thepage}

% configurazione emoji
\usepackage{fontspec}
\usepackage{emoji}
\setemojifont{NotoColorEmoji.ttf}[Path=/usr/share/fonts/truetype/noto/]

\newtheorem{definition}{Definizione}
\newtheorem{lemma}{Lemma}
\newtheorem{theorem}{Teorema}
\newtheorem{corollary}{Corollario}

%% cambio nome al comando proof
\renewcommand*{\proofname}{Dimostrazione}

\begin{document}
\include{frontmatter/main.tex}

\tableofcontents

\include{quote/main.tex}

%% Aggiungere i capitoli qui sotto
\include{capitoli/richiami/main.tex}
\include{capitoli/immagini/main.tex}

\end{document}


\end{document}

\documentclass[a4paper,12 pt]{report}
\usepackage[T1]{fontenc}
\usepackage[utf8]{inputenc}
\usepackage{lmodern}
\usepackage{listings}

\usepackage{float}
\usepackage{subcaption}
\usepackage{wrapfig}
\usepackage{fancyhdr}
\usepackage{amsthm}

\usepackage{pgfplots}
\setlength {\marginparwidth }{2cm}
\usepackage{todonotes}
\newcommand{\TODO}[2][]
{\todo[size=\scriptsize, color=red, #1]{#2}}


asdf
\pgfplotsset{compat=1.18}
% forza le footnote a stare il più in basso possibile
\usepackage[bottom]{footmisc}


%% STILE LISTINGS
%%aaa
\usepackage{xcolor}

\definecolor{codegreen}{rgb}{0,0.6,0}
\definecolor{codegray}{rgb}{0.5,0.5,0.5}
\definecolor{codepurple}{rgb}{0.58,0,0.82}
\definecolor{backcolour}{rgb}{0.95,0.95,0.92}

\lstdefinestyle{mystyle}{
    backgroundcolor=\color{backcolour},   
    commentstyle=\color{codegreen},
    keywordstyle=\color{magenta},
    numberstyle=\tiny\color{codegray},
    stringstyle=\color{codepurple},
    basicstyle=\ttfamily\footnotesize,
    breakatwhitespace=false,         
    breaklines=true,                 
    captionpos=b,                    
    keepspaces=true,                 
    numbers=left,                    
    numbersep=5pt,                  
    showspaces=false,                
    showstringspaces=false,
    showtabs=false,                  
    tabsize=2
}

\lstset{style=mystyle}

%% -----

% mostra le subsubsection nell'indice
\setcounter{tocdepth}{3}
\setcounter{secnumdepth}{3}

% Resetta la numerazione dei chapter quando
% una nuova part viene creata
\makeatletter
\@addtoreset{chapter}{part}
\makeatother

% Rimuove l'indentazione quando si crea un nuovo paragrafo
\setlength{\parindent}{0pt}

% footer
\pagestyle{fancyplain}
% rimuove la riga nell'header
\fancyhf{} % sets both header and footer to nothing
\renewcommand{\headrulewidth}{0pt}
\fancyfoot[L]{\href{https://github.com/Typing-Monkeys/AppuntiUniversita}{Typing Monkeys}}
\fancyfoot[C]{\emoji{gorilla}}
\fancyfoot[R]{\thepage}

% configurazione emoji
\usepackage{fontspec}
\usepackage{emoji}
\setemojifont{NotoColorEmoji.ttf}[Path=/usr/share/fonts/truetype/noto/]

\newtheorem{definition}{Definizione}
\newtheorem{lemma}{Lemma}
\newtheorem{theorem}{Teorema}
\newtheorem{corollary}{Corollario}

%% cambio nome al comando proof
\renewcommand*{\proofname}{Dimostrazione}

\begin{document}
\documentclass[a4paper,12 pt]{report}
\usepackage[T1]{fontenc}
\usepackage[utf8]{inputenc}
\usepackage{lmodern}
\usepackage{listings}

\usepackage{float}
\usepackage{subcaption}
\usepackage{wrapfig}
\usepackage{fancyhdr}
\usepackage{amsthm}

\usepackage{pgfplots}
\setlength {\marginparwidth }{2cm}
\usepackage{todonotes}
\newcommand{\TODO}[2][]
{\todo[size=\scriptsize, color=red, #1]{#2}}


asdf
\pgfplotsset{compat=1.18}
% forza le footnote a stare il più in basso possibile
\usepackage[bottom]{footmisc}


%% STILE LISTINGS
%%aaa
\usepackage{xcolor}

\definecolor{codegreen}{rgb}{0,0.6,0}
\definecolor{codegray}{rgb}{0.5,0.5,0.5}
\definecolor{codepurple}{rgb}{0.58,0,0.82}
\definecolor{backcolour}{rgb}{0.95,0.95,0.92}

\lstdefinestyle{mystyle}{
    backgroundcolor=\color{backcolour},   
    commentstyle=\color{codegreen},
    keywordstyle=\color{magenta},
    numberstyle=\tiny\color{codegray},
    stringstyle=\color{codepurple},
    basicstyle=\ttfamily\footnotesize,
    breakatwhitespace=false,         
    breaklines=true,                 
    captionpos=b,                    
    keepspaces=true,                 
    numbers=left,                    
    numbersep=5pt,                  
    showspaces=false,                
    showstringspaces=false,
    showtabs=false,                  
    tabsize=2
}

\lstset{style=mystyle}

%% -----

% mostra le subsubsection nell'indice
\setcounter{tocdepth}{3}
\setcounter{secnumdepth}{3}

% Resetta la numerazione dei chapter quando
% una nuova part viene creata
\makeatletter
\@addtoreset{chapter}{part}
\makeatother

% Rimuove l'indentazione quando si crea un nuovo paragrafo
\setlength{\parindent}{0pt}

% footer
\pagestyle{fancyplain}
% rimuove la riga nell'header
\fancyhf{} % sets both header and footer to nothing
\renewcommand{\headrulewidth}{0pt}
\fancyfoot[L]{\href{https://github.com/Typing-Monkeys/AppuntiUniversita}{Typing Monkeys}}
\fancyfoot[C]{\emoji{gorilla}}
\fancyfoot[R]{\thepage}

% configurazione emoji
\usepackage{fontspec}
\usepackage{emoji}
\setemojifont{NotoColorEmoji.ttf}[Path=/usr/share/fonts/truetype/noto/]

\newtheorem{definition}{Definizione}
\newtheorem{lemma}{Lemma}
\newtheorem{theorem}{Teorema}
\newtheorem{corollary}{Corollario}

%% cambio nome al comando proof
\renewcommand*{\proofname}{Dimostrazione}

\begin{document}
\include{frontmatter/main.tex}

\tableofcontents

\include{quote/main.tex}

%% Aggiungere i capitoli qui sotto
\include{capitoli/richiami/main.tex}
\include{capitoli/immagini/main.tex}

\end{document}


\tableofcontents

\documentclass[a4paper,12 pt]{report}
\usepackage[T1]{fontenc}
\usepackage[utf8]{inputenc}
\usepackage{lmodern}
\usepackage{listings}

\usepackage{float}
\usepackage{subcaption}
\usepackage{wrapfig}
\usepackage{fancyhdr}
\usepackage{amsthm}

\usepackage{pgfplots}
\setlength {\marginparwidth }{2cm}
\usepackage{todonotes}
\newcommand{\TODO}[2][]
{\todo[size=\scriptsize, color=red, #1]{#2}}


asdf
\pgfplotsset{compat=1.18}
% forza le footnote a stare il più in basso possibile
\usepackage[bottom]{footmisc}


%% STILE LISTINGS
%%aaa
\usepackage{xcolor}

\definecolor{codegreen}{rgb}{0,0.6,0}
\definecolor{codegray}{rgb}{0.5,0.5,0.5}
\definecolor{codepurple}{rgb}{0.58,0,0.82}
\definecolor{backcolour}{rgb}{0.95,0.95,0.92}

\lstdefinestyle{mystyle}{
    backgroundcolor=\color{backcolour},   
    commentstyle=\color{codegreen},
    keywordstyle=\color{magenta},
    numberstyle=\tiny\color{codegray},
    stringstyle=\color{codepurple},
    basicstyle=\ttfamily\footnotesize,
    breakatwhitespace=false,         
    breaklines=true,                 
    captionpos=b,                    
    keepspaces=true,                 
    numbers=left,                    
    numbersep=5pt,                  
    showspaces=false,                
    showstringspaces=false,
    showtabs=false,                  
    tabsize=2
}

\lstset{style=mystyle}

%% -----

% mostra le subsubsection nell'indice
\setcounter{tocdepth}{3}
\setcounter{secnumdepth}{3}

% Resetta la numerazione dei chapter quando
% una nuova part viene creata
\makeatletter
\@addtoreset{chapter}{part}
\makeatother

% Rimuove l'indentazione quando si crea un nuovo paragrafo
\setlength{\parindent}{0pt}

% footer
\pagestyle{fancyplain}
% rimuove la riga nell'header
\fancyhf{} % sets both header and footer to nothing
\renewcommand{\headrulewidth}{0pt}
\fancyfoot[L]{\href{https://github.com/Typing-Monkeys/AppuntiUniversita}{Typing Monkeys}}
\fancyfoot[C]{\emoji{gorilla}}
\fancyfoot[R]{\thepage}

% configurazione emoji
\usepackage{fontspec}
\usepackage{emoji}
\setemojifont{NotoColorEmoji.ttf}[Path=/usr/share/fonts/truetype/noto/]

\newtheorem{definition}{Definizione}
\newtheorem{lemma}{Lemma}
\newtheorem{theorem}{Teorema}
\newtheorem{corollary}{Corollario}

%% cambio nome al comando proof
\renewcommand*{\proofname}{Dimostrazione}

\begin{document}
\include{frontmatter/main.tex}

\tableofcontents

\include{quote/main.tex}

%% Aggiungere i capitoli qui sotto
\include{capitoli/richiami/main.tex}
\include{capitoli/immagini/main.tex}

\end{document}


%% Aggiungere i capitoli qui sotto
\documentclass[a4paper,12 pt]{report}
\usepackage[T1]{fontenc}
\usepackage[utf8]{inputenc}
\usepackage{lmodern}
\usepackage{listings}

\usepackage{float}
\usepackage{subcaption}
\usepackage{wrapfig}
\usepackage{fancyhdr}
\usepackage{amsthm}

\usepackage{pgfplots}
\setlength {\marginparwidth }{2cm}
\usepackage{todonotes}
\newcommand{\TODO}[2][]
{\todo[size=\scriptsize, color=red, #1]{#2}}


asdf
\pgfplotsset{compat=1.18}
% forza le footnote a stare il più in basso possibile
\usepackage[bottom]{footmisc}


%% STILE LISTINGS
%%aaa
\usepackage{xcolor}

\definecolor{codegreen}{rgb}{0,0.6,0}
\definecolor{codegray}{rgb}{0.5,0.5,0.5}
\definecolor{codepurple}{rgb}{0.58,0,0.82}
\definecolor{backcolour}{rgb}{0.95,0.95,0.92}

\lstdefinestyle{mystyle}{
    backgroundcolor=\color{backcolour},   
    commentstyle=\color{codegreen},
    keywordstyle=\color{magenta},
    numberstyle=\tiny\color{codegray},
    stringstyle=\color{codepurple},
    basicstyle=\ttfamily\footnotesize,
    breakatwhitespace=false,         
    breaklines=true,                 
    captionpos=b,                    
    keepspaces=true,                 
    numbers=left,                    
    numbersep=5pt,                  
    showspaces=false,                
    showstringspaces=false,
    showtabs=false,                  
    tabsize=2
}

\lstset{style=mystyle}

%% -----

% mostra le subsubsection nell'indice
\setcounter{tocdepth}{3}
\setcounter{secnumdepth}{3}

% Resetta la numerazione dei chapter quando
% una nuova part viene creata
\makeatletter
\@addtoreset{chapter}{part}
\makeatother

% Rimuove l'indentazione quando si crea un nuovo paragrafo
\setlength{\parindent}{0pt}

% footer
\pagestyle{fancyplain}
% rimuove la riga nell'header
\fancyhf{} % sets both header and footer to nothing
\renewcommand{\headrulewidth}{0pt}
\fancyfoot[L]{\href{https://github.com/Typing-Monkeys/AppuntiUniversita}{Typing Monkeys}}
\fancyfoot[C]{\emoji{gorilla}}
\fancyfoot[R]{\thepage}

% configurazione emoji
\usepackage{fontspec}
\usepackage{emoji}
\setemojifont{NotoColorEmoji.ttf}[Path=/usr/share/fonts/truetype/noto/]

\newtheorem{definition}{Definizione}
\newtheorem{lemma}{Lemma}
\newtheorem{theorem}{Teorema}
\newtheorem{corollary}{Corollario}

%% cambio nome al comando proof
\renewcommand*{\proofname}{Dimostrazione}

\begin{document}
\include{frontmatter/main.tex}

\tableofcontents

\include{quote/main.tex}

%% Aggiungere i capitoli qui sotto
\include{capitoli/richiami/main.tex}
\include{capitoli/immagini/main.tex}

\end{document}

\documentclass[a4paper,12 pt]{report}
\usepackage[T1]{fontenc}
\usepackage[utf8]{inputenc}
\usepackage{lmodern}
\usepackage{listings}

\usepackage{float}
\usepackage{subcaption}
\usepackage{wrapfig}
\usepackage{fancyhdr}
\usepackage{amsthm}

\usepackage{pgfplots}
\setlength {\marginparwidth }{2cm}
\usepackage{todonotes}
\newcommand{\TODO}[2][]
{\todo[size=\scriptsize, color=red, #1]{#2}}


asdf
\pgfplotsset{compat=1.18}
% forza le footnote a stare il più in basso possibile
\usepackage[bottom]{footmisc}


%% STILE LISTINGS
%%aaa
\usepackage{xcolor}

\definecolor{codegreen}{rgb}{0,0.6,0}
\definecolor{codegray}{rgb}{0.5,0.5,0.5}
\definecolor{codepurple}{rgb}{0.58,0,0.82}
\definecolor{backcolour}{rgb}{0.95,0.95,0.92}

\lstdefinestyle{mystyle}{
    backgroundcolor=\color{backcolour},   
    commentstyle=\color{codegreen},
    keywordstyle=\color{magenta},
    numberstyle=\tiny\color{codegray},
    stringstyle=\color{codepurple},
    basicstyle=\ttfamily\footnotesize,
    breakatwhitespace=false,         
    breaklines=true,                 
    captionpos=b,                    
    keepspaces=true,                 
    numbers=left,                    
    numbersep=5pt,                  
    showspaces=false,                
    showstringspaces=false,
    showtabs=false,                  
    tabsize=2
}

\lstset{style=mystyle}

%% -----

% mostra le subsubsection nell'indice
\setcounter{tocdepth}{3}
\setcounter{secnumdepth}{3}

% Resetta la numerazione dei chapter quando
% una nuova part viene creata
\makeatletter
\@addtoreset{chapter}{part}
\makeatother

% Rimuove l'indentazione quando si crea un nuovo paragrafo
\setlength{\parindent}{0pt}

% footer
\pagestyle{fancyplain}
% rimuove la riga nell'header
\fancyhf{} % sets both header and footer to nothing
\renewcommand{\headrulewidth}{0pt}
\fancyfoot[L]{\href{https://github.com/Typing-Monkeys/AppuntiUniversita}{Typing Monkeys}}
\fancyfoot[C]{\emoji{gorilla}}
\fancyfoot[R]{\thepage}

% configurazione emoji
\usepackage{fontspec}
\usepackage{emoji}
\setemojifont{NotoColorEmoji.ttf}[Path=/usr/share/fonts/truetype/noto/]

\newtheorem{definition}{Definizione}
\newtheorem{lemma}{Lemma}
\newtheorem{theorem}{Teorema}
\newtheorem{corollary}{Corollario}

%% cambio nome al comando proof
\renewcommand*{\proofname}{Dimostrazione}

\begin{document}
\include{frontmatter/main.tex}

\tableofcontents

\include{quote/main.tex}

%% Aggiungere i capitoli qui sotto
\include{capitoli/richiami/main.tex}
\include{capitoli/immagini/main.tex}

\end{document}


\end{document}


\end{document}

\documentclass[a4paper,12 pt]{report}
\usepackage[T1]{fontenc}
\usepackage[utf8]{inputenc}
\usepackage{lmodern}
\usepackage{listings}

\usepackage{float}
\usepackage{subcaption}
\usepackage{wrapfig}
\usepackage{fancyhdr}
\usepackage{amsthm}

\usepackage{pgfplots}
\setlength {\marginparwidth }{2cm}
\usepackage{todonotes}
\newcommand{\TODO}[2][]
{\todo[size=\scriptsize, color=red, #1]{#2}}


asdf
\pgfplotsset{compat=1.18}
% forza le footnote a stare il più in basso possibile
\usepackage[bottom]{footmisc}


%% STILE LISTINGS
%%aaa
\usepackage{xcolor}

\definecolor{codegreen}{rgb}{0,0.6,0}
\definecolor{codegray}{rgb}{0.5,0.5,0.5}
\definecolor{codepurple}{rgb}{0.58,0,0.82}
\definecolor{backcolour}{rgb}{0.95,0.95,0.92}

\lstdefinestyle{mystyle}{
    backgroundcolor=\color{backcolour},   
    commentstyle=\color{codegreen},
    keywordstyle=\color{magenta},
    numberstyle=\tiny\color{codegray},
    stringstyle=\color{codepurple},
    basicstyle=\ttfamily\footnotesize,
    breakatwhitespace=false,         
    breaklines=true,                 
    captionpos=b,                    
    keepspaces=true,                 
    numbers=left,                    
    numbersep=5pt,                  
    showspaces=false,                
    showstringspaces=false,
    showtabs=false,                  
    tabsize=2
}

\lstset{style=mystyle}

%% -----

% mostra le subsubsection nell'indice
\setcounter{tocdepth}{3}
\setcounter{secnumdepth}{3}

% Resetta la numerazione dei chapter quando
% una nuova part viene creata
\makeatletter
\@addtoreset{chapter}{part}
\makeatother

% Rimuove l'indentazione quando si crea un nuovo paragrafo
\setlength{\parindent}{0pt}

% footer
\pagestyle{fancyplain}
% rimuove la riga nell'header
\fancyhf{} % sets both header and footer to nothing
\renewcommand{\headrulewidth}{0pt}
\fancyfoot[L]{\href{https://github.com/Typing-Monkeys/AppuntiUniversita}{Typing Monkeys}}
\fancyfoot[C]{\emoji{gorilla}}
\fancyfoot[R]{\thepage}

% configurazione emoji
\usepackage{fontspec}
\usepackage{emoji}
\setemojifont{NotoColorEmoji.ttf}[Path=/usr/share/fonts/truetype/noto/]

\newtheorem{definition}{Definizione}
\newtheorem{lemma}{Lemma}
\newtheorem{theorem}{Teorema}
\newtheorem{corollary}{Corollario}

%% cambio nome al comando proof
\renewcommand*{\proofname}{Dimostrazione}

\begin{document}
\documentclass[a4paper,12 pt]{report}
\usepackage[T1]{fontenc}
\usepackage[utf8]{inputenc}
\usepackage{lmodern}
\usepackage{listings}

\usepackage{float}
\usepackage{subcaption}
\usepackage{wrapfig}
\usepackage{fancyhdr}
\usepackage{amsthm}

\usepackage{pgfplots}
\setlength {\marginparwidth }{2cm}
\usepackage{todonotes}
\newcommand{\TODO}[2][]
{\todo[size=\scriptsize, color=red, #1]{#2}}


asdf
\pgfplotsset{compat=1.18}
% forza le footnote a stare il più in basso possibile
\usepackage[bottom]{footmisc}


%% STILE LISTINGS
%%aaa
\usepackage{xcolor}

\definecolor{codegreen}{rgb}{0,0.6,0}
\definecolor{codegray}{rgb}{0.5,0.5,0.5}
\definecolor{codepurple}{rgb}{0.58,0,0.82}
\definecolor{backcolour}{rgb}{0.95,0.95,0.92}

\lstdefinestyle{mystyle}{
    backgroundcolor=\color{backcolour},   
    commentstyle=\color{codegreen},
    keywordstyle=\color{magenta},
    numberstyle=\tiny\color{codegray},
    stringstyle=\color{codepurple},
    basicstyle=\ttfamily\footnotesize,
    breakatwhitespace=false,         
    breaklines=true,                 
    captionpos=b,                    
    keepspaces=true,                 
    numbers=left,                    
    numbersep=5pt,                  
    showspaces=false,                
    showstringspaces=false,
    showtabs=false,                  
    tabsize=2
}

\lstset{style=mystyle}

%% -----

% mostra le subsubsection nell'indice
\setcounter{tocdepth}{3}
\setcounter{secnumdepth}{3}

% Resetta la numerazione dei chapter quando
% una nuova part viene creata
\makeatletter
\@addtoreset{chapter}{part}
\makeatother

% Rimuove l'indentazione quando si crea un nuovo paragrafo
\setlength{\parindent}{0pt}

% footer
\pagestyle{fancyplain}
% rimuove la riga nell'header
\fancyhf{} % sets both header and footer to nothing
\renewcommand{\headrulewidth}{0pt}
\fancyfoot[L]{\href{https://github.com/Typing-Monkeys/AppuntiUniversita}{Typing Monkeys}}
\fancyfoot[C]{\emoji{gorilla}}
\fancyfoot[R]{\thepage}

% configurazione emoji
\usepackage{fontspec}
\usepackage{emoji}
\setemojifont{NotoColorEmoji.ttf}[Path=/usr/share/fonts/truetype/noto/]

\newtheorem{definition}{Definizione}
\newtheorem{lemma}{Lemma}
\newtheorem{theorem}{Teorema}
\newtheorem{corollary}{Corollario}

%% cambio nome al comando proof
\renewcommand*{\proofname}{Dimostrazione}

\begin{document}
\documentclass[a4paper,12 pt]{report}
\usepackage[T1]{fontenc}
\usepackage[utf8]{inputenc}
\usepackage{lmodern}
\usepackage{listings}

\usepackage{float}
\usepackage{subcaption}
\usepackage{wrapfig}
\usepackage{fancyhdr}
\usepackage{amsthm}

\usepackage{pgfplots}
\setlength {\marginparwidth }{2cm}
\usepackage{todonotes}
\newcommand{\TODO}[2][]
{\todo[size=\scriptsize, color=red, #1]{#2}}


asdf
\pgfplotsset{compat=1.18}
% forza le footnote a stare il più in basso possibile
\usepackage[bottom]{footmisc}


%% STILE LISTINGS
%%aaa
\usepackage{xcolor}

\definecolor{codegreen}{rgb}{0,0.6,0}
\definecolor{codegray}{rgb}{0.5,0.5,0.5}
\definecolor{codepurple}{rgb}{0.58,0,0.82}
\definecolor{backcolour}{rgb}{0.95,0.95,0.92}

\lstdefinestyle{mystyle}{
    backgroundcolor=\color{backcolour},   
    commentstyle=\color{codegreen},
    keywordstyle=\color{magenta},
    numberstyle=\tiny\color{codegray},
    stringstyle=\color{codepurple},
    basicstyle=\ttfamily\footnotesize,
    breakatwhitespace=false,         
    breaklines=true,                 
    captionpos=b,                    
    keepspaces=true,                 
    numbers=left,                    
    numbersep=5pt,                  
    showspaces=false,                
    showstringspaces=false,
    showtabs=false,                  
    tabsize=2
}

\lstset{style=mystyle}

%% -----

% mostra le subsubsection nell'indice
\setcounter{tocdepth}{3}
\setcounter{secnumdepth}{3}

% Resetta la numerazione dei chapter quando
% una nuova part viene creata
\makeatletter
\@addtoreset{chapter}{part}
\makeatother

% Rimuove l'indentazione quando si crea un nuovo paragrafo
\setlength{\parindent}{0pt}

% footer
\pagestyle{fancyplain}
% rimuove la riga nell'header
\fancyhf{} % sets both header and footer to nothing
\renewcommand{\headrulewidth}{0pt}
\fancyfoot[L]{\href{https://github.com/Typing-Monkeys/AppuntiUniversita}{Typing Monkeys}}
\fancyfoot[C]{\emoji{gorilla}}
\fancyfoot[R]{\thepage}

% configurazione emoji
\usepackage{fontspec}
\usepackage{emoji}
\setemojifont{NotoColorEmoji.ttf}[Path=/usr/share/fonts/truetype/noto/]

\newtheorem{definition}{Definizione}
\newtheorem{lemma}{Lemma}
\newtheorem{theorem}{Teorema}
\newtheorem{corollary}{Corollario}

%% cambio nome al comando proof
\renewcommand*{\proofname}{Dimostrazione}

\begin{document}
\include{frontmatter/main.tex}

\tableofcontents

\include{quote/main.tex}

%% Aggiungere i capitoli qui sotto
\include{capitoli/richiami/main.tex}
\include{capitoli/immagini/main.tex}

\end{document}


\tableofcontents

\documentclass[a4paper,12 pt]{report}
\usepackage[T1]{fontenc}
\usepackage[utf8]{inputenc}
\usepackage{lmodern}
\usepackage{listings}

\usepackage{float}
\usepackage{subcaption}
\usepackage{wrapfig}
\usepackage{fancyhdr}
\usepackage{amsthm}

\usepackage{pgfplots}
\setlength {\marginparwidth }{2cm}
\usepackage{todonotes}
\newcommand{\TODO}[2][]
{\todo[size=\scriptsize, color=red, #1]{#2}}


asdf
\pgfplotsset{compat=1.18}
% forza le footnote a stare il più in basso possibile
\usepackage[bottom]{footmisc}


%% STILE LISTINGS
%%aaa
\usepackage{xcolor}

\definecolor{codegreen}{rgb}{0,0.6,0}
\definecolor{codegray}{rgb}{0.5,0.5,0.5}
\definecolor{codepurple}{rgb}{0.58,0,0.82}
\definecolor{backcolour}{rgb}{0.95,0.95,0.92}

\lstdefinestyle{mystyle}{
    backgroundcolor=\color{backcolour},   
    commentstyle=\color{codegreen},
    keywordstyle=\color{magenta},
    numberstyle=\tiny\color{codegray},
    stringstyle=\color{codepurple},
    basicstyle=\ttfamily\footnotesize,
    breakatwhitespace=false,         
    breaklines=true,                 
    captionpos=b,                    
    keepspaces=true,                 
    numbers=left,                    
    numbersep=5pt,                  
    showspaces=false,                
    showstringspaces=false,
    showtabs=false,                  
    tabsize=2
}

\lstset{style=mystyle}

%% -----

% mostra le subsubsection nell'indice
\setcounter{tocdepth}{3}
\setcounter{secnumdepth}{3}

% Resetta la numerazione dei chapter quando
% una nuova part viene creata
\makeatletter
\@addtoreset{chapter}{part}
\makeatother

% Rimuove l'indentazione quando si crea un nuovo paragrafo
\setlength{\parindent}{0pt}

% footer
\pagestyle{fancyplain}
% rimuove la riga nell'header
\fancyhf{} % sets both header and footer to nothing
\renewcommand{\headrulewidth}{0pt}
\fancyfoot[L]{\href{https://github.com/Typing-Monkeys/AppuntiUniversita}{Typing Monkeys}}
\fancyfoot[C]{\emoji{gorilla}}
\fancyfoot[R]{\thepage}

% configurazione emoji
\usepackage{fontspec}
\usepackage{emoji}
\setemojifont{NotoColorEmoji.ttf}[Path=/usr/share/fonts/truetype/noto/]

\newtheorem{definition}{Definizione}
\newtheorem{lemma}{Lemma}
\newtheorem{theorem}{Teorema}
\newtheorem{corollary}{Corollario}

%% cambio nome al comando proof
\renewcommand*{\proofname}{Dimostrazione}

\begin{document}
\include{frontmatter/main.tex}

\tableofcontents

\include{quote/main.tex}

%% Aggiungere i capitoli qui sotto
\include{capitoli/richiami/main.tex}
\include{capitoli/immagini/main.tex}

\end{document}


%% Aggiungere i capitoli qui sotto
\documentclass[a4paper,12 pt]{report}
\usepackage[T1]{fontenc}
\usepackage[utf8]{inputenc}
\usepackage{lmodern}
\usepackage{listings}

\usepackage{float}
\usepackage{subcaption}
\usepackage{wrapfig}
\usepackage{fancyhdr}
\usepackage{amsthm}

\usepackage{pgfplots}
\setlength {\marginparwidth }{2cm}
\usepackage{todonotes}
\newcommand{\TODO}[2][]
{\todo[size=\scriptsize, color=red, #1]{#2}}


asdf
\pgfplotsset{compat=1.18}
% forza le footnote a stare il più in basso possibile
\usepackage[bottom]{footmisc}


%% STILE LISTINGS
%%aaa
\usepackage{xcolor}

\definecolor{codegreen}{rgb}{0,0.6,0}
\definecolor{codegray}{rgb}{0.5,0.5,0.5}
\definecolor{codepurple}{rgb}{0.58,0,0.82}
\definecolor{backcolour}{rgb}{0.95,0.95,0.92}

\lstdefinestyle{mystyle}{
    backgroundcolor=\color{backcolour},   
    commentstyle=\color{codegreen},
    keywordstyle=\color{magenta},
    numberstyle=\tiny\color{codegray},
    stringstyle=\color{codepurple},
    basicstyle=\ttfamily\footnotesize,
    breakatwhitespace=false,         
    breaklines=true,                 
    captionpos=b,                    
    keepspaces=true,                 
    numbers=left,                    
    numbersep=5pt,                  
    showspaces=false,                
    showstringspaces=false,
    showtabs=false,                  
    tabsize=2
}

\lstset{style=mystyle}

%% -----

% mostra le subsubsection nell'indice
\setcounter{tocdepth}{3}
\setcounter{secnumdepth}{3}

% Resetta la numerazione dei chapter quando
% una nuova part viene creata
\makeatletter
\@addtoreset{chapter}{part}
\makeatother

% Rimuove l'indentazione quando si crea un nuovo paragrafo
\setlength{\parindent}{0pt}

% footer
\pagestyle{fancyplain}
% rimuove la riga nell'header
\fancyhf{} % sets both header and footer to nothing
\renewcommand{\headrulewidth}{0pt}
\fancyfoot[L]{\href{https://github.com/Typing-Monkeys/AppuntiUniversita}{Typing Monkeys}}
\fancyfoot[C]{\emoji{gorilla}}
\fancyfoot[R]{\thepage}

% configurazione emoji
\usepackage{fontspec}
\usepackage{emoji}
\setemojifont{NotoColorEmoji.ttf}[Path=/usr/share/fonts/truetype/noto/]

\newtheorem{definition}{Definizione}
\newtheorem{lemma}{Lemma}
\newtheorem{theorem}{Teorema}
\newtheorem{corollary}{Corollario}

%% cambio nome al comando proof
\renewcommand*{\proofname}{Dimostrazione}

\begin{document}
\include{frontmatter/main.tex}

\tableofcontents

\include{quote/main.tex}

%% Aggiungere i capitoli qui sotto
\include{capitoli/richiami/main.tex}
\include{capitoli/immagini/main.tex}

\end{document}

\documentclass[a4paper,12 pt]{report}
\usepackage[T1]{fontenc}
\usepackage[utf8]{inputenc}
\usepackage{lmodern}
\usepackage{listings}

\usepackage{float}
\usepackage{subcaption}
\usepackage{wrapfig}
\usepackage{fancyhdr}
\usepackage{amsthm}

\usepackage{pgfplots}
\setlength {\marginparwidth }{2cm}
\usepackage{todonotes}
\newcommand{\TODO}[2][]
{\todo[size=\scriptsize, color=red, #1]{#2}}


asdf
\pgfplotsset{compat=1.18}
% forza le footnote a stare il più in basso possibile
\usepackage[bottom]{footmisc}


%% STILE LISTINGS
%%aaa
\usepackage{xcolor}

\definecolor{codegreen}{rgb}{0,0.6,0}
\definecolor{codegray}{rgb}{0.5,0.5,0.5}
\definecolor{codepurple}{rgb}{0.58,0,0.82}
\definecolor{backcolour}{rgb}{0.95,0.95,0.92}

\lstdefinestyle{mystyle}{
    backgroundcolor=\color{backcolour},   
    commentstyle=\color{codegreen},
    keywordstyle=\color{magenta},
    numberstyle=\tiny\color{codegray},
    stringstyle=\color{codepurple},
    basicstyle=\ttfamily\footnotesize,
    breakatwhitespace=false,         
    breaklines=true,                 
    captionpos=b,                    
    keepspaces=true,                 
    numbers=left,                    
    numbersep=5pt,                  
    showspaces=false,                
    showstringspaces=false,
    showtabs=false,                  
    tabsize=2
}

\lstset{style=mystyle}

%% -----

% mostra le subsubsection nell'indice
\setcounter{tocdepth}{3}
\setcounter{secnumdepth}{3}

% Resetta la numerazione dei chapter quando
% una nuova part viene creata
\makeatletter
\@addtoreset{chapter}{part}
\makeatother

% Rimuove l'indentazione quando si crea un nuovo paragrafo
\setlength{\parindent}{0pt}

% footer
\pagestyle{fancyplain}
% rimuove la riga nell'header
\fancyhf{} % sets both header and footer to nothing
\renewcommand{\headrulewidth}{0pt}
\fancyfoot[L]{\href{https://github.com/Typing-Monkeys/AppuntiUniversita}{Typing Monkeys}}
\fancyfoot[C]{\emoji{gorilla}}
\fancyfoot[R]{\thepage}

% configurazione emoji
\usepackage{fontspec}
\usepackage{emoji}
\setemojifont{NotoColorEmoji.ttf}[Path=/usr/share/fonts/truetype/noto/]

\newtheorem{definition}{Definizione}
\newtheorem{lemma}{Lemma}
\newtheorem{theorem}{Teorema}
\newtheorem{corollary}{Corollario}

%% cambio nome al comando proof
\renewcommand*{\proofname}{Dimostrazione}

\begin{document}
\include{frontmatter/main.tex}

\tableofcontents

\include{quote/main.tex}

%% Aggiungere i capitoli qui sotto
\include{capitoli/richiami/main.tex}
\include{capitoli/immagini/main.tex}

\end{document}


\end{document}


\tableofcontents

\documentclass[a4paper,12 pt]{report}
\usepackage[T1]{fontenc}
\usepackage[utf8]{inputenc}
\usepackage{lmodern}
\usepackage{listings}

\usepackage{float}
\usepackage{subcaption}
\usepackage{wrapfig}
\usepackage{fancyhdr}
\usepackage{amsthm}

\usepackage{pgfplots}
\setlength {\marginparwidth }{2cm}
\usepackage{todonotes}
\newcommand{\TODO}[2][]
{\todo[size=\scriptsize, color=red, #1]{#2}}


asdf
\pgfplotsset{compat=1.18}
% forza le footnote a stare il più in basso possibile
\usepackage[bottom]{footmisc}


%% STILE LISTINGS
%%aaa
\usepackage{xcolor}

\definecolor{codegreen}{rgb}{0,0.6,0}
\definecolor{codegray}{rgb}{0.5,0.5,0.5}
\definecolor{codepurple}{rgb}{0.58,0,0.82}
\definecolor{backcolour}{rgb}{0.95,0.95,0.92}

\lstdefinestyle{mystyle}{
    backgroundcolor=\color{backcolour},   
    commentstyle=\color{codegreen},
    keywordstyle=\color{magenta},
    numberstyle=\tiny\color{codegray},
    stringstyle=\color{codepurple},
    basicstyle=\ttfamily\footnotesize,
    breakatwhitespace=false,         
    breaklines=true,                 
    captionpos=b,                    
    keepspaces=true,                 
    numbers=left,                    
    numbersep=5pt,                  
    showspaces=false,                
    showstringspaces=false,
    showtabs=false,                  
    tabsize=2
}

\lstset{style=mystyle}

%% -----

% mostra le subsubsection nell'indice
\setcounter{tocdepth}{3}
\setcounter{secnumdepth}{3}

% Resetta la numerazione dei chapter quando
% una nuova part viene creata
\makeatletter
\@addtoreset{chapter}{part}
\makeatother

% Rimuove l'indentazione quando si crea un nuovo paragrafo
\setlength{\parindent}{0pt}

% footer
\pagestyle{fancyplain}
% rimuove la riga nell'header
\fancyhf{} % sets both header and footer to nothing
\renewcommand{\headrulewidth}{0pt}
\fancyfoot[L]{\href{https://github.com/Typing-Monkeys/AppuntiUniversita}{Typing Monkeys}}
\fancyfoot[C]{\emoji{gorilla}}
\fancyfoot[R]{\thepage}

% configurazione emoji
\usepackage{fontspec}
\usepackage{emoji}
\setemojifont{NotoColorEmoji.ttf}[Path=/usr/share/fonts/truetype/noto/]

\newtheorem{definition}{Definizione}
\newtheorem{lemma}{Lemma}
\newtheorem{theorem}{Teorema}
\newtheorem{corollary}{Corollario}

%% cambio nome al comando proof
\renewcommand*{\proofname}{Dimostrazione}

\begin{document}
\documentclass[a4paper,12 pt]{report}
\usepackage[T1]{fontenc}
\usepackage[utf8]{inputenc}
\usepackage{lmodern}
\usepackage{listings}

\usepackage{float}
\usepackage{subcaption}
\usepackage{wrapfig}
\usepackage{fancyhdr}
\usepackage{amsthm}

\usepackage{pgfplots}
\setlength {\marginparwidth }{2cm}
\usepackage{todonotes}
\newcommand{\TODO}[2][]
{\todo[size=\scriptsize, color=red, #1]{#2}}


asdf
\pgfplotsset{compat=1.18}
% forza le footnote a stare il più in basso possibile
\usepackage[bottom]{footmisc}


%% STILE LISTINGS
%%aaa
\usepackage{xcolor}

\definecolor{codegreen}{rgb}{0,0.6,0}
\definecolor{codegray}{rgb}{0.5,0.5,0.5}
\definecolor{codepurple}{rgb}{0.58,0,0.82}
\definecolor{backcolour}{rgb}{0.95,0.95,0.92}

\lstdefinestyle{mystyle}{
    backgroundcolor=\color{backcolour},   
    commentstyle=\color{codegreen},
    keywordstyle=\color{magenta},
    numberstyle=\tiny\color{codegray},
    stringstyle=\color{codepurple},
    basicstyle=\ttfamily\footnotesize,
    breakatwhitespace=false,         
    breaklines=true,                 
    captionpos=b,                    
    keepspaces=true,                 
    numbers=left,                    
    numbersep=5pt,                  
    showspaces=false,                
    showstringspaces=false,
    showtabs=false,                  
    tabsize=2
}

\lstset{style=mystyle}

%% -----

% mostra le subsubsection nell'indice
\setcounter{tocdepth}{3}
\setcounter{secnumdepth}{3}

% Resetta la numerazione dei chapter quando
% una nuova part viene creata
\makeatletter
\@addtoreset{chapter}{part}
\makeatother

% Rimuove l'indentazione quando si crea un nuovo paragrafo
\setlength{\parindent}{0pt}

% footer
\pagestyle{fancyplain}
% rimuove la riga nell'header
\fancyhf{} % sets both header and footer to nothing
\renewcommand{\headrulewidth}{0pt}
\fancyfoot[L]{\href{https://github.com/Typing-Monkeys/AppuntiUniversita}{Typing Monkeys}}
\fancyfoot[C]{\emoji{gorilla}}
\fancyfoot[R]{\thepage}

% configurazione emoji
\usepackage{fontspec}
\usepackage{emoji}
\setemojifont{NotoColorEmoji.ttf}[Path=/usr/share/fonts/truetype/noto/]

\newtheorem{definition}{Definizione}
\newtheorem{lemma}{Lemma}
\newtheorem{theorem}{Teorema}
\newtheorem{corollary}{Corollario}

%% cambio nome al comando proof
\renewcommand*{\proofname}{Dimostrazione}

\begin{document}
\include{frontmatter/main.tex}

\tableofcontents

\include{quote/main.tex}

%% Aggiungere i capitoli qui sotto
\include{capitoli/richiami/main.tex}
\include{capitoli/immagini/main.tex}

\end{document}


\tableofcontents

\documentclass[a4paper,12 pt]{report}
\usepackage[T1]{fontenc}
\usepackage[utf8]{inputenc}
\usepackage{lmodern}
\usepackage{listings}

\usepackage{float}
\usepackage{subcaption}
\usepackage{wrapfig}
\usepackage{fancyhdr}
\usepackage{amsthm}

\usepackage{pgfplots}
\setlength {\marginparwidth }{2cm}
\usepackage{todonotes}
\newcommand{\TODO}[2][]
{\todo[size=\scriptsize, color=red, #1]{#2}}


asdf
\pgfplotsset{compat=1.18}
% forza le footnote a stare il più in basso possibile
\usepackage[bottom]{footmisc}


%% STILE LISTINGS
%%aaa
\usepackage{xcolor}

\definecolor{codegreen}{rgb}{0,0.6,0}
\definecolor{codegray}{rgb}{0.5,0.5,0.5}
\definecolor{codepurple}{rgb}{0.58,0,0.82}
\definecolor{backcolour}{rgb}{0.95,0.95,0.92}

\lstdefinestyle{mystyle}{
    backgroundcolor=\color{backcolour},   
    commentstyle=\color{codegreen},
    keywordstyle=\color{magenta},
    numberstyle=\tiny\color{codegray},
    stringstyle=\color{codepurple},
    basicstyle=\ttfamily\footnotesize,
    breakatwhitespace=false,         
    breaklines=true,                 
    captionpos=b,                    
    keepspaces=true,                 
    numbers=left,                    
    numbersep=5pt,                  
    showspaces=false,                
    showstringspaces=false,
    showtabs=false,                  
    tabsize=2
}

\lstset{style=mystyle}

%% -----

% mostra le subsubsection nell'indice
\setcounter{tocdepth}{3}
\setcounter{secnumdepth}{3}

% Resetta la numerazione dei chapter quando
% una nuova part viene creata
\makeatletter
\@addtoreset{chapter}{part}
\makeatother

% Rimuove l'indentazione quando si crea un nuovo paragrafo
\setlength{\parindent}{0pt}

% footer
\pagestyle{fancyplain}
% rimuove la riga nell'header
\fancyhf{} % sets both header and footer to nothing
\renewcommand{\headrulewidth}{0pt}
\fancyfoot[L]{\href{https://github.com/Typing-Monkeys/AppuntiUniversita}{Typing Monkeys}}
\fancyfoot[C]{\emoji{gorilla}}
\fancyfoot[R]{\thepage}

% configurazione emoji
\usepackage{fontspec}
\usepackage{emoji}
\setemojifont{NotoColorEmoji.ttf}[Path=/usr/share/fonts/truetype/noto/]

\newtheorem{definition}{Definizione}
\newtheorem{lemma}{Lemma}
\newtheorem{theorem}{Teorema}
\newtheorem{corollary}{Corollario}

%% cambio nome al comando proof
\renewcommand*{\proofname}{Dimostrazione}

\begin{document}
\include{frontmatter/main.tex}

\tableofcontents

\include{quote/main.tex}

%% Aggiungere i capitoli qui sotto
\include{capitoli/richiami/main.tex}
\include{capitoli/immagini/main.tex}

\end{document}


%% Aggiungere i capitoli qui sotto
\documentclass[a4paper,12 pt]{report}
\usepackage[T1]{fontenc}
\usepackage[utf8]{inputenc}
\usepackage{lmodern}
\usepackage{listings}

\usepackage{float}
\usepackage{subcaption}
\usepackage{wrapfig}
\usepackage{fancyhdr}
\usepackage{amsthm}

\usepackage{pgfplots}
\setlength {\marginparwidth }{2cm}
\usepackage{todonotes}
\newcommand{\TODO}[2][]
{\todo[size=\scriptsize, color=red, #1]{#2}}


asdf
\pgfplotsset{compat=1.18}
% forza le footnote a stare il più in basso possibile
\usepackage[bottom]{footmisc}


%% STILE LISTINGS
%%aaa
\usepackage{xcolor}

\definecolor{codegreen}{rgb}{0,0.6,0}
\definecolor{codegray}{rgb}{0.5,0.5,0.5}
\definecolor{codepurple}{rgb}{0.58,0,0.82}
\definecolor{backcolour}{rgb}{0.95,0.95,0.92}

\lstdefinestyle{mystyle}{
    backgroundcolor=\color{backcolour},   
    commentstyle=\color{codegreen},
    keywordstyle=\color{magenta},
    numberstyle=\tiny\color{codegray},
    stringstyle=\color{codepurple},
    basicstyle=\ttfamily\footnotesize,
    breakatwhitespace=false,         
    breaklines=true,                 
    captionpos=b,                    
    keepspaces=true,                 
    numbers=left,                    
    numbersep=5pt,                  
    showspaces=false,                
    showstringspaces=false,
    showtabs=false,                  
    tabsize=2
}

\lstset{style=mystyle}

%% -----

% mostra le subsubsection nell'indice
\setcounter{tocdepth}{3}
\setcounter{secnumdepth}{3}

% Resetta la numerazione dei chapter quando
% una nuova part viene creata
\makeatletter
\@addtoreset{chapter}{part}
\makeatother

% Rimuove l'indentazione quando si crea un nuovo paragrafo
\setlength{\parindent}{0pt}

% footer
\pagestyle{fancyplain}
% rimuove la riga nell'header
\fancyhf{} % sets both header and footer to nothing
\renewcommand{\headrulewidth}{0pt}
\fancyfoot[L]{\href{https://github.com/Typing-Monkeys/AppuntiUniversita}{Typing Monkeys}}
\fancyfoot[C]{\emoji{gorilla}}
\fancyfoot[R]{\thepage}

% configurazione emoji
\usepackage{fontspec}
\usepackage{emoji}
\setemojifont{NotoColorEmoji.ttf}[Path=/usr/share/fonts/truetype/noto/]

\newtheorem{definition}{Definizione}
\newtheorem{lemma}{Lemma}
\newtheorem{theorem}{Teorema}
\newtheorem{corollary}{Corollario}

%% cambio nome al comando proof
\renewcommand*{\proofname}{Dimostrazione}

\begin{document}
\include{frontmatter/main.tex}

\tableofcontents

\include{quote/main.tex}

%% Aggiungere i capitoli qui sotto
\include{capitoli/richiami/main.tex}
\include{capitoli/immagini/main.tex}

\end{document}

\documentclass[a4paper,12 pt]{report}
\usepackage[T1]{fontenc}
\usepackage[utf8]{inputenc}
\usepackage{lmodern}
\usepackage{listings}

\usepackage{float}
\usepackage{subcaption}
\usepackage{wrapfig}
\usepackage{fancyhdr}
\usepackage{amsthm}

\usepackage{pgfplots}
\setlength {\marginparwidth }{2cm}
\usepackage{todonotes}
\newcommand{\TODO}[2][]
{\todo[size=\scriptsize, color=red, #1]{#2}}


asdf
\pgfplotsset{compat=1.18}
% forza le footnote a stare il più in basso possibile
\usepackage[bottom]{footmisc}


%% STILE LISTINGS
%%aaa
\usepackage{xcolor}

\definecolor{codegreen}{rgb}{0,0.6,0}
\definecolor{codegray}{rgb}{0.5,0.5,0.5}
\definecolor{codepurple}{rgb}{0.58,0,0.82}
\definecolor{backcolour}{rgb}{0.95,0.95,0.92}

\lstdefinestyle{mystyle}{
    backgroundcolor=\color{backcolour},   
    commentstyle=\color{codegreen},
    keywordstyle=\color{magenta},
    numberstyle=\tiny\color{codegray},
    stringstyle=\color{codepurple},
    basicstyle=\ttfamily\footnotesize,
    breakatwhitespace=false,         
    breaklines=true,                 
    captionpos=b,                    
    keepspaces=true,                 
    numbers=left,                    
    numbersep=5pt,                  
    showspaces=false,                
    showstringspaces=false,
    showtabs=false,                  
    tabsize=2
}

\lstset{style=mystyle}

%% -----

% mostra le subsubsection nell'indice
\setcounter{tocdepth}{3}
\setcounter{secnumdepth}{3}

% Resetta la numerazione dei chapter quando
% una nuova part viene creata
\makeatletter
\@addtoreset{chapter}{part}
\makeatother

% Rimuove l'indentazione quando si crea un nuovo paragrafo
\setlength{\parindent}{0pt}

% footer
\pagestyle{fancyplain}
% rimuove la riga nell'header
\fancyhf{} % sets both header and footer to nothing
\renewcommand{\headrulewidth}{0pt}
\fancyfoot[L]{\href{https://github.com/Typing-Monkeys/AppuntiUniversita}{Typing Monkeys}}
\fancyfoot[C]{\emoji{gorilla}}
\fancyfoot[R]{\thepage}

% configurazione emoji
\usepackage{fontspec}
\usepackage{emoji}
\setemojifont{NotoColorEmoji.ttf}[Path=/usr/share/fonts/truetype/noto/]

\newtheorem{definition}{Definizione}
\newtheorem{lemma}{Lemma}
\newtheorem{theorem}{Teorema}
\newtheorem{corollary}{Corollario}

%% cambio nome al comando proof
\renewcommand*{\proofname}{Dimostrazione}

\begin{document}
\include{frontmatter/main.tex}

\tableofcontents

\include{quote/main.tex}

%% Aggiungere i capitoli qui sotto
\include{capitoli/richiami/main.tex}
\include{capitoli/immagini/main.tex}

\end{document}


\end{document}


%% Aggiungere i capitoli qui sotto
\documentclass[a4paper,12 pt]{report}
\usepackage[T1]{fontenc}
\usepackage[utf8]{inputenc}
\usepackage{lmodern}
\usepackage{listings}

\usepackage{float}
\usepackage{subcaption}
\usepackage{wrapfig}
\usepackage{fancyhdr}
\usepackage{amsthm}

\usepackage{pgfplots}
\setlength {\marginparwidth }{2cm}
\usepackage{todonotes}
\newcommand{\TODO}[2][]
{\todo[size=\scriptsize, color=red, #1]{#2}}


asdf
\pgfplotsset{compat=1.18}
% forza le footnote a stare il più in basso possibile
\usepackage[bottom]{footmisc}


%% STILE LISTINGS
%%aaa
\usepackage{xcolor}

\definecolor{codegreen}{rgb}{0,0.6,0}
\definecolor{codegray}{rgb}{0.5,0.5,0.5}
\definecolor{codepurple}{rgb}{0.58,0,0.82}
\definecolor{backcolour}{rgb}{0.95,0.95,0.92}

\lstdefinestyle{mystyle}{
    backgroundcolor=\color{backcolour},   
    commentstyle=\color{codegreen},
    keywordstyle=\color{magenta},
    numberstyle=\tiny\color{codegray},
    stringstyle=\color{codepurple},
    basicstyle=\ttfamily\footnotesize,
    breakatwhitespace=false,         
    breaklines=true,                 
    captionpos=b,                    
    keepspaces=true,                 
    numbers=left,                    
    numbersep=5pt,                  
    showspaces=false,                
    showstringspaces=false,
    showtabs=false,                  
    tabsize=2
}

\lstset{style=mystyle}

%% -----

% mostra le subsubsection nell'indice
\setcounter{tocdepth}{3}
\setcounter{secnumdepth}{3}

% Resetta la numerazione dei chapter quando
% una nuova part viene creata
\makeatletter
\@addtoreset{chapter}{part}
\makeatother

% Rimuove l'indentazione quando si crea un nuovo paragrafo
\setlength{\parindent}{0pt}

% footer
\pagestyle{fancyplain}
% rimuove la riga nell'header
\fancyhf{} % sets both header and footer to nothing
\renewcommand{\headrulewidth}{0pt}
\fancyfoot[L]{\href{https://github.com/Typing-Monkeys/AppuntiUniversita}{Typing Monkeys}}
\fancyfoot[C]{\emoji{gorilla}}
\fancyfoot[R]{\thepage}

% configurazione emoji
\usepackage{fontspec}
\usepackage{emoji}
\setemojifont{NotoColorEmoji.ttf}[Path=/usr/share/fonts/truetype/noto/]

\newtheorem{definition}{Definizione}
\newtheorem{lemma}{Lemma}
\newtheorem{theorem}{Teorema}
\newtheorem{corollary}{Corollario}

%% cambio nome al comando proof
\renewcommand*{\proofname}{Dimostrazione}

\begin{document}
\documentclass[a4paper,12 pt]{report}
\usepackage[T1]{fontenc}
\usepackage[utf8]{inputenc}
\usepackage{lmodern}
\usepackage{listings}

\usepackage{float}
\usepackage{subcaption}
\usepackage{wrapfig}
\usepackage{fancyhdr}
\usepackage{amsthm}

\usepackage{pgfplots}
\setlength {\marginparwidth }{2cm}
\usepackage{todonotes}
\newcommand{\TODO}[2][]
{\todo[size=\scriptsize, color=red, #1]{#2}}


asdf
\pgfplotsset{compat=1.18}
% forza le footnote a stare il più in basso possibile
\usepackage[bottom]{footmisc}


%% STILE LISTINGS
%%aaa
\usepackage{xcolor}

\definecolor{codegreen}{rgb}{0,0.6,0}
\definecolor{codegray}{rgb}{0.5,0.5,0.5}
\definecolor{codepurple}{rgb}{0.58,0,0.82}
\definecolor{backcolour}{rgb}{0.95,0.95,0.92}

\lstdefinestyle{mystyle}{
    backgroundcolor=\color{backcolour},   
    commentstyle=\color{codegreen},
    keywordstyle=\color{magenta},
    numberstyle=\tiny\color{codegray},
    stringstyle=\color{codepurple},
    basicstyle=\ttfamily\footnotesize,
    breakatwhitespace=false,         
    breaklines=true,                 
    captionpos=b,                    
    keepspaces=true,                 
    numbers=left,                    
    numbersep=5pt,                  
    showspaces=false,                
    showstringspaces=false,
    showtabs=false,                  
    tabsize=2
}

\lstset{style=mystyle}

%% -----

% mostra le subsubsection nell'indice
\setcounter{tocdepth}{3}
\setcounter{secnumdepth}{3}

% Resetta la numerazione dei chapter quando
% una nuova part viene creata
\makeatletter
\@addtoreset{chapter}{part}
\makeatother

% Rimuove l'indentazione quando si crea un nuovo paragrafo
\setlength{\parindent}{0pt}

% footer
\pagestyle{fancyplain}
% rimuove la riga nell'header
\fancyhf{} % sets both header and footer to nothing
\renewcommand{\headrulewidth}{0pt}
\fancyfoot[L]{\href{https://github.com/Typing-Monkeys/AppuntiUniversita}{Typing Monkeys}}
\fancyfoot[C]{\emoji{gorilla}}
\fancyfoot[R]{\thepage}

% configurazione emoji
\usepackage{fontspec}
\usepackage{emoji}
\setemojifont{NotoColorEmoji.ttf}[Path=/usr/share/fonts/truetype/noto/]

\newtheorem{definition}{Definizione}
\newtheorem{lemma}{Lemma}
\newtheorem{theorem}{Teorema}
\newtheorem{corollary}{Corollario}

%% cambio nome al comando proof
\renewcommand*{\proofname}{Dimostrazione}

\begin{document}
\include{frontmatter/main.tex}

\tableofcontents

\include{quote/main.tex}

%% Aggiungere i capitoli qui sotto
\include{capitoli/richiami/main.tex}
\include{capitoli/immagini/main.tex}

\end{document}


\tableofcontents

\documentclass[a4paper,12 pt]{report}
\usepackage[T1]{fontenc}
\usepackage[utf8]{inputenc}
\usepackage{lmodern}
\usepackage{listings}

\usepackage{float}
\usepackage{subcaption}
\usepackage{wrapfig}
\usepackage{fancyhdr}
\usepackage{amsthm}

\usepackage{pgfplots}
\setlength {\marginparwidth }{2cm}
\usepackage{todonotes}
\newcommand{\TODO}[2][]
{\todo[size=\scriptsize, color=red, #1]{#2}}


asdf
\pgfplotsset{compat=1.18}
% forza le footnote a stare il più in basso possibile
\usepackage[bottom]{footmisc}


%% STILE LISTINGS
%%aaa
\usepackage{xcolor}

\definecolor{codegreen}{rgb}{0,0.6,0}
\definecolor{codegray}{rgb}{0.5,0.5,0.5}
\definecolor{codepurple}{rgb}{0.58,0,0.82}
\definecolor{backcolour}{rgb}{0.95,0.95,0.92}

\lstdefinestyle{mystyle}{
    backgroundcolor=\color{backcolour},   
    commentstyle=\color{codegreen},
    keywordstyle=\color{magenta},
    numberstyle=\tiny\color{codegray},
    stringstyle=\color{codepurple},
    basicstyle=\ttfamily\footnotesize,
    breakatwhitespace=false,         
    breaklines=true,                 
    captionpos=b,                    
    keepspaces=true,                 
    numbers=left,                    
    numbersep=5pt,                  
    showspaces=false,                
    showstringspaces=false,
    showtabs=false,                  
    tabsize=2
}

\lstset{style=mystyle}

%% -----

% mostra le subsubsection nell'indice
\setcounter{tocdepth}{3}
\setcounter{secnumdepth}{3}

% Resetta la numerazione dei chapter quando
% una nuova part viene creata
\makeatletter
\@addtoreset{chapter}{part}
\makeatother

% Rimuove l'indentazione quando si crea un nuovo paragrafo
\setlength{\parindent}{0pt}

% footer
\pagestyle{fancyplain}
% rimuove la riga nell'header
\fancyhf{} % sets both header and footer to nothing
\renewcommand{\headrulewidth}{0pt}
\fancyfoot[L]{\href{https://github.com/Typing-Monkeys/AppuntiUniversita}{Typing Monkeys}}
\fancyfoot[C]{\emoji{gorilla}}
\fancyfoot[R]{\thepage}

% configurazione emoji
\usepackage{fontspec}
\usepackage{emoji}
\setemojifont{NotoColorEmoji.ttf}[Path=/usr/share/fonts/truetype/noto/]

\newtheorem{definition}{Definizione}
\newtheorem{lemma}{Lemma}
\newtheorem{theorem}{Teorema}
\newtheorem{corollary}{Corollario}

%% cambio nome al comando proof
\renewcommand*{\proofname}{Dimostrazione}

\begin{document}
\include{frontmatter/main.tex}

\tableofcontents

\include{quote/main.tex}

%% Aggiungere i capitoli qui sotto
\include{capitoli/richiami/main.tex}
\include{capitoli/immagini/main.tex}

\end{document}


%% Aggiungere i capitoli qui sotto
\documentclass[a4paper,12 pt]{report}
\usepackage[T1]{fontenc}
\usepackage[utf8]{inputenc}
\usepackage{lmodern}
\usepackage{listings}

\usepackage{float}
\usepackage{subcaption}
\usepackage{wrapfig}
\usepackage{fancyhdr}
\usepackage{amsthm}

\usepackage{pgfplots}
\setlength {\marginparwidth }{2cm}
\usepackage{todonotes}
\newcommand{\TODO}[2][]
{\todo[size=\scriptsize, color=red, #1]{#2}}


asdf
\pgfplotsset{compat=1.18}
% forza le footnote a stare il più in basso possibile
\usepackage[bottom]{footmisc}


%% STILE LISTINGS
%%aaa
\usepackage{xcolor}

\definecolor{codegreen}{rgb}{0,0.6,0}
\definecolor{codegray}{rgb}{0.5,0.5,0.5}
\definecolor{codepurple}{rgb}{0.58,0,0.82}
\definecolor{backcolour}{rgb}{0.95,0.95,0.92}

\lstdefinestyle{mystyle}{
    backgroundcolor=\color{backcolour},   
    commentstyle=\color{codegreen},
    keywordstyle=\color{magenta},
    numberstyle=\tiny\color{codegray},
    stringstyle=\color{codepurple},
    basicstyle=\ttfamily\footnotesize,
    breakatwhitespace=false,         
    breaklines=true,                 
    captionpos=b,                    
    keepspaces=true,                 
    numbers=left,                    
    numbersep=5pt,                  
    showspaces=false,                
    showstringspaces=false,
    showtabs=false,                  
    tabsize=2
}

\lstset{style=mystyle}

%% -----

% mostra le subsubsection nell'indice
\setcounter{tocdepth}{3}
\setcounter{secnumdepth}{3}

% Resetta la numerazione dei chapter quando
% una nuova part viene creata
\makeatletter
\@addtoreset{chapter}{part}
\makeatother

% Rimuove l'indentazione quando si crea un nuovo paragrafo
\setlength{\parindent}{0pt}

% footer
\pagestyle{fancyplain}
% rimuove la riga nell'header
\fancyhf{} % sets both header and footer to nothing
\renewcommand{\headrulewidth}{0pt}
\fancyfoot[L]{\href{https://github.com/Typing-Monkeys/AppuntiUniversita}{Typing Monkeys}}
\fancyfoot[C]{\emoji{gorilla}}
\fancyfoot[R]{\thepage}

% configurazione emoji
\usepackage{fontspec}
\usepackage{emoji}
\setemojifont{NotoColorEmoji.ttf}[Path=/usr/share/fonts/truetype/noto/]

\newtheorem{definition}{Definizione}
\newtheorem{lemma}{Lemma}
\newtheorem{theorem}{Teorema}
\newtheorem{corollary}{Corollario}

%% cambio nome al comando proof
\renewcommand*{\proofname}{Dimostrazione}

\begin{document}
\include{frontmatter/main.tex}

\tableofcontents

\include{quote/main.tex}

%% Aggiungere i capitoli qui sotto
\include{capitoli/richiami/main.tex}
\include{capitoli/immagini/main.tex}

\end{document}

\documentclass[a4paper,12 pt]{report}
\usepackage[T1]{fontenc}
\usepackage[utf8]{inputenc}
\usepackage{lmodern}
\usepackage{listings}

\usepackage{float}
\usepackage{subcaption}
\usepackage{wrapfig}
\usepackage{fancyhdr}
\usepackage{amsthm}

\usepackage{pgfplots}
\setlength {\marginparwidth }{2cm}
\usepackage{todonotes}
\newcommand{\TODO}[2][]
{\todo[size=\scriptsize, color=red, #1]{#2}}


asdf
\pgfplotsset{compat=1.18}
% forza le footnote a stare il più in basso possibile
\usepackage[bottom]{footmisc}


%% STILE LISTINGS
%%aaa
\usepackage{xcolor}

\definecolor{codegreen}{rgb}{0,0.6,0}
\definecolor{codegray}{rgb}{0.5,0.5,0.5}
\definecolor{codepurple}{rgb}{0.58,0,0.82}
\definecolor{backcolour}{rgb}{0.95,0.95,0.92}

\lstdefinestyle{mystyle}{
    backgroundcolor=\color{backcolour},   
    commentstyle=\color{codegreen},
    keywordstyle=\color{magenta},
    numberstyle=\tiny\color{codegray},
    stringstyle=\color{codepurple},
    basicstyle=\ttfamily\footnotesize,
    breakatwhitespace=false,         
    breaklines=true,                 
    captionpos=b,                    
    keepspaces=true,                 
    numbers=left,                    
    numbersep=5pt,                  
    showspaces=false,                
    showstringspaces=false,
    showtabs=false,                  
    tabsize=2
}

\lstset{style=mystyle}

%% -----

% mostra le subsubsection nell'indice
\setcounter{tocdepth}{3}
\setcounter{secnumdepth}{3}

% Resetta la numerazione dei chapter quando
% una nuova part viene creata
\makeatletter
\@addtoreset{chapter}{part}
\makeatother

% Rimuove l'indentazione quando si crea un nuovo paragrafo
\setlength{\parindent}{0pt}

% footer
\pagestyle{fancyplain}
% rimuove la riga nell'header
\fancyhf{} % sets both header and footer to nothing
\renewcommand{\headrulewidth}{0pt}
\fancyfoot[L]{\href{https://github.com/Typing-Monkeys/AppuntiUniversita}{Typing Monkeys}}
\fancyfoot[C]{\emoji{gorilla}}
\fancyfoot[R]{\thepage}

% configurazione emoji
\usepackage{fontspec}
\usepackage{emoji}
\setemojifont{NotoColorEmoji.ttf}[Path=/usr/share/fonts/truetype/noto/]

\newtheorem{definition}{Definizione}
\newtheorem{lemma}{Lemma}
\newtheorem{theorem}{Teorema}
\newtheorem{corollary}{Corollario}

%% cambio nome al comando proof
\renewcommand*{\proofname}{Dimostrazione}

\begin{document}
\include{frontmatter/main.tex}

\tableofcontents

\include{quote/main.tex}

%% Aggiungere i capitoli qui sotto
\include{capitoli/richiami/main.tex}
\include{capitoli/immagini/main.tex}

\end{document}


\end{document}

\documentclass[a4paper,12 pt]{report}
\usepackage[T1]{fontenc}
\usepackage[utf8]{inputenc}
\usepackage{lmodern}
\usepackage{listings}

\usepackage{float}
\usepackage{subcaption}
\usepackage{wrapfig}
\usepackage{fancyhdr}
\usepackage{amsthm}

\usepackage{pgfplots}
\setlength {\marginparwidth }{2cm}
\usepackage{todonotes}
\newcommand{\TODO}[2][]
{\todo[size=\scriptsize, color=red, #1]{#2}}


asdf
\pgfplotsset{compat=1.18}
% forza le footnote a stare il più in basso possibile
\usepackage[bottom]{footmisc}


%% STILE LISTINGS
%%aaa
\usepackage{xcolor}

\definecolor{codegreen}{rgb}{0,0.6,0}
\definecolor{codegray}{rgb}{0.5,0.5,0.5}
\definecolor{codepurple}{rgb}{0.58,0,0.82}
\definecolor{backcolour}{rgb}{0.95,0.95,0.92}

\lstdefinestyle{mystyle}{
    backgroundcolor=\color{backcolour},   
    commentstyle=\color{codegreen},
    keywordstyle=\color{magenta},
    numberstyle=\tiny\color{codegray},
    stringstyle=\color{codepurple},
    basicstyle=\ttfamily\footnotesize,
    breakatwhitespace=false,         
    breaklines=true,                 
    captionpos=b,                    
    keepspaces=true,                 
    numbers=left,                    
    numbersep=5pt,                  
    showspaces=false,                
    showstringspaces=false,
    showtabs=false,                  
    tabsize=2
}

\lstset{style=mystyle}

%% -----

% mostra le subsubsection nell'indice
\setcounter{tocdepth}{3}
\setcounter{secnumdepth}{3}

% Resetta la numerazione dei chapter quando
% una nuova part viene creata
\makeatletter
\@addtoreset{chapter}{part}
\makeatother

% Rimuove l'indentazione quando si crea un nuovo paragrafo
\setlength{\parindent}{0pt}

% footer
\pagestyle{fancyplain}
% rimuove la riga nell'header
\fancyhf{} % sets both header and footer to nothing
\renewcommand{\headrulewidth}{0pt}
\fancyfoot[L]{\href{https://github.com/Typing-Monkeys/AppuntiUniversita}{Typing Monkeys}}
\fancyfoot[C]{\emoji{gorilla}}
\fancyfoot[R]{\thepage}

% configurazione emoji
\usepackage{fontspec}
\usepackage{emoji}
\setemojifont{NotoColorEmoji.ttf}[Path=/usr/share/fonts/truetype/noto/]

\newtheorem{definition}{Definizione}
\newtheorem{lemma}{Lemma}
\newtheorem{theorem}{Teorema}
\newtheorem{corollary}{Corollario}

%% cambio nome al comando proof
\renewcommand*{\proofname}{Dimostrazione}

\begin{document}
\documentclass[a4paper,12 pt]{report}
\usepackage[T1]{fontenc}
\usepackage[utf8]{inputenc}
\usepackage{lmodern}
\usepackage{listings}

\usepackage{float}
\usepackage{subcaption}
\usepackage{wrapfig}
\usepackage{fancyhdr}
\usepackage{amsthm}

\usepackage{pgfplots}
\setlength {\marginparwidth }{2cm}
\usepackage{todonotes}
\newcommand{\TODO}[2][]
{\todo[size=\scriptsize, color=red, #1]{#2}}


asdf
\pgfplotsset{compat=1.18}
% forza le footnote a stare il più in basso possibile
\usepackage[bottom]{footmisc}


%% STILE LISTINGS
%%aaa
\usepackage{xcolor}

\definecolor{codegreen}{rgb}{0,0.6,0}
\definecolor{codegray}{rgb}{0.5,0.5,0.5}
\definecolor{codepurple}{rgb}{0.58,0,0.82}
\definecolor{backcolour}{rgb}{0.95,0.95,0.92}

\lstdefinestyle{mystyle}{
    backgroundcolor=\color{backcolour},   
    commentstyle=\color{codegreen},
    keywordstyle=\color{magenta},
    numberstyle=\tiny\color{codegray},
    stringstyle=\color{codepurple},
    basicstyle=\ttfamily\footnotesize,
    breakatwhitespace=false,         
    breaklines=true,                 
    captionpos=b,                    
    keepspaces=true,                 
    numbers=left,                    
    numbersep=5pt,                  
    showspaces=false,                
    showstringspaces=false,
    showtabs=false,                  
    tabsize=2
}

\lstset{style=mystyle}

%% -----

% mostra le subsubsection nell'indice
\setcounter{tocdepth}{3}
\setcounter{secnumdepth}{3}

% Resetta la numerazione dei chapter quando
% una nuova part viene creata
\makeatletter
\@addtoreset{chapter}{part}
\makeatother

% Rimuove l'indentazione quando si crea un nuovo paragrafo
\setlength{\parindent}{0pt}

% footer
\pagestyle{fancyplain}
% rimuove la riga nell'header
\fancyhf{} % sets both header and footer to nothing
\renewcommand{\headrulewidth}{0pt}
\fancyfoot[L]{\href{https://github.com/Typing-Monkeys/AppuntiUniversita}{Typing Monkeys}}
\fancyfoot[C]{\emoji{gorilla}}
\fancyfoot[R]{\thepage}

% configurazione emoji
\usepackage{fontspec}
\usepackage{emoji}
\setemojifont{NotoColorEmoji.ttf}[Path=/usr/share/fonts/truetype/noto/]

\newtheorem{definition}{Definizione}
\newtheorem{lemma}{Lemma}
\newtheorem{theorem}{Teorema}
\newtheorem{corollary}{Corollario}

%% cambio nome al comando proof
\renewcommand*{\proofname}{Dimostrazione}

\begin{document}
\include{frontmatter/main.tex}

\tableofcontents

\include{quote/main.tex}

%% Aggiungere i capitoli qui sotto
\include{capitoli/richiami/main.tex}
\include{capitoli/immagini/main.tex}

\end{document}


\tableofcontents

\documentclass[a4paper,12 pt]{report}
\usepackage[T1]{fontenc}
\usepackage[utf8]{inputenc}
\usepackage{lmodern}
\usepackage{listings}

\usepackage{float}
\usepackage{subcaption}
\usepackage{wrapfig}
\usepackage{fancyhdr}
\usepackage{amsthm}

\usepackage{pgfplots}
\setlength {\marginparwidth }{2cm}
\usepackage{todonotes}
\newcommand{\TODO}[2][]
{\todo[size=\scriptsize, color=red, #1]{#2}}


asdf
\pgfplotsset{compat=1.18}
% forza le footnote a stare il più in basso possibile
\usepackage[bottom]{footmisc}


%% STILE LISTINGS
%%aaa
\usepackage{xcolor}

\definecolor{codegreen}{rgb}{0,0.6,0}
\definecolor{codegray}{rgb}{0.5,0.5,0.5}
\definecolor{codepurple}{rgb}{0.58,0,0.82}
\definecolor{backcolour}{rgb}{0.95,0.95,0.92}

\lstdefinestyle{mystyle}{
    backgroundcolor=\color{backcolour},   
    commentstyle=\color{codegreen},
    keywordstyle=\color{magenta},
    numberstyle=\tiny\color{codegray},
    stringstyle=\color{codepurple},
    basicstyle=\ttfamily\footnotesize,
    breakatwhitespace=false,         
    breaklines=true,                 
    captionpos=b,                    
    keepspaces=true,                 
    numbers=left,                    
    numbersep=5pt,                  
    showspaces=false,                
    showstringspaces=false,
    showtabs=false,                  
    tabsize=2
}

\lstset{style=mystyle}

%% -----

% mostra le subsubsection nell'indice
\setcounter{tocdepth}{3}
\setcounter{secnumdepth}{3}

% Resetta la numerazione dei chapter quando
% una nuova part viene creata
\makeatletter
\@addtoreset{chapter}{part}
\makeatother

% Rimuove l'indentazione quando si crea un nuovo paragrafo
\setlength{\parindent}{0pt}

% footer
\pagestyle{fancyplain}
% rimuove la riga nell'header
\fancyhf{} % sets both header and footer to nothing
\renewcommand{\headrulewidth}{0pt}
\fancyfoot[L]{\href{https://github.com/Typing-Monkeys/AppuntiUniversita}{Typing Monkeys}}
\fancyfoot[C]{\emoji{gorilla}}
\fancyfoot[R]{\thepage}

% configurazione emoji
\usepackage{fontspec}
\usepackage{emoji}
\setemojifont{NotoColorEmoji.ttf}[Path=/usr/share/fonts/truetype/noto/]

\newtheorem{definition}{Definizione}
\newtheorem{lemma}{Lemma}
\newtheorem{theorem}{Teorema}
\newtheorem{corollary}{Corollario}

%% cambio nome al comando proof
\renewcommand*{\proofname}{Dimostrazione}

\begin{document}
\include{frontmatter/main.tex}

\tableofcontents

\include{quote/main.tex}

%% Aggiungere i capitoli qui sotto
\include{capitoli/richiami/main.tex}
\include{capitoli/immagini/main.tex}

\end{document}


%% Aggiungere i capitoli qui sotto
\documentclass[a4paper,12 pt]{report}
\usepackage[T1]{fontenc}
\usepackage[utf8]{inputenc}
\usepackage{lmodern}
\usepackage{listings}

\usepackage{float}
\usepackage{subcaption}
\usepackage{wrapfig}
\usepackage{fancyhdr}
\usepackage{amsthm}

\usepackage{pgfplots}
\setlength {\marginparwidth }{2cm}
\usepackage{todonotes}
\newcommand{\TODO}[2][]
{\todo[size=\scriptsize, color=red, #1]{#2}}


asdf
\pgfplotsset{compat=1.18}
% forza le footnote a stare il più in basso possibile
\usepackage[bottom]{footmisc}


%% STILE LISTINGS
%%aaa
\usepackage{xcolor}

\definecolor{codegreen}{rgb}{0,0.6,0}
\definecolor{codegray}{rgb}{0.5,0.5,0.5}
\definecolor{codepurple}{rgb}{0.58,0,0.82}
\definecolor{backcolour}{rgb}{0.95,0.95,0.92}

\lstdefinestyle{mystyle}{
    backgroundcolor=\color{backcolour},   
    commentstyle=\color{codegreen},
    keywordstyle=\color{magenta},
    numberstyle=\tiny\color{codegray},
    stringstyle=\color{codepurple},
    basicstyle=\ttfamily\footnotesize,
    breakatwhitespace=false,         
    breaklines=true,                 
    captionpos=b,                    
    keepspaces=true,                 
    numbers=left,                    
    numbersep=5pt,                  
    showspaces=false,                
    showstringspaces=false,
    showtabs=false,                  
    tabsize=2
}

\lstset{style=mystyle}

%% -----

% mostra le subsubsection nell'indice
\setcounter{tocdepth}{3}
\setcounter{secnumdepth}{3}

% Resetta la numerazione dei chapter quando
% una nuova part viene creata
\makeatletter
\@addtoreset{chapter}{part}
\makeatother

% Rimuove l'indentazione quando si crea un nuovo paragrafo
\setlength{\parindent}{0pt}

% footer
\pagestyle{fancyplain}
% rimuove la riga nell'header
\fancyhf{} % sets both header and footer to nothing
\renewcommand{\headrulewidth}{0pt}
\fancyfoot[L]{\href{https://github.com/Typing-Monkeys/AppuntiUniversita}{Typing Monkeys}}
\fancyfoot[C]{\emoji{gorilla}}
\fancyfoot[R]{\thepage}

% configurazione emoji
\usepackage{fontspec}
\usepackage{emoji}
\setemojifont{NotoColorEmoji.ttf}[Path=/usr/share/fonts/truetype/noto/]

\newtheorem{definition}{Definizione}
\newtheorem{lemma}{Lemma}
\newtheorem{theorem}{Teorema}
\newtheorem{corollary}{Corollario}

%% cambio nome al comando proof
\renewcommand*{\proofname}{Dimostrazione}

\begin{document}
\include{frontmatter/main.tex}

\tableofcontents

\include{quote/main.tex}

%% Aggiungere i capitoli qui sotto
\include{capitoli/richiami/main.tex}
\include{capitoli/immagini/main.tex}

\end{document}

\documentclass[a4paper,12 pt]{report}
\usepackage[T1]{fontenc}
\usepackage[utf8]{inputenc}
\usepackage{lmodern}
\usepackage{listings}

\usepackage{float}
\usepackage{subcaption}
\usepackage{wrapfig}
\usepackage{fancyhdr}
\usepackage{amsthm}

\usepackage{pgfplots}
\setlength {\marginparwidth }{2cm}
\usepackage{todonotes}
\newcommand{\TODO}[2][]
{\todo[size=\scriptsize, color=red, #1]{#2}}


asdf
\pgfplotsset{compat=1.18}
% forza le footnote a stare il più in basso possibile
\usepackage[bottom]{footmisc}


%% STILE LISTINGS
%%aaa
\usepackage{xcolor}

\definecolor{codegreen}{rgb}{0,0.6,0}
\definecolor{codegray}{rgb}{0.5,0.5,0.5}
\definecolor{codepurple}{rgb}{0.58,0,0.82}
\definecolor{backcolour}{rgb}{0.95,0.95,0.92}

\lstdefinestyle{mystyle}{
    backgroundcolor=\color{backcolour},   
    commentstyle=\color{codegreen},
    keywordstyle=\color{magenta},
    numberstyle=\tiny\color{codegray},
    stringstyle=\color{codepurple},
    basicstyle=\ttfamily\footnotesize,
    breakatwhitespace=false,         
    breaklines=true,                 
    captionpos=b,                    
    keepspaces=true,                 
    numbers=left,                    
    numbersep=5pt,                  
    showspaces=false,                
    showstringspaces=false,
    showtabs=false,                  
    tabsize=2
}

\lstset{style=mystyle}

%% -----

% mostra le subsubsection nell'indice
\setcounter{tocdepth}{3}
\setcounter{secnumdepth}{3}

% Resetta la numerazione dei chapter quando
% una nuova part viene creata
\makeatletter
\@addtoreset{chapter}{part}
\makeatother

% Rimuove l'indentazione quando si crea un nuovo paragrafo
\setlength{\parindent}{0pt}

% footer
\pagestyle{fancyplain}
% rimuove la riga nell'header
\fancyhf{} % sets both header and footer to nothing
\renewcommand{\headrulewidth}{0pt}
\fancyfoot[L]{\href{https://github.com/Typing-Monkeys/AppuntiUniversita}{Typing Monkeys}}
\fancyfoot[C]{\emoji{gorilla}}
\fancyfoot[R]{\thepage}

% configurazione emoji
\usepackage{fontspec}
\usepackage{emoji}
\setemojifont{NotoColorEmoji.ttf}[Path=/usr/share/fonts/truetype/noto/]

\newtheorem{definition}{Definizione}
\newtheorem{lemma}{Lemma}
\newtheorem{theorem}{Teorema}
\newtheorem{corollary}{Corollario}

%% cambio nome al comando proof
\renewcommand*{\proofname}{Dimostrazione}

\begin{document}
\include{frontmatter/main.tex}

\tableofcontents

\include{quote/main.tex}

%% Aggiungere i capitoli qui sotto
\include{capitoli/richiami/main.tex}
\include{capitoli/immagini/main.tex}

\end{document}


\end{document}


\end{document}


\end{document}
